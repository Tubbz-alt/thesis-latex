\section{Fluctuation strength on real sound: motorbike exhaust and marimba
    tremolo}

The paper by \citet{Miura2013Fluctuation} proposes a modified model of
fluctuation strength to analyze motorbike exhaust and marimba tremolo sounds.
The model is based on a multiple regression composed of Zwicker's fluctuation
strength model value, and several other factors depending on the sound
analyzed. The most important details of this study are presented below:

\begin{itemize}
    \item The method uses information about the fluctuating waveform to extract
        parameters from it and use them to estimate the fluctuation strength.
    \item They use Zwicker's amplitude-modulated tones fluctuation strenght
        model as the reference point for comparisons.
    \item When analizing the fluctuation strength on a mandolin, two types of
        fluctuation where defined:
        \begin{itemize}
            \item 1st fluctuations: the fluctuation arising from the plucking of
                the strings.
            \item 2nd fluctuations: variation in the onset time and amplitude
                due to inconsistencies in the performance of the instrument.
        \end{itemize}
    \item The correlation with the fluctuation strength reported scores was
        higher for the proposed model than for Zwicker's model, for the mandolin
        tremolo case.
    \item Then they expand the analysis to marimba tremolo, which amplitude
        envelope is also dependent on the stifness of the mallets used with the
        marimba. To accomodate for this, they introduce what they call 3rd
        fluctuations parameters, which consists of several features regarding
        the amplitude envelope such as slope for rising and falling, acoustic
        power, etc.
    \item They use a magnitude estimation procedure to obtain the fluctuation
        strength ratings for the sounds used.
    \item They use synthesized marimba sound to manipulate easily the amplitude
        envelope parameters when generating the sample sounds.
    \item Again, by using the measures from the 1st, 2nd and 3rd fluctuation a
        higher correlation coefficient is attained in comparison with the case
        when only the base Zwicker value is used.
    \item To assess whether using 3rd fluctuations make a difference regarding
        the expected fluctuation strength value, motorcycle exhaust noises where
        then analyzed. These have the particularity that, on a set of pulses,
        each pulse shape differs from the others. 
    \item They found that by including the information regarding the specified
        features better correlation coefficients could be obtained between the
        participants answers and the model.
\end{itemize}
