\documentclass{report}
\usepackage[media=screen]{mystyle}

\begin{document}

Rodrigo: \textbf{Modeling the sensation of fluctuation strength}

Auditory perceptual attributes constitute the basic dimensions in which a sound
event can be decomposed from a perceptual point of view. Among these attributes
is the fluctuation strength attribute, which corresponds to the sensation that
arises when a sound presents a slow amplitude envelope. Another related
perceptual attribute to fluctuation strength is roughness, which from a physical
point of view is very similar, while from a perceptual perspective it gives a
different sensation altogether.

Most of the research of the perceptual effects of modulating envelopes has been
centered on the roughness perceptual attribute, leaving the fluctuation strength
attribute less explored. Computational models have been developed for roughness,
whereas for fluctuation strength they have not. Furthermore, there exist
methodological issues regarding the experimental procedures involving
fluctuation strength, due to confusion between the two perceptual attributes and
to unfamiliarity to fluctuation strength. As so, the ongoing research has a
two-fold objective: (1) to come up with a clear experimental procedure on how
to obtain reliable fluctuation strength judgments; (2) to accommodate existing
roughness model to the fluctuation strength phenomena.

\end{document}
