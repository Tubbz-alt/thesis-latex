\documentclass{report}
\usepackage[media=screen]{mystyle}

\begin{document}

Rodrigo: \textbf{Modeling the sensation of fluctuation strength}

Auditory perceptual attributes constitute the basic dimensions in which a sound
event can be decomposed from a perceptual point of view. Among these, the
perceptual attributes of roughness and fluctuation strength are related to the
sensation produced by modulations in amplitude and frequency. Most of the
literature on the perceptual effects of modulating envelopes has been centered
on roughness, leaving fluctuation strength less explored. Computational models
have been developed for roughness, whereas for fluctuation strength they have
not. Furthermore, there exist methodological issues regarding the experimental
procedures involving fluctuation strength, due to confusion between the two
perceptual attributes and to unfamiliarity to fluctuation strength. As so, the
ongoing research has a two-fold objective: (1) to come up with a clear
experimental procedure on how to obtain reliable fluctuation strength judgments;
(2) to accommodate existing roughness model to the fluctuation strength
phenomena.

\end{document}
