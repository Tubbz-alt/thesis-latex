\documentclass{beamer}

\mode<presentation>
\usetheme{Pittsburgh}
\usecolortheme{whale}
\setbeamertemplate{section in toc}{\inserttocsectionnumber.~\inserttocsection}
\setbeamertemplate{subsection in toc}[subsections numbered]
\setbeamertemplate{enumerate items}[default]
\setbeamerfont{caption}{size=\tiny}
\beamertemplatenavigationsymbolsempty{}

\newcommand{\nofootline}{\setbeamertemplate{footline}{}}
\newcommand{\mainfootline}{
  \setbeamertemplate{footline}{
  \hspace*{\fill}
    \insertframenumber{}/\inserttotalframenumber{}
    \hspace*{\fill}
    \vspace{2mm}
  }
}

\usepackage[utf8]{inputenc}
\usepackage[T1]{fontenc}
\usepackage[american]{babel}
\usepackage{cmbright}
\usepackage[scaled=0.85]{beramono}
\usepackage{multimedia}
\usepackage{hyperref}
\usepackage{graphicx}
\usepackage{silence}
\usepackage{csquotes}
\usepackage[backend=biber]{biblatex}
\bibliography{/Users/rodrigo/Library/texmf/tex/latex/References.bib}
\WarningFilter{biblatex}{Patching footnotes failed}
\DeclareFieldFormat*{citetitle}{#1}
\DeclareFieldFormat*{title}{#1}

\title{Modeling the sensation of fluctuation strength}
\author{Rodrigo García León
  \texorpdfstring{\\ M.Sc Student Human-Technology Interaction
  \\ \texttt{r.garcia.leon@student.tue.nl}}{}}
\institute{Department of Industrial Engineering \& Innovation Sciences
  \texorpdfstring{\\Eindhoven University of Technology}{}}
\date{\today}

\newcommand{\playbutton}{\raisebox{-1ex}{\hbox%
  {\includegraphics[height=18pt]{play}}}}
\newcommand{\addsound}[1]{
  \movie[label=#1]{}{#1.wav}
  \hyperlinkmovie{#1}{\playbutton}
}
\newcommand{\addcite}[1]{%
  \citeauthor{#1}
  (\citeyear{#1}),
  \citetitle{#1}
}

\AtBeginSection[] {
  {
    \nofootline{}
    \begin{frame}
      \frametitle{Outline}
      \tableofcontents[currentsection]
    \end{frame}
  }
  \addtocounter{framenumber}{-1}
}

\mainfootline{}

\begin{document}

{\nofootline{} \frame{\titlepage}}

\section*{Outline}
{
  \nofootline{}
  \begin{frame}
    \frametitle{Outline}
    \tableofcontents
  \end{frame}
}

\section{Introduction}

\setcounter{framenumber}{0}

\subsection{Perceptual Attributes}
\begin{frame}
  \frametitle{Perceptual Attributes}
  \begin{itemize}
    \item Discernible dimensions in which an auditory event can be decomposed
    \pause{}
    \item Derived from physical characteristics of sounds (frequency, sound
      pressure level, etc.)
    \pause{}
    \item Examples of them are: loudness, sharpness, roughness, fluctuations
      strength; among others.
    \pause{}
    \item Allow to understand auditory events from a perceptual point of view
  \end{itemize}
\end{frame}

\subsection{Fluctuation Strength}
\begin{frame}
  \frametitle{Fluctuation Strength}
  \begin{itemize}
    \item Sensation that arises from modulated sounds with a slowly varying
      envelope (i.e., a modulation frequency $f_m < 20$ Hz)
    \pause{}
    \item Envelope fluctuation can be amplitude-modulated (AM) or
      frequency-modulated (FM)
  \end{itemize}

  \pause{}

  \begin{columns}
    \begin{column}{0.175\textwidth}
      \hbox{\hbox to 12mm{AM tone} \addsound{am}} \\
      \medskip
      { \scriptsize
        $f_m = 4$ [Hz] \\ $f_c = 1$ [kHz] \\
        SPL = 70 [dB] \\ $m_d = 40$ [dB] \\}
    \end{column}
    \begin{column}{0.175\textwidth}
      \includegraphics[height=0.25\textheight]{am_time}
    \end{column}
    \begin{column}{0.175\textwidth}
      \includegraphics[height=0.25\textheight]{am_frequency}
    \end{column}
  \end{columns}

  \pause{}

  \vspace{2mm}

  \begin{columns}
    \begin{column}{0.175\textwidth}
      \hbox{\hbox to 12mm{FM tone} \addsound{fm}} \\
      \medskip
      { \scriptsize
        $f_m = 4$ [Hz] \\ $f_c = 1.5$ [kHz] \\
        SPL = 70 [dB] \\ $d_f = 700$ [Hz] \\}
    \end{column}
    \begin{column}{0.175\textwidth}
      \includegraphics[height=0.25\textheight]{fm_time}
    \end{column}
    \begin{column}{0.175\textwidth}
      \includegraphics[height=0.25\textheight]{fm_frequency}
    \end{column}
  \end{columns}
\end{frame}

\begin{frame}
  \frametitle{Fluctuation Strength and Roughness}
  \begin{itemize}
    \item Above $f_m > 20$ Hz the fluctuation strength sensation diminishes
      and the roughness sensation begins to take effect
  \end{itemize}

  \pause{}

  \begin{columns}
    \begin{column}{0.175\textwidth}
      \hbox{\hbox to 24mm{Fluctuating tone} \addsound{am}} \\
      \medskip
      { \scriptsize
        $f_m = 4$ [Hz] \\ $f_c = 1$ [kHz] \\
        SPL = 70 [dB] \\ $m_d = 40$ [dB] \\}
    \end{column}
    \begin{column}{0.175\textwidth}
      \includegraphics[height=0.25\textheight]{am_time}
    \end{column}
    \begin{column}{0.175\textwidth}
      \includegraphics[height=0.25\textheight]{am_frequency}
    \end{column}
  \end{columns}

  \pause{}

  \vspace{2mm}

  \begin{columns}
    \begin{column}{0.175\textwidth}
      \hbox{\hbox to 16mm{Rough tone} \addsound{rough}} \\
      \medskip
      { \scriptsize
        $f_m = 70$ [Hz] \\ $f_c = 1.5$ [kHz] \\
        SPL = 70 [dB] \\ $m_d = 40$ [dB] \\}
    \end{column}
    \begin{column}{0.175\textwidth}
      \includegraphics[height=0.25\textheight]{rough_time}
    \end{column}
    \begin{column}{0.175\textwidth}
      \includegraphics[height=0.25\textheight]{rough_frequency}
    \end{column}
  \end{columns}
\end{frame}

\begin{frame}
  \frametitle{Band-pass Characteristic}
  \begin{figure}
    \includegraphics[width=\textwidth]{fs_vs_md}
    \caption{Fluctuation strength as a function of modulation frequency for: (a)
      amplitude-modulated broad-band noise, (b) amplitude-modulated tones and
      (c) frequency-modulated tones; as presented by
      \addcite{Fastl2007Psychoacoustics}}
  \end{figure}
\end{frame}

\subsection{Current Research}
\begin{frame}
  \frametitle{Importance}
  \begin{itemize}
    \item Possible relation with the speech system
    \pause{}
    \begin{itemize}
      \item Maximum fluctuation strength value occurs around $f_m = 4$ Hz
      \pause{}
      \item Humans beings utter syllables at a rate of 4 syllables per second
    \end{itemize}
    \pause{}
    \item Influences pleasantness of sounds
    \pause{}
    \begin{itemize}
      \item Functionally (alarm signals)
      \pause{}
      \item Musicality (instruments enjoyableness)
    \end{itemize}
  \end{itemize}
\end{frame}

\begin{frame}
  \frametitle{Rationale for Research}
  \begin{itemize}
    \item Methodological issues in past studies
    \pause{}
    \begin{itemize}
      \item Unfamiliarity with fluctuation strength
      \pause{}
      \item Confusion between roughness and fluctuation strength
    \end{itemize}
    \pause{}
    \item Not widely available work when it comes to modeling the fluctuation
      strength sensation
  \end{itemize}
\end{frame}

\begin{frame}
  \frametitle{Objectives}
  \begin{itemize}
    \item Formulate a clear methodological procedure to eliminate problems
      found in past studies
    \pause{}
    \item Propose a fluctuation strength model
    \pause{}
    \begin{itemize}
      \item Based on existing roughness models
      \pause{}
      \item Adjusted to collected experimental data
    \end{itemize}
  \end{itemize}
\end{frame}

\section{Experimental Design}

\begin{frame}
  \frametitle{General Details}
  \begin{itemize}
    \item Two conditions: AM tones and FM tones
    \pause{}
    \item 24 participants assigned to one of the two conditions
    \pause{}
    \item Two parts
      \begin{enumerate}
        \item Training phase
        \item Experimental phase
      \end{enumerate}
    \pause{}
    \item Duration of approximately 60 minutes
  \end{itemize}
\end{frame}

\subsection{Training Phase}
\begin{frame}
  \frametitle{Training Phase}
  \begin{enumerate}
    \item Comparison between stimuli: the following types of stimuli were
      presented sequentially to participants (both for AM and FM):
      \begin{itemize}
        \item Pure tone
        \item Tone with low value of fluctuation
        \item Tone with high value of fluctuation
        \item Tone with low value of roughness
        \item Tone with high value of roughness
      \end{itemize}
    \pause{}
    \item Continuous stimuli: two stimuli (AM and FM), composed of stimuli
      presenting all the values of modulation frequency used in the experiment,
      were presented to the participants
    \pause{}
    \item Test section: composed of 4 trials, designed to familiarize
      participants with the experimental procedure and interface
  \end{enumerate}
\end{frame}

\subsection{Experimental Phase}
\begin{frame}
  \frametitle{Experimental Phase}
  \begin{itemize}
    \item Magnitude estimation procedure
    \begin{itemize}
      \item Trial composed of a pair, a standard or anchor with a stimulus,
        separated by 800 ms of silence
      \item Two standards
      \item Four repetitions per pair
      \item Pairs presentation was randomized
    \end{itemize}
    \pause{}
    \item Four parametric variations
    \begin{itemize}
      \item Modulation frequency ($f_m$)
      \item Center frequency ($f_c$)
      \item Sound pressure level (SPL)
      \item Modulation depth ($m_d$) for AM tones;\\Frequency deviation ($d_f$)
        for FM tones
    \end{itemize}
  \end{itemize}
\end{frame}

\begin{frame}
  \frametitle{APEX Software Platform}
  \centering
  \includegraphics[width=\textwidth]{apex}
  \vspace{2mm}
  \addsound{pair}
\end{frame}

\subsection{Results}
\begin{frame}
  \frametitle{Results --- AM Tones}
  \begin{columns}
    \begin{column}{0.5\textwidth}
      \includegraphics[width=0.9\textwidth]{AM-fm_all_standards}
    \end{column}
    \begin{column}{0.5\textwidth}
      \includegraphics[width=0.9\textwidth]{AM-fc_all_standards}
    \end{column}
  \end{columns}

  \vspace{2mm}

  \begin{columns}
    \begin{column}{0.5\textwidth}
      \includegraphics[width=0.9\textwidth]{AM-SPL_all_standards}
    \end{column}
    \begin{column}{0.5\textwidth}
      \includegraphics[width=0.9\textwidth]{AM-md_all_standards}
    \end{column}
  \end{columns}

\end{frame}

\begin{frame}
  \frametitle{Results --- FM Tones}
  \begin{columns}
    \begin{column}{0.5\textwidth}
      \includegraphics[width=0.9\textwidth]{FM-fm_all_standards}
    \end{column}
    \begin{column}{0.5\textwidth}
      \includegraphics[width=0.9\textwidth]{FM-fc_all_standards}
    \end{column}
  \end{columns}

  \vspace{2mm}

  \begin{columns}
    \begin{column}{0.5\textwidth}
      \includegraphics[width=0.9\textwidth]{FM-SPL_all_standards}
    \end{column}
    \begin{column}{0.5\textwidth}
      \includegraphics[width=0.9\textwidth]{FM-df_all_standards}
    \end{column}
  \end{columns}

\end{frame}

\section{Model Development}

\subsection{Roughness Model}
\begin{frame}
  \frametitle{Roughness Model}
  \begin{figure}
    \centering
    \includegraphics[height=0.7\textheight]{model}
    \caption{Structure of roughness model, as presented by
      \addcite{daniel1997psychoacoustical}}
  \end{figure}
\end{frame}

\section*{Credits}
{
  \nofootline{}
  \begin{frame}
    \frametitle{Credits}
    play \playbutton{} by Mike Ashley from the Noun Project
  \end{frame}
}
\addtocounter{framenumber}{-1}

\end{document}
