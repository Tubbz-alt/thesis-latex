\documentclass{beamer}

\mode<presentation>
\usetheme{Pittsburgh}
\usecolortheme{whale}
\setbeamertemplate{section in toc}{\inserttocsectionnumber.~\inserttocsection}
\setbeamertemplate{subsection in toc}[subsections numbered]
\setbeamertemplate{enumerate items}[default]
\setbeamerfont{caption}{size=\tiny}
\beamertemplatenavigationsymbolsempty{}

\newcommand{\nofootline}{\setbeamertemplate{footline}{}}
\newcommand{\mainfootline}{
  \setbeamertemplate{footline}{
  \hspace*{\fill}
    \insertframenumber{}/\inserttotalframenumber{}
    \hspace*{\fill}
    \vspace{2mm}
  }
}

\usepackage[utf8]{inputenc}
\usepackage[T1]{fontenc}
\usepackage[american]{babel}
\usepackage{cmbright}
\usepackage[scaled=0.85]{beramono}
\usepackage{multimedia}
\usepackage{hyperref}
\usepackage{graphicx}
\usepackage{tikz}
\usepackage{silence}
\usepackage{csquotes}
\usepackage[backend=biber]{biblatex}
\bibliography{/Users/rodrigo/Library/texmf/tex/latex/References.bib}
\WarningFilter{biblatex}{Patching footnotes failed}
\DeclareFieldFormat*{citetitle}{#1}
\DeclareFieldFormat*{title}{#1}

\graphicspath{ {img/} }

\title{Modeling the sensation of fluctuation strength}
\author{Rodrigo García León
  \texorpdfstring{\\ M.Sc Student Human-Technology Interaction
  \\ \texttt{r.garcia.leon@student.tue.nl}}{}}
\institute{Department of Industrial Engineering \& Innovation Sciences
  \texorpdfstring{\\Eindhoven University of Technology}{}}
\date{\today}

\newcommand{\playbutton}{\raisebox{-1ex}{\hbox%
  {\includegraphics[height=18pt]{play}}}}
\newcommand{\addsound}[1]{
  \movie[label=#1]{}{snd/#1.wav}
  \hyperlinkmovie{#1}{\playbutton}
}
\newcommand{\addcite}[1]{%
  \citeauthor{#1}
  (\citeyear{#1}),
  \citetitle{#1}
}

\AtBeginSection[] {
  {
    \nofootline{}
    \begin{frame}
      \frametitle{Outline}
      \tableofcontents[currentsection]
    \end{frame}
  }
  \addtocounter{framenumber}{-1}
}

\mainfootline{}

\begin{document}

{\nofootline{} \frame{\titlepage}}

\section*{Outline}
{
  \nofootline{}
  \begin{frame}
    \frametitle{Outline}
    \tableofcontents
  \end{frame}
}

\section{Introduction}

\setcounter{framenumber}{0}

\subsection{Perceptual Attributes}
\begin{frame}
  \frametitle{Perceptual Attributes}
  \begin{itemize}
    \item Discernible dimensions in which an auditory event can be decomposed
    \pause{}
    \item Derived from physical characteristics of sounds (frequency, sound
      pressure level, etc.)
    \pause{}
    \item Examples of them are: loudness, sharpness, roughness, fluctuations
      strength; among others.
    \pause{}
    \item Allow to understand auditory events from a perceptual point of view
  \end{itemize}
\end{frame}

\subsection{Fluctuation Strength}
\begin{frame}
  \frametitle{Fluctuation Strength}
  \begin{itemize}
    \item Sensation that arises from modulated sounds with a slowly varying
      envelope (i.e., a modulation frequency $f_m < 20$ Hz)
    \pause{}
    \item Envelope fluctuation can be amplitude-modulated (AM) or
      frequency-modulated (FM)
  \end{itemize}

  \pause{}

  \begin{columns}
    \begin{column}{0.175\textwidth}
      \hbox{\hbox to 12mm{AM tone} \addsound{am}} \\
      \medskip
      { \scriptsize
        $f_m = 4$ [Hz] \\ $f_c = 1$ [kHz] \\
        SPL = 70 [dB] \\ $m_d = 40$ [dB] \\}
    \end{column}
    \begin{column}{0.175\textwidth}
      \includegraphics[height=0.25\textheight]{am_time}
    \end{column}
    \begin{column}{0.175\textwidth}
      \includegraphics[height=0.25\textheight]{am_frequency}
    \end{column}
  \end{columns}

  \pause{}

  \vspace{2mm}

  \begin{columns}
    \begin{column}{0.175\textwidth}
      \hbox{\hbox to 12mm{FM tone} \addsound{fm}} \\
      \medskip
      { \scriptsize
        $f_m = 4$ [Hz] \\ $f_c = 1.5$ [kHz] \\
        SPL = 70 [dB] \\ $d_f = 700$ [Hz] \\}
    \end{column}
    \begin{column}{0.175\textwidth}
      \includegraphics[height=0.25\textheight]{fm_time}
    \end{column}
    \begin{column}{0.175\textwidth}
      \includegraphics[height=0.25\textheight]{fm_frequency}
    \end{column}
  \end{columns}
\end{frame}

\begin{frame}
  \frametitle{Fluctuation Strength and Roughness}
  \begin{itemize}
    \item Above $f_m > 20$ Hz the fluctuation strength sensation diminishes
      and the roughness sensation begins to take effect
  \end{itemize}

  \pause{}

  \begin{columns}
    \begin{column}{0.175\textwidth}
      \hbox{\hbox to 24mm{Fluctuating tone} \addsound{am}} \\
      \medskip
      { \scriptsize
        $f_m = 4$ [Hz] \\ $f_c = 1$ [kHz] \\
        SPL = 70 [dB] \\ $m_d = 40$ [dB] \\}
    \end{column}
    \begin{column}{0.175\textwidth}
      \includegraphics[height=0.25\textheight]{am_time}
    \end{column}
    \begin{column}{0.175\textwidth}
      \includegraphics[height=0.25\textheight]{am_frequency}
    \end{column}
  \end{columns}

  \pause{}

  \vspace{2mm}

  \begin{columns}
    \begin{column}{0.175\textwidth}
      \hbox{\hbox to 16mm{Rough tone} \addsound{rough}} \\
      \medskip
      { \scriptsize
        $f_m = 70$ [Hz] \\ $f_c = 1$ [kHz] \\
        SPL = 70 [dB] \\ $m_d = 40$ [dB] \\}
    \end{column}
    \begin{column}{0.175\textwidth}
      \includegraphics[height=0.25\textheight]{rough_time}
    \end{column}
    \begin{column}{0.175\textwidth}
      \includegraphics[height=0.25\textheight]{rough_frequency}
    \end{column}
  \end{columns}
\end{frame}

\begin{frame}
  \frametitle{Band-pass Characteristic}
  \begin{figure}
    \includegraphics[width=\textwidth]{fs_vs_md}
    \caption{Fluctuation strength as a function of modulation frequency for: (a)
      amplitude-modulated broad-band noise, (b) amplitude-modulated tones and
      (c) frequency-modulated tones; as presented by
      \addcite{Fastl2007Psychoacoustics}}
  \end{figure}
\end{frame}

\subsection{Current Research}
\begin{frame}
  \frametitle{Importance}
  \begin{itemize}
    \item Possible relation with the speech system
    \pause{}
    \begin{itemize}
      \item Maximum fluctuation strength value occurs around $f_m = 4$ Hz
      \pause{}
      \item Humans beings utter syllables at a rate of 4 syllables per second
    \end{itemize}
    \pause{}
    \item Influences pleasantness of sounds
    \pause{}
    \begin{itemize}
      \item Functionally (alarm signals)
      \pause{}
      \item Musicality (sound quality and naturalness)
    \end{itemize}
  \end{itemize}
\end{frame}

\begin{frame}
  \frametitle{Rationale for Research}
  \begin{itemize}
    \item Methodological issues in past studies
    \pause{}
    \begin{itemize}
      \item Unfamiliarity with fluctuation strength
      \pause{}
      \item Confusion between roughness and fluctuation strength
    \end{itemize}
    \pause{}
    \item To our knowledge, there is no publicly available work when it comes to
      modeling the fluctuation strength sensation
  \end{itemize}
\end{frame}

\begin{frame}
  \frametitle{Objectives}
  \begin{itemize}
    \item Formulate a clear methodological procedure to eliminate problems
      found in past studies
    \pause{}
    \item Propose a fluctuation strength model
    \pause{}
    \begin{itemize}
      \item Based on existing roughness models
      \pause{}
      \item Adjusted to collected experimental data
    \end{itemize}
  \end{itemize}
\end{frame}

\section{Experimental Design}

\begin{frame}
  \frametitle{General Details}
  \begin{itemize}
    \item Two conditions: AM tones and FM tones
    \pause{}
    \item 24 participants assigned to one of the two conditions
    \pause{}
    \item Two parts
      \begin{enumerate}
        \item Training phase
        \item Experimental phase
      \end{enumerate}
    \pause{}
    \item Duration of approximately 60 minutes
  \end{itemize}
\end{frame}

\subsection{Training Phase}
\begin{frame}
  \frametitle{Training Phase}
  \begin{enumerate}
    \item Comparison between stimuli: five types of stimuli were presented to
      participants (both for AM and FM)
    \pause{}
      \only<2>{
      \begin{itemize}
        \item Pure tone \addsound{training_1}
        \item Tone with low value of fluctuation \addsound{training_2}
        \item Tone with high value of fluctuation \addsound{training_3}
        \item Tone with low value of roughness \addsound{training_4}
        \item Tone with high value of roughness \addsound{training_5}
      \end{itemize}
      }
    \pause{}
    \item Sequential stimuli: two stimuli (AM and FM), composed of stimuli
      having all the values of modulation frequency used in the experiment, were
      presented to the participants
    \pause{}
    \item Test section: composed of 4 trials, designed to familiarize
      participants with the experimental procedure and interface
  \end{enumerate}
\end{frame}

\subsection{Experimental Phase}
\begin{frame}
  \frametitle{Experimental Phase}
  \begin{itemize}
    \item Magnitude estimation procedure
    \pause{}
    \begin{itemize}
      \item Trial composed of a pair, a standard (reference) with a stimulus,
        separated by 800 ms of silence
      \pause{}
      \item Two standards
      \pause{}
      \item Four repetitions per pair
      \pause{}
      \item Pairs presentation was randomized
    \end{itemize}
    \pause{}
    \item Four parametric variations
    \pause{}
    \begin{itemize}
      \item Modulation frequency ($f_m$)
      \pause{}
      \item Center frequency ($f_c$)
      \pause{}
      \item Sound pressure level (SPL)
      \pause{}
      \item Modulation depth ($m_d$) for AM tones;\\Frequency deviation ($d_f$)
        for FM tones
    \end{itemize}
  \end{itemize}
\end{frame}

\begin{frame}
  \frametitle{APEX Software Platform}
  \centering
  \includegraphics[width=\textwidth]{apex}
  \vspace{2mm}
  \addsound{pair}
\end{frame}

\subsection{Results}
\begin{frame}
  \frametitle{Results --- AM tones, $f_m$}
  \begin{figure}
    \centering
    \begin{minipage}{0.45\textwidth}
      \includegraphics[width=\textwidth]{Fastl2007_AM-fm}
    \end{minipage}
    \hfill
    \begin{minipage}{0.45\textwidth}
      \includegraphics[width=\textwidth]{Garcia2015_AM-fm}
    \end{minipage}
    \caption{Relative fluctuation strength as a function of modulation
    frequency for AM tones --- \citeauthor{Fastl2007Psychoacoustics} (left),
    current research (right)}
  \end{figure}
\end{frame}

\begin{frame}
  \frametitle{Results --- FM tones, $f_m$}
  \begin{figure}
    \centering
    \begin{minipage}{0.45\textwidth}
      \includegraphics[width=\textwidth]{Fastl2007_FM-fm}
    \end{minipage}
    \hfill
    \begin{minipage}{0.45\textwidth}
      \includegraphics[width=\textwidth]{Garcia2015_FM-fm}
    \end{minipage}
    \caption{Relative fluctuation strength as a function of modulation
    frequency for FM tones --- \citeauthor{Fastl2007Psychoacoustics} (left),
    current research (right)}
  \end{figure}
\end{frame}

\section{Model Development}

\begin{frame}
  \frametitle{Roughness Model}
  \begin{columns}
    \begin{column}{0.5\textwidth}
      \begin{itemize}
        \item<1-> Presented by \addcite{daniel1997psychoacoustical}
        \item<2-> Composed of three stages
        \begin{enumerate}
          \item<3-> Peripheral stage
          \item<4-> Modulation depth extraction stage
          \item<5-> Specific roughness stage
        \end{enumerate}
        \item<6-> Based on dependence of roughness on modulation depth
        \begin{itemize}
          \item<7-> $R \sim m^p$
        \end{itemize}
      \end{itemize}
    \end{column}
    \begin{column}{0.5\textwidth}
      \includegraphics[width=\textwidth]{model}
    \end{column}
  \end{columns}
\end{frame}

\subsection{Peripheral Stage}
\begin{frame}
  \frametitle{Peripheral Stage}
  \begin{columns}
    \begin{column}{0.5\textwidth}
    Fluctuation strength model adaptations
      \begin{itemize}
        \item<1-> Separation of input signal into frames
        \item<2-> Outer and middle ear transmission effects
        \item<3-> Critical-band filterbank
      \end{itemize}
    \end{column}
    \begin{column}{0.5\textwidth}
      \begin{tikzpicture}
        \node[anchor=south west,inner sep=0] at (0,0) {
          \includegraphics[width=\textwidth]{model}
        };
        \filldraw[fill=white,draw=white,opacity=0.6] (1mm,1mm) rectangle (53mm,45mm);
        \draw[blue,ultra thick,rounded corners] (1mm,45mm) rectangle (53mm,77mm);
      \end{tikzpicture}
    \end{column}
  \end{columns}
\end{frame}

\subsection{Modulation Depth Extraction Stage}
\begin{frame}
  \frametitle{Modulation Depth Extraction Stage}
  \begin{columns}
    \begin{column}{0.5\textwidth}
      Fluctuation strength model adaptations
      \begin{itemize}
        \item<1-> Extraction of modulation depth per channel
        \begin{itemize}
          \item<2-> RMS (root mean square) value of bandpass filtered signal
            divided by its mean
          \item<3-> ${m_i}^* = \tilde{h}_{BP,i}(t)/h_{0,i}$
        \end{itemize}
        \item<4-> Bandpass filter using $H_i$ parameter
        \begin{itemize}
          \item<5-> Models the dependence on modulation frequency
        \end{itemize}
      \end{itemize}
    \end{column}
    \begin{column}{0.5\textwidth}
      \begin{tikzpicture}
        \node[anchor=south west,inner sep=0] at (0,0) {
          \includegraphics[width=\textwidth]{model}
        };
        \filldraw[fill=white,draw=white,opacity=0.6] (1mm,1mm) rectangle (53mm,26mm);
        \filldraw[fill=white,draw=white,opacity=0.6] (1mm,45mm) rectangle (53mm,77mm);
        \draw[blue,ultra thick,rounded corners] (1mm,26mm) rectangle (53mm,45mm);
      \end{tikzpicture}
    \end{column}
  \end{columns}
\end{frame}

\begin{frame}
  \frametitle{Roughness of AM Tones}
  \begin{figure}
    \centering
    \includegraphics[height=0.5\textheight]{model_roughness}
    \caption{Roughness of AM tones, adapted from
      \addcite{daniel1997psychoacoustical}}
  \end{figure}
\end{frame}

\subsection{Specific Fluctuation Strength Stage}
\begin{frame}
  \frametitle{Specific Fluctuation Strength Stage}
  \begin{columns}
    \begin{column}{0.5\textwidth}
      Fluctuation strength model adaptations
      \begin{itemize}
        \item<1-> Specific fluctuation strength \\
          $f_i = [g(z_i) \cdot {m_i}^* \cdot k_{i-2} \cdot k_i]^2$
        \item<2-> Channel weighting using $g(z_i)$ parameter
        \begin{itemize}
          \item<3-> Models the dependence on center frequency
        \end{itemize}
        \item<4-> Cross correlations among channels $k_i$
        \item<5-> Total fluctuation strength \\
          $F = \displaystyle\sum_{i=1}^{47} f_i$
      \end{itemize}
    \end{column}
    \begin{column}{0.5\textwidth}
      \begin{tikzpicture}
        \node[anchor=south west,inner sep=0] at (0,0) {
          \includegraphics[width=\textwidth]{model}
        };
        \filldraw[fill=white,draw=white,opacity=0.6] (1mm,26mm) rectangle (53mm,77mm);
        \draw[blue,ultra thick,rounded corners] (1mm,1mm) rectangle (53mm,26mm);
      \end{tikzpicture}
    \end{column}
  \end{columns}
\end{frame}

\subsection{Preliminary Results}
\begin{frame}
  \frametitle{Preliminary Results}
  \begin{figure}
    \centering
    \begin{minipage}{0.45\textwidth}
      \includegraphics[width=\textwidth]{model_am}
    \end{minipage}
    \hfill
    \begin{minipage}{0.45\textwidth}
      \includegraphics[width=\textwidth]{model_fm}
    \end{minipage}
    \caption{Relative fluctuation strength as a function of modulation
      frequency --- comparison between experimental data and model output}
  \end{figure}
\end{frame}

\section{Discussion}
\begin{frame}
  \frametitle{Discussion}
  \begin{itemize}
    \item<1-> Training phase helps participants understand the concept of
      fluctuation strength
    \begin{itemize}
      \item<2-> It does not eliminate confusion completely
      \item<3-> Unfamiliarity, forgetfulness, over-thinking
    \end{itemize}
    \item<4-> \citeauthor{Fastl2007Psychoacoustics} data and obtained data are
      qualitatively similar
    \begin{itemize}
      \item<5-> Differences in sensitivity to center frequency and frequency
        spread
    \end{itemize}
    \item<6-> It is possible to adjust \citeauthor{daniel1997psychoacoustical}
      model to the fluctuation strength sensation
    \begin{itemize}
      \item<7-> Model designed specifically for AM tones
      \item<8-> Slow computation due to increased frame size
    \end{itemize}
  \end{itemize}
\end{frame}

\section*{Questions}
{
  \nofootline{}
  \begin{frame}
    \frametitle{Questions}
    \centering
    \vspace*{\fill}
    \includegraphics[height=0.6\textheight]{question}
    \vspace*{\fill}
  \end{frame}
}
\addtocounter{framenumber}{-1}

\section*{Credits}
{
  \nofootline{}
  \begin{frame}
    \frametitle{Credits}
    \begin{itemize}
      \item play \playbutton{} by Mike Ashley from Noun Project
      \item question
        \raisebox{-1ex}{\hbox{\includegraphics[height=18pt]{question}}} by
        Henry Ryder from Noun Project
    \end{itemize}
  \end{frame}
}
\addtocounter{framenumber}{-1}

\end{document}
