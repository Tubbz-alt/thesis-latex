\section{Fluctuation Strength of Mixed Fluctuating Sound Sources}

\begin{itemize}
    \item Fastl's model cannot be extended to combinations of sound sources with
        different modulation frequencies.
    \item Magnitude-estimation procedure.
    \item For broadband noises $\Delta L$ is largely independent of frequency
        because the nonlinearity of the upper slope in masking patterns overlaps
        with the adjacent frequency bands but for AM tones some frequency
        dependence occurs. (?)
    \item A mixed signal of two (or more) AM signals, with different modulation
        frequencies each, exhibit a compound modulation frequency.
    \item It has been demonstrated that the time variance of psychoacoustical
        parameters such as loudness, roughness and fluctuation strength tends to
        show a different global value than the average value of the
        instantaneous ones.
    \item Participants can listen to the 10 reference sounds along with the
        test sound as much as they can. This seems like a mix between
        magnitude-estimation and ``random access'' procedure.
    \item  Particularly, there is the well known problem that fluctuation
        strength is frequently confused with roughness.
    \item Training phase to familiarize subjects with fluctuation strength.
    \item No correlation between the fluctuation strength predicted by the model
        and that obtained from participants.
    \item  The result was that fluctuation strength models for sounds with the
        same modulation frequency, as those available in the literature, are not
        extensible for combined signals modulated with different modulation
        frequencies.
\end{itemize}
