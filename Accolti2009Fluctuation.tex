\section{Fluctuation Strength of Mixed Fluctuating Sound Sources}

The study of \citet{Accolti2009Fluctuation} tries to apply Zwicker's model to
sounds composed of two amplitude-modulated sources. Below the most important
points of their research:

\begin{itemize}
    \item A mixed signal of two (or more) AM signals, with different modulation
        frequencies each, exhibit a compound modulation frequency.
    \item It has been demonstrated that the time variance of psychoacoustical
        parameters such as loudness, roughness and fluctuation strength tends to
        show a different global value than the average value of the
        instantaneous ones. As so, they take into account the variance of the
        compound modulation frequency to compensate for this fact. 
    \item Participants can listen to the 10 reference sounds along with the
        test sound as much as they can. This seems like a mix between
        magnitude-estimation and ``random access'' procedure.
    \item The authors state that it is a well known problem that fluctuation
        strength is frequently confused with roughness. Therefore, they use a
        training phase to familiarize subjects with fluctuation strength.
    \item No correlation between the fluctuation strength predicted by the model
        and that obtained from participants.
    \item  The result was that fluctuation strength models for sounds with the
        same modulation frequency, as those available in the literature, are not
        extensible for combined signals modulated with different modulation
        frequencies.
\end{itemize}
