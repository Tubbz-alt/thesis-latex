\section{Fluctuation strength and temporal masking patterns of
\texorpdfstring{\\}{} amplitude-modulated broadband noise}

\begin{itemize}
    \item Magnitude-estimation procedure.
    \item Temporal masking depth presents a low-pass characteristics as a
        function of modulation frequency.
    \item Slow fluctuations (i.e.\ fluctuation strength) can increase the
        ``pleasantness'' of sounds.
    \item Fastl's fluctuation strength model is based on the temporal masking
        pattern, which is a psychoacoustical measure.
    \item  Two anchors values were used, one eliciting large fluctuation
        strength values (assigned value of 100), and another one for small
        fluctuation strength values (assigned value of 10).
    \item The dependence on modulation frequency was measured with bursts of 4 s
        duration, separated by an interval of 800 ms with pauses of 3 s between
        pairs. (?)
    \item Each pair of stimuli was presented four times. Intra-individual
        differences differed by less than $\pm 10$.
    \item Temporal masking patterns were determined by the Békésy-tracking
        method. (?)
    \item Memory effects arise on low fluctuation strength.
    \item In essence, the model proposed relates fluctuation strength to the
        magnitude of the variation of the temporal masking pattern including a
        weighting factor which depends on variation speed.
    \item Temporal masking patterns are regarded as a useful generally
        applicable intermediate value for the quantitative description of
        hearing sensations, produced by sounds with strong temporal structure.

\end{itemize}
