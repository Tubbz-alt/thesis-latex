\documentclass[a4paper]{article}
\usepackage{mystyle}

\newcommand{\figStds}[3]{
\begin{figure}[ht!]
  \centering
  \includegraphics[height=8cm]{img/#1-#2-standards.eps}
  \caption{#3}
\label{fig:#1-#2}
\end{figure}
}

\newcommand{\indRes}[4]{
  \subsubsection{#1}
  \foreach \i in {#4} {
    \figStds{#2}
      {\i}
      {#3, participant \i}
  }
}

\begin{document}

\customtitle{Pilot Experiments Results}

\section{Introduction} % (fold)
\label{sec:introduction}

This document reports the results of the pilot experiments, intended to validate
the experimental procedure for fluctuation strength using amplitude-modulated
(AM) and frequency-modulated (FM) tones.

% section introduction (end)

\section{Description} % (fold)
\label{sec:description}

They were in total 9 participants in the pilot experiments. Not all participants
were subjected to the same experimental conditions, and not all of them used the
same version of the experiments. Table~\ref{tab:partexpconver} contains the
conditions and version to which the participants were subjected.

\begin{table}
  \centering
  \begin{tabu}{ l l l l }
    \tabucline[1pt]{-}
    Participant & AM    & FM    & Version \\\tabucline[1pt]{-}
    1           & All   & All   & 1 \\ \hline
    2           & All   & None  & 1 \\ \hline
    3           & All   & None  & 1 \\ \hline
    4           & All   & None  & 1 \\ \hline
    5           & fm    & None  & 2 \\ \hline
    6           & fm    & None  & 3 \\ \hline
    7           & None  & fm,fc & 3 \\ \hline
    8           & None  & All   & 3 \\ \hline
    9           & None  & All   & 4 \\ \hline
    \tabucline[1pt]{-}
  \end{tabu}
  \caption{Participants experimental conditions and versions}
\label{tab:partexpconver}
\end{table}

The experimental procedure was varied during the pilot experiment to accommodate
perceived errors during the realization of them. The first version of the
experiment yielded unsatisfactory results with regards to the relation between
fluctuation strength and modulation frequency (participants 2, 3 and 4). The
procedure was then modified, adding two more AM tones with modulation
frequencies of 64 and 128 Hz. The idea behind the addition of these two tones
was that, if participant have stimuli that are clearly rough, it would be easier
for them to distinguish between a fluctuating and a rough tone. Additionally,
FM tones were included in the training, since up to this point only AM tones
were used in the training phase This constitutes the second version of the
experiment.

Participant 5 was the only participant that was subjected to version 2 of the
experiment. Its results did not show any significant improvements with regard
to the confusion between fluctuation strength and roughness. However, by talking
to the participants it was discovered that the instructions during the training
phase were not clear enough, and this may have affected the results of the
experiment. Several participants associated the rate of change of the stimuli
(modulation frequency in this case) to a bigger fluctuation in the presented
sounds. As so, they tended to deem as highly fluctuating sounds that had a high
modulation frequency. Participants 4 and 5 explicitly stated that they were
counting the number of cycles in the stimuli, due to confusion on what to
answer.

Taking all these comments as feedback, version 3 of the experiment was
elaborated. In this version, explicit instructions regarding the rough tones
were given. It was indicated that the sensation of fluctuation was unrelated to
the apparent `speed' of the stimuli, and that the answers should be based on a
intuitive way, based on the arising sensation and not rationalizing it (for
instance by counting cycles). Using this approach participants were able to
understand better the fluctuation strength concept, some of them even coming
with analogies to the sensation itself (the sound of an ambulance alarm, the
sound of a washing machine).

The final version, number 4, of the experiment added a small test experiment
before starting the actual experiment sections. This was added as a suggestion
of participant 8, who indicated that although the training phase was effective
in making clear the fluctuation strength concept, it did not showed the
participant how to do the expected judgments using the magnitude estimation
procedure. Moreover, a latin square randomization approach was used, to
distribute possible learning effects of participants among the experimental
conditions.

% section description (end)

\section{Results} % (fold)
\label{sec:results}

The following are the results of all the participants grouped together by
experiment. In the appendix the individual results can be found. In all the
following figures, the error bars indicate the standard deviation of the
reported values.

\subsection{AM Tones} % (fold)
\label{subsec:results_am_tones}

\figStds{AM-fm}
  {All}
  {Relative fluctuation strength as a function of modulation frequency for AM
  tones}

\figStds{AM-fc}
  {All}
  {Relative fluctuation strength as a function of center frequency for AM
  tones}

\figStds{AM-SPL}
  {All}
  {Relative fluctuation strength as a function of sound pressure level for AM
  tones}

\figStds{AM-md}
  {All}
  {Relative fluctuation strength as a function of modulation depth for AM
  tones}

% subsection results_am_tones (end)

\subsection{FM Tones} % (fold)
\label{subsec:results_fm_tones}

\figStds{FM-fm}
  {All}
  {Relative fluctuation strength as a function of modulation frequency for FM
  tones}

\figStds{FM-fc}
  {All}
  {Relative fluctuation strength as a function of center frequency for FM
  tones}

\figStds{FM-SPL}
  {All}
  {Relative fluctuation strength as a function of sound pressure level for FM
  tones}

\figStds{FM-df}
  {All}
  {Relative fluctuation strength as a function of frequency deviation for FM
  tones}

% subsection results_fm_tones (end)

% section results (end)

\clearpage

\appendix

\section{Individual Results} % (fold)
\label{sec:individual_results}

\subsection{AM Tones} % (fold)
\label{subsec:individual_results_am_tones}

\indRes{Modulation Frequency}
  {AM-fm}
  {Relative fluctuation strength as a function of modulation frequency for AM
  tones}
  {1,2,3,4,5,6}

\indRes{Center Frequency}
  {AM-fc}
  {Relative fluctuation strength as a function of center frequency for AM
  tones}
  {1,2,3,4}

\indRes{Sound Pressure Level}
  {AM-SPL}
  {Relative fluctuation strength as a function of sound pressure level for AM
  tones}
  {1,2,3,4}

\indRes{Modulation Depth}
  {AM-md}
  {Relative fluctuation strength as a function of modulation depth for AM
  tones}
  {1,2,3,4}

% subsection individual_results_am_tones (end)

\clearpage

\subsection{FM Tones} % (fold)
\label{subsec:individual_results_fm_tones}

\indRes{Modulation Frequency}
  {FM-fm}
  {Relative fluctuation strength as a function of modulation frequency for FM
  tones}
  {1,7,8,9}

\indRes{Center Frequency}
  {FM-fc}
  {Relative fluctuation strength as a function of center frequency for FM
  tones}
  {1,7,8,9}

\indRes{Sound Pressure Level}
  {FM-SPL}
  {Relative fluctuation strength as a function of sound pressure level for FM
  tones}
  {1,8,9}

\indRes{Frequency Deviation}
  {FM-df}
  {Relative fluctuation strength as a function of modulation depth for FM
  tones}
  {1,8,9}

% subsection individual_results_fm_tones (end)

% section individual_results (end)

\end{document}
