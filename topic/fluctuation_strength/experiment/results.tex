\documentclass[a4paper]{article}
\usepackage{mystyle}

\newcommand{\figRes}[4]{
\begin{figure}[ht!]
  \centering
  \includegraphics[height=8cm]{img/#1-#2-#3.eps}
  \caption{#4}
\label{fig:#1-#2-#3}
\end{figure}
}

\newcommand{\figPair}[3]{
  \figRes{#1}
    {#2}
    {standards}
    {#3, standards}
  \figRes{#1}
    {#2}
    {comparison}
    {#3, comparison}
}

\newcommand{\indRes}[4]{
  \subsubsection{#1}
  \foreach \i in {#4} {
    \figPair{#2}
      {\i}
      {#3, participant \i}
    \clearpage
  }
}

\begin{document}

\customtitle{Partial Results}

\section{Introduction} % (fold)
\label{sec:introduction}

This document reports the partial results of the fluctuation strength
experiments.

% section introduction (end)

\section{Results} % (fold)
\label{sec:results}

The following are the results of all the participants grouped together by
experiment. In the appendix the individual results can be found. In all the
following figures, the error bars indicate the interquartile range and the
marker indicates the median value. The combined curve of the two standards is
obtained considering the median value of all the observations in total for a
given parameter value (8 in total). In Table~\ref{tab:standards} the standards
used in each experiments are detailed.

\begin{table}
  \centering
  \rowcolors{3}{}{gray!25}
  \begin{tabu}{l | l l l l l}
  \tabucline[2pt]{-}
  \rowcolor{white}
  \multirow{2}{*}{Experiment} & \multicolumn{5}{c}{Parameters} \\\cline{2-6}
  & fm [Hz] & fc [Hz] & SPL [dB] & md [dB] & df [Hz]
  \\\tabucline[2pt]{-}
  \cellcolor{white}
  \multirow{2}{*}{AM-fm}  & 4 & 1000 & 70 & 40 & --- \\\cline{2-6}
                          & 0.25 & 1000 & 70 & 40 & --- \\\tabucline[1pt]{-}
  \cellcolor{white}
  \multirow{2}{*}{AM-fc}  & 4 & 1000 & 70 & 40 & --- \\\cline{2-6}
                          & 4 & 250 & 70 & 40 & --- \\\tabucline[1pt]{-}
  \cellcolor{white}
  \multirow{2}{*}{AM-SPL} & 4 & 1000 & 70 & 40 & --- \\\cline{2-6}
                          & 4 & 1000 & 50 & 40 & --- \\\tabucline[1pt]{-}
  \cellcolor{white}
  \multirow{2}{*}{AM-md}  & 4 & 1000 & 70 & 40 & --- \\\cline{2-6}
                          & 4 & 1000 & 70 & 4 & --- \\\tabucline[1pt]{-}
  \cellcolor{white}
  \multirow{2}{*}{FM-fm}  & 4 & 1500 & 70 & --- & 700 \\\cline{2-6}
                          & 0.5 & 1500 & 70 & --- & 700 \\\tabucline[1pt]{-}
  \cellcolor{white}
  \multirow{2}{*}{FM-fc}  & 4 & 6000 & 70 & --- & 200 \\\cline{2-6}
                          & 4 & 500 & 70 & --- & 200 \\\tabucline[1pt]{-}
  \cellcolor{white}
  \multirow{2}{*}{FM-SPL} & 4 & 1500 & 60 & --- & 700 \\\cline{2-6}
                          & 4 & 1500 & 40 & --- & 700 \\\tabucline[1pt]{-}
  \cellcolor{white}
  \multirow{2}{*}{FM-df}  & 4 & 1500 & 70 & --- & 700 \\\cline{2-6}
                          & 4 & 1500 & 70 & --- & 32 \\\tabucline[1pt]{-}
  \tabucline[2pt]{-}
  \end{tabu}
  \caption{Description of standards used per experiment. The light gray color
  row represents the first standard.}
\label{tab:standards}
\end{table}

\subsection{AM Tones} % (fold)
\label{subsec:results_am_tones}

\figPair{AM-fm}
  {All}
  {Relative fluctuation strength as a function of modulation frequency for AM
  tones}

\figPair{AM-fc}
  {All}
  {Relative fluctuation strength as a function of center frequency for AM
  tones}

\figPair{AM-SPL}
  {All}
  {Relative fluctuation strength as a function of sound pressure level for AM
  tones}

\figPair{AM-md}
  {All}
  {Relative fluctuation strength as a function of modulation depth for AM
  tones}

% subsection results_am_tones (end)

\subsection{FM Tones} % (fold)
\label{subsec:results_fm_tones}

\figPair{FM-fm}
  {All}
  {Relative fluctuation strength as a function of modulation frequency for FM
  tones}

\figPair{FM-fc}
  {All}
  {Relative fluctuation strength as a function of center frequency for FM
  tones}

\figPair{FM-SPL}
  {All}
  {Relative fluctuation strength as a function of sound pressure level for FM
  tones}

\figPair{FM-df}
  {All}
  {Relative fluctuation strength as a function of frequency deviation for FM
  tones}

% subsection results_fm_tones (end)

% section results (end)

\clearpage

\appendix

\section{Individual Results} % (fold)
\label{sec:individual_results}

\subsection{AM Tones} % (fold)
\label{subsec:individual_results_am_tones}

\indRes{Modulation Frequency}
  {AM-fm}
  {Relative fluctuation strength as a function of modulation frequency for AM
  tones}
  {1,3,5,7,9,11,13,15,17,21,23}

\clearpage

\indRes{Center Frequency}
  {AM-fc}
  {Relative fluctuation strength as a function of center frequency for AM
  tones}
  {1,3,5,7,9,11,13,15,17,19,21,23}

\clearpage

\indRes{Sound Pressure Level}
  {AM-SPL}
  {Relative fluctuation strength as a function of sound pressure level for AM
  tones}
  {1,3,5,7,9,11,13,15,17,21,23}

\clearpage

\indRes{Modulation Depth}
  {AM-md}
  {Relative fluctuation strength as a function of modulation depth for AM
  tones}
  {1,3,5,7,9,11,13,15,17,19,21,23}

\clearpage

% subsection individual_results_am_tones (end)

\subsection{FM Tones} % (fold)
\label{subsec:individual_results_fm_tones}

\indRes{Modulation Frequency}
  {FM-fm}
  {Relative fluctuation strength as a function of modulation frequency for FM
  tones}
  {2,4,6,8,10,12,14,16,18,20,22,24}

\clearpage

\indRes{Center Frequency}
  {FM-fc}
  {Relative fluctuation strength as a function of center frequency for FM
  tones}
  {2,4,6,8,10,12,14,16,18,20,22,24}

\clearpage

\indRes{Sound Pressure Level}
  {FM-SPL}
  {Relative fluctuation strength as a function of sound pressure level for FM
  tones}
  {2,4,6,8,10,12,14,16,18,20,22,24}

\clearpage

\indRes{Frequency Deviation}
  {FM-df}
  {Relative fluctuation strength as a function of frequency deviation for FM
  tones}
  {2,4,6,8,10,12,14,16,18,20,22,24}

\clearpage

% subsection individual_results_fm_tones (end)

% section individual_results (end)

\end{document}
