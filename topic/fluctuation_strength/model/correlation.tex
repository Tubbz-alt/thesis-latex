\documentclass[a4paper]{article}
\usepackage{mystyle}

\begin{document}

\customtitle{Fluctuation Strength Cross Correlation Analysis}

\section{Introduction}

This document presents a small analysis conducted in order to understand the
effect of the cross correlation used in the fluctuation strength model.

\section{Description}

\subsection{Test Tones}

Two test tones were used in the analysis, an AM tone ($f_c = 1000$ Hz, $f_m = 4$
Hz, $m_d = 40$ dB, SPL = 70 dB) and a FM tone ($f_c = 1500$ Hz, $f_m = 4$ Hz,
$d_f = 700$ Hz, SPL = 70 dB).

\section{Results}

Figure \ref{fig:corr} shows all the correlation coefficients for all the
channel bands for the two test tones. For the AM tone, all the values are close
to 1, whereas for the FM tone there is a gap between the 13th and the 26th
channel were the values get significantly lower.

\begin{figure}[ht]
    \centering
    \includegraphics[height=8cm]{img/model_correlations_AM_FM}
    \caption{Correlation coefficients for the two test tones}
    \label{fig:corr}
\end{figure}

In order two understand the difference between the two test tones, two channel
bands that present a different results in both tones were chosen. The 24th and
the 26th channel bands present a value close to 1 for the AM tone, and the
lowest value present in the case of the FM tone.

Figure \ref{fig:am2426} and Figure \ref{fig:fm2425} show the time signals from
the signals corresponding to the channel bands used in the correlation
calculation. In the case of the AM tone, the two signals are very close to each
other in regards to their phase, whereas for the FM tone this is not the case.

\begin{figure}[ht]
    \centering
    \includegraphics[height=8cm]{img/model_excitation_AM_24_26}
    \caption{AM tone - Time signals for 24th and 26th channel bands}
    \label{fig:am2426}
\end{figure}

\begin{figure}[ht]
    \centering
    \includegraphics[height=8cm]{img/model_excitation_FM_24_26}
    \caption{FM tone - Time signals for 24th and 26th channel bands}
    \label{fig:fm2425}
\end{figure}

\end{document}
