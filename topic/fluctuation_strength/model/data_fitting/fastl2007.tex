\documentclass{article}
\usepackage{mystyle}
\usepackage{fastl2007-commands}

\begin{document}

\mytitle{Fastl's Data Fitting}

\section{Introduction}
\label{sec:introduction}

This document presents the results of the data fitting process to
Fastl's~\cite{Fastl2007Psychoacoustics} data. First, the weights used in the
process are described. Them a set of $R^2$ are generated for several
combinations of these weights to find the best fit given the chosen weights set.
After that, a linear polynomial fit is plotted with the best values of the
weights. Finally, the results of the model compared to the experimental data and
to Fastl's data are presented.

\section{Model Weights}
\label{sec:model_weights}

\subsection{Modulation Frequency}
\label{sub:modulation_frequency}

To model the dependency of fluctuation strength on modulation frequency, the
vector Hweight is adjusted using the parameter $w_h$. Two Hweight vectors were
obtained, $H0_{AM}$ and $H0_{FM}$, which are intended to fit AM and FM tones
respectively. By using $w_h$, a proportion between the two vectors is obtained.
When $w_h=0$ only $H0_{FM}$ is used. Values of $w_h$ from 0 to 100 in
increments of 10 were used and adjusted $R^2$ values were obtained from those,
as will be explained in \cref{sec:linear_polynomial_fit}.

\subsection{Center Frequency}
\label{sub:center_frequency}

To model the dependency of fluctuation strength on center frequency, the vector
gzi is adjusted using the parameter $w_g$. Similarly to the Hweight vector, two
vectors were obtained for AM and FM tones. The same range of values of $w_h$ was
also used for $w_g$.

\section{Linear Polynomial Fit}
\label{sec:linear_polynomial_fit}

\Cref{fig:fastl2007_r-squared_comparison_wh,fig:fastl2007_r-squared_comparison_wg}
report the values of $R^2$ for AM tones, FM tones, the sum of the values for
both types of tones, and finally the absolute value of the difference of the
value of both types of tones. All these values can be found also in tables in
the appendix section at the end of this document.

\myfigurepair%
  {fastl2007_r-squared_comparison_wh}
  {$R^2$ as a function of $w_h$ for $f_m$ response curves}
  {fastl2007_r-squared_comparison_wg}
  {$R^2$ as a function of $w_g$ for $f_c$ response curves}

From the figures the values that minimize the difference between conditions
and still maintained a high sum between them are $w_h = 40$ and $w_g = 90$. As
so, these values will be considered to be the ones that yield the best model
fit. Furthermore, linear polynomial fit plots using the chosen weight values for
all the experimental conditions are presented in
\cref{fig:fastl2007_linear_am,fig:fastl2007_linear_fm}.

\myfigurequad%
  {fastl2007_linear_am-fm}
  {Modulation frequency}
  {fastl2007_linear_am-fc}
  {Center frequency}
  {fastl2007_linear_am-spl}
  {Sound pressure level}
  {fastl2007_linear_am-md}
  {Modulation depth}
  {
    \caption{Linear polynomial fit for AM tones response curves}
    \label{fig:fastl2007_linear_am}
  }

\myfigurequad%
  {fastl2007_linear_fm-fm}
  {Modulation frequency}
  {fastl2007_linear_fm-fc}
  {Center frequency}
  {fastl2007_linear_fm-spl}
  {Sound pressure level}
  {fastl2007_linear_fm-df}
  {Frequency deviation}
  {
    \caption{Linear polynomial fit for FM tones response curves}
    \label{fig:fastl2007_linear_fm}
  }

\clearpage

\section{Results}
\label{sec:results}

The comparison between the results of the model, the experimental data and
Fastl's data are shown in \cref{fig:AM_all_comparison,fig:FM_all_comparison}.

\begin{comparison}

\myfigurequad%
  {AM-fm_all_comparison}
  {Modulation frequency}
  {AM-fc_all_comparison}
  {Center frequency}
  {AM-SPL_all_comparison}
  {Sound pressure level}
  {AM-md_all_comparison}
  {Modulation depth}
  {
    \caption{Relative fluctuation strength for AM tones}
    \label{fig:AM_all_comparison}
  }

\myfigurequad%
  {FM-fm_all_comparison}
  {Modulation frequency}
  {FM-fc_all_comparison}
  {Center frequency}
  {FM-SPL_all_comparison}
  {Sound pressure level}
  {FM-df_all_comparison}
  {Frequency deviation}
  {
    \caption{Relative fluctuation strength for FM tones}
    \label{fig:FM_all_comparison}
  }

\end{comparison}

\section{Discussion}
\label{sec:discussion}

\begin{itemize}
  \item For modulation frequency and center frequency, a trade-off must be made
  when adjusting the model parameters.
  \item It remains unclear how the curves regarding SPL, modulation depth and
  frequency deviation can be adjusted
\end{itemize}

\mybibliography{}

\clearpage

\appendix

\section{Adjusted R-squared and model weights}

\mytablepair{AM-fm}{FM-fm}
\mytablepair{AM-fc}{FM-fc}
\mytablepair{AM-SPL}{FM-SPL}
\mytablepair{AM-md}{FM-df}

\end{document}
