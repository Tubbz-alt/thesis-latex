\documentclass{article}
\usepackage{mystyle}
\usepackage{fastl2007-commands}

\begin{document}

\mytitle{Fastl's Data Fitting}

\section{Introduction}
\label{sec:introduction}

This document presents the results of the data fitting process to
Fastl's~\cite{Fastl2007Psychoacoustics} data. First, the weights used in the
process are described and analysed. After that, a linear polynomial fit is
obtained to determine the goodness of fit of the model. Finally, the results
of the model compared to the experimental data and to Fastl's data are
presented.

\section{Weights Adjustment}
\label{sec:weights_adjustment}

This section presents the results of different weights for the Hweight and gzi
vectors.

\subsection{Modulation Frequency}
\label{sub:modulation_frequency}

To model the dependency of fluctuation strength on modulation frequency, the
vector Hweight is adjusted using the parameter $w_h$. Two Hweight vectors were
obtained, $H0_{AM}$ and $H0_{FM}$, which are intended to fit AM and FM tones
separately. By using $w_h$, a proportion between the two vectors is obtained.
When $w_h=0$ only $H0_{FM}$ is used. The following are the results for several
values of $w_h$.

\subsection{Center Frequency}
\label{sub:center_frequency}

To model the dependency of fluctuation strength on center frequency, the vector
gzi is adjusted using the parameter $w_g$. Similarly to the Hweight vector, two
vectors were obtained for AM and FM tones. The following are the results for
several values of $w_g$.


\section{Linear Polynomial Fit}
\label{sec:linear_polynomial_fit}

\myfigurepair%
  {fastl2007_r-squared_comparison_wh}
  {$R^2$ as a function of $w_h$ for $f_m$ response curves}
  {fastl2007_r-squared_comparison_wg}
  {$R^2$ as a function of $w_g$ for $f_c$ response curves}

\myfigurequad%
  {fastl2007_linear_am-fm}
  {Linear polynomial fit for AM-fm response curve}
  {fastl2007_linear_am-fc}
  {Linear polynomial fit for AM-fc response curve}
  {fastl2007_linear_am-spl}
  {Linear polynomial fit for AM-SPL response curve}
  {fastl2007_linear_am-md}
  {Linear polynomial fit for AM-md response curve}

\myfigurequad%
  {fastl2007_linear_fm-fm}
  {Linear polynomial fit for FM-fm response curve}
  {fastl2007_linear_fm-fc}
  {Linear polynomial fit for FM-fc response curve}
  {fastl2007_linear_fm-spl}
  {Linear polynomial fit for FM-SPL response curve}
  {fastl2007_linear_fm-df}
  {Linear polynomial fit for FM-df response curve}

The following are the results of the analysis. In each figure the adjusted
R-Squared is reported for the given polynomial fit.

\section{Results}
\label{sec:results}

The following are the results of the model, experimental data and Fastl data.

\begin{comparison}

\subsection{AM Tones}
\label{subsec:results_am_tones}

\myfigurequad%
  {AM-fm_all_comparison}
  {Relative fluctuation strength as a function of modulation frequency}
  {AM-fc_all_comparison}
  {Relative fluctuation strength as a function of center frequency}
  {AM-SPL_all_comparison}
  {Relative fluctuation strength as a function of sound pressure level}
  {AM-md_all_comparison}
  {Relative fluctuation strength as a function of modulation depth}

\subsection{FM Tones}
\label{subsec:results_fm_tones}

\myfigurequad%
  {FM-fm_all_comparison}
  {Relative fluctuation strength as a function of modulation frequency}
  {FM-fc_all_comparison}
  {Relative fluctuation strength as a function of center frequency}
  {FM-SPL_all_comparison}
  {Relative fluctuation strength as a function of sound pressure level}
  {FM-df_all_comparison}
  {Relative fluctuation strength as a function of frequency deviation}

\end{comparison}

\section{Discussion}
\label{sec:discussion}

\begin{itemize}
  \item For modulation frequency and center frequency, a tradeoff must be made
  when adjusting the model parameters. The chosen weights adjust better to the
  AM tones, but by changing the weight the contrary can be achieved
  \item It remains unclear how the curves regarding SPL, modulation depth and
  frequency deviation can be adjusted
\end{itemize}

\mybibliography{}

\clearpage

\appendix

\section{Adjusted R-squared and model weights}

\mytablepair{AM-fm}{FM-fm}
\mytablepair{AM-fc}{FM-fc}
\mytablepair{AM-SPL}{FM-SPL}
\mytablepair{AM-md}{FM-df}

\end{document}
