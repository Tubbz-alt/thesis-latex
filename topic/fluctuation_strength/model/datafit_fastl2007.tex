\documentclass[a4paper]{article}
\usepackage{mystyle}

\newcommand{\pairHeight}{4cm}

\newcommand{\figPair}[5]{
  \begin{figure}[!ht]
    \centering
    \begin{minipage}{0.45\textwidth}
    \centering
      \includegraphics[width=0.95\textwidth,height=\pairHeight]
        {img/Model_#1_wh-#2_wg-#3_AM-#4}
      \caption{AM Tones}
    \end{minipage}
    \begin{minipage}{0.45\textwidth}
    \centering
      \includegraphics[width=0.95\textwidth,height=\pairHeight]
        {img/Model_#1_wh-#2_wg-#3_FM-#4}
      \caption{FM Tones}
    \end{minipage}
    \caption{#5 --- $w_h=#2$, $w_g=#3$}
  \end{figure}
}

\newcommand{\figCorr}[2]{
  \begin{figure}[ht!]
    \centering
    \includegraphics[height=6cm]{img/Linear_fastl2007_model_#1.eps}
    \caption{#2}
  \end{figure}
}

\begin{document}

\customtitle{Fluctuation Strength\\Data Fit Analysis}

\section{Introduction} % (fold)
\label{sec:introduction}

This document presents the results of the data fit process to
Fastl's~\cite{Fastl2007Psychoacoustics} data. First, the weights used in the
process are described and analysed. After that, a linear polynomial fit is
obtained to determine the goodness of fit of the model. Finally, the results
of the model compared to the experimental data and to Fastl's data are
presented.

% section introduction (end)

\section{Weights Adjustment} % (fold)
\label{sec:weights_adjustment}

This section presents the results of different weights for the Hweight and gzi
vectors.

\subsection{Modulation Frequency} % (fold)
\label{sub:modulation_frequency}

To model the dependency of fluctuation strength on modulation frequency, the
vector Hweight is adjusted using the parameter $w_h$. Two Hweight vectors were
obtained, $H0_{AM}$ and $H0_{FM}$, which are intended to fit AM and FM tones
separately. By using $w_h$, a proportion between the two vectors is obtained.
When $w_h=0$ only $H0_{FM}$ is used. The following are the results for several
values of $w_h$.

\figPair{fastl2007}{0}{50}{fm}{Modulation Frequency}
\figPair{fastl2007}{30}{50}{fm}{Modulation Frequency}
\figPair{fastl2007}{50}{50}{fm}{Modulation Frequency}
\figPair{fastl2007}{70}{50}{fm}{Modulation Frequency}
\figPair{fastl2007}{100}{50}{fm}{Modulation Frequency}

\clearpage

% subsection modulation_frequency (end)

\subsection{Center Frequency} % (fold)
\label{sub:center_frequency}

To model the dependency of fluctuation strength on center frequency, the vector
gzi is adjusted using the parameter $w_g$. Similarly to the Hweight vector, two
vectors were obtained for AM and FM tones. The following are the results for
several values of $w_g$.

\figPair{fastl2007}{50}{0}{fc}{Center Frequency}
\figPair{fastl2007}{50}{30}{fc}{Center Frequency}
\figPair{fastl2007}{50}{50}{fc}{Center Frequency}
\figPair{fastl2007}{50}{70}{fc}{Center Frequency}
\figPair{fastl2007}{50}{100}{fc}{Center Frequency}

\clearpage

% subsection center_frequency (end)

% section weights_adjustment (end)

\section{Linear Polynomial Fit} % (fold)
\label{sec:linear_polynomial_fit}

The following are the results of the analysis. In each figure the adjusted
R-Squared is reported for the given polynomial fit.

\figCorr{AM-fm}{AM --- Modulation Frequency}
\figCorr{AM-fc}{AM --- Center Frequency}
\figCorr{AM-SPL}{AM --- Sound Pressure Level}
\figCorr{AM-md}{AM --- Modulation Depth}
\figCorr{FM-fm}{FM --- Modulation Frequency}
\figCorr{FM-fc}{FM --- Center Frequency}
\figCorr{FM-SPL}{FM --- Sound Pressure Level}
\figCorr{FM-df}{FM --- Frequency Deviation}

% section linear_polynomial_fit (end)

\custombibliography{}

\end{document}
