\documentclass{article}
\usepackage{mystyle}
\usepackage{power_coefficients_analysis-commands}

\begin{document}

\mytitle{Roughness Model\\Power Coefficients Analysis}

\section{Introduction}

This document reports the results from modifying the coefficients present in the
power law used to establish a relation between modulation depth and roughness.

\section{Model}

The sensation of fluctuation strength is modeled after its dependency on the
modulation depth of the incoming signal, as stated in \cref{eq:eq1}
\cite[p.~115]{daniel1997psychoacoustical}.

\begin{equation}
  \label{eq:eq1}
  R \sim m ^ p
\end{equation}

Furthermore, \citeauthor{daniel1997psychoacoustical} model is based on the
concept of specific roughness. For each auditory filter, a roughness value is
calculated and then the sum of them is obtained to obtain the total roughness
value. \Cref{eq:eq2} illustrates how to obtain a specific roughness for a given
auditory band.

\begin{equation}
  \label{eq:eq2}
  r_i = (g(z_i) \cdot m_i \cdot k_{i-2} \cdot k_i)^2
\end{equation}

\Cref{eq:eq2} can be further generalized by specifying separate power
coefficients for the terms inside the parenthesis. \Cref{eq:eq3} corresponds to
a more generalized equation, where it can be seen that the original
\cref{eq:eq2} can be obtained by setting $p_g = p_m = p_k = 2$.

\begin{equation}
  \label{eq:eq3}
  r_i = g(z_i)^{p_g} \cdot m_{i}^{p_m} \cdot (k_{i-2} \cdot k_i)^{p_k}
\end{equation}

Using \cref{eq:eq3} the effect of the power coefficients on the total roughness
value will be investigated.

\section{Results}

To understand the effect of each power coefficient on the roughness, one of them
will be varied while the other two remain constant with a value of 2. The set
of values to use when varying the power coefficient will be displayed in the
results figures.

Varying $p_g$ results in a minimal effect on most curves,
\cref{fig:p_g_am,fig:p_g_fm}. Only \cref{fig:p_g_am-fc} presents a significant
variation, which makes sense since this coefficient affects $g_{zi}$, which in
turn models the effect of center frequency on roughness.

\myfigurecoefficients{p_g}{am}{md}
\myfigurecoefficients{p_g}{fm}{df}

Varying $p_m$ yields the significant changes, more pronounced for AM tones than
for FM tones. For FM tones the responses to modulation frequency and center
frequency are affected (\cref{fig:p_m_fm-fm,fig:p_m_fm-fc}). For AM tones all
responses are affected. Particularly important is the change on modulation depth
(\cref{fig:p_m_am-md}), which shows a way to adjust this response for the data
fitting process.

\myfigurecoefficients{p_m}{am}{md}
\myfigurecoefficients{p_m}{fm}{df}

Varying $p_k$ has little effect on AM tones. However, for FM tones the
responses are affected. Again, the most important result to highlight
corresponds to the frequency deviation curve (\cref{fig:p_k_fm-df}), which
shows also a possible way to adjust this response for the data fitting process.

\myfigurecoefficients{p_k}{am}{md}
\myfigurecoefficients{p_k}{fm}{df}

\section{Conclusions}

Adjusting the power coefficients $p_m$ and $p_k$ allows to fine tune the model
response curves for the modulation depth and frequency deviation cases. However,
the effect of modifying these values not only affect these responses but all of
them.

\mybibliography{}

\end{document}
