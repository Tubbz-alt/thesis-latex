Topic: Modeling the Sensation of Fluctuation Strength
Date & Time: 08-12-2015, 10:00
Location: IPO 0.26
Supervisors: A.G. Kohlrausch (TU/e), R.H. Cuijpers (TU/e), A.A. Osses Vecchi (TU/e)

Perceptual attributes are discernible dimensions into which auditory events can
be decomposed. They allow to understand the perceptual and cognitive effects
caused by sound signals. Among these perceptual attributes is the attribute of
fluctuation strength, which corresponds to a fluctuation or circulation
sensation that certain sounds (e.g., ambulance sirens) give.

Fluctuation strength has been related to several cognitive and perceptual
processes, such as speech production and comprehension, cognitive performance,
perceived sound quality, among others. However, the attribute itself has not
been studied as deeply as other perceptual attributes. Additionally, there is
not a publicly available model that reflects its characteristics. Furthermore,
recent studies have run into problems when trying to reproduce the findings
reported in the available literature. The present study addressed all these
problems related to the research of fluctuation strength, limiting its scope
to amplitude-modulated (AM) and frequency-modulated (FM) tones.

In order to deal with possible methodological issues, a new experimental
procedure was devised aimed at correcting possible biases. The procedure
included the use of a training phase to familiarize subjects with the sensation
of fluctuation strength. Moreover, subjective data that assessed the
dependencies of fluctuation strength on several key parameters was obtained,
using a magnitude estimation procedure. 24 subjects in total participated in
this study. In general, the training phase helped participants understand better
the concept of fluctuation strength. Furthermore, the obtained subjective data
and the data found in the literature were deemed as qualitatively similar.

Using the obtained subjective data, a model adjusted to it was developed. The
proposed model was based on a model for the perceptual attribute of
roughness. Similarities between the physical phenomena related to both
perceptual attributes made it possible to come up with the model itself.
The model was able to produce qualitatively similar results to those present
in the obtained subjective data. However, the model adapts better for AM tones
than to FM tones.

Overall, even though some differences exists between the obtained data and the
data from the literature, the proposed experimental procedure was considered
to be adequate in order to obtain subjective data regarding the attribute of
fluctuation strength. Along the same lines, although the proposed model
presented some important limitations to consider, it was possible to obtain
similar results from it when compared to the obtained data. As so, the model
was deemed to be adequate in modeling the obtained data.
