\documentclass[../main.tex]{subfiles}

\begin{document}

\chapter{Model Development}

\begin{modelchapter}

The following chapter details the process to develop a model for the fluctuation
strength sensation, based on \citeauthor{daniel1997psychoacoustical}'s roughness
model. The model will be analyzed in stages, where the changes needed to adapt
it to this particular case will be detailed. After that, a procedure to adjust
the parameters of the model is presented. Finally, the results of the
fluctuation strength model will be presented, and the limitations of the data
fitting will be addressed.

\section{Roughness Model}

\citeauthor{daniel1997psychoacoustical}'s roughness model will be used as a
basis for a fluctuation strength model adapted to the data obtained in the
experimental stage of this study. It has been shown already that roughness and
fluctuation strength sensations are similar from a physical point of view; both
attributes arise from modulated sounds. As so, the methodology used to model
the roughness sensation can be adequate to also model the fluctuation strength
sensation.

The structure of the roughness model is presented in \cref{fig:roughness_model}.
The model can be separated into three stages:
\begin{enumerate}
  \item Peripheral stage
  \item Modulation depth extraction stage
  \item Specific roughness stage
\end{enumerate}

\begin{figure}[!ht]
  \centering
  \includegraphics[height=10cm]{roughness_model}
  \caption{Structure of roughness model, as presented in
    \cite[pp.~116]{daniel1997psychoacoustical}}
  \label{fig:roughness_model}
\end{figure}

\subsection{Peripheral Stage}

First, the input signal is divided into frames, which in the original roughness
model have a 200 ms duration. For the fluctuations strength case this must be
increased, in order to achieve higher frequency detail resolution and be able
to process stimuli with frequency components with little separation among them
(e.g., $f_m = 4$ Hz). This can be achieves manipulating two variables, namely
the sampling frequency and the number of samples. For this study it has been
decided to keep the sampling frequency at 44.1 kHz, the most common frequency
used in audio recordings. The number of samples is chosen such that it contains
at least three periods of the slowest frequency component, following the
criteria presented by~\cite[pp.~97]{Boersma1993}. The slowest stimulus has a
modulation frequency of 0.25 Hz, thus three periods would constitute a 12 s
frame duration. This results in a number of samples of 1048576 ($2^{20}$) and a
duration of 23.77 s, using the closest value that is a power of 2.

After the frame separation, a Blackman window is applied to the frame to reduce
spectral leakage. Following this, the outer and middle ear transmission effects
are taken into account by transforming the frame into the frequency domain, and
multiplying its components by the parameter $a_0$, shown in \cref{fig:a0}.

\begin{figure}[!ht]
  \centering
  \includegraphics[height=8cm]{a0}
  \caption{Outer and middle ear transmission effects parameter $a_0$}
  \label{fig:a0}
\end{figure}

After this, the input frame spectrum is transformed into excitation patterns
using \citeauthor{Terhardt1979}'s approach \cite{Terhardt1979}. This method is a
non-linear one, where the lower and upper slopes of the excitation patterns
differ. The lower slopes are independent of the stimulus center frequency, and
are defined by:
\begin{equation}
  S_1 = -27 \quad \frac{\text{dB}}{\text{Bark}}
  \label{eq:lower_slopes}
\end{equation}

The upper slopes depends on the center frequency of the stimulus, and are
defined by the following equation:
\begin{equation}
  S_2 = [-24-\frac{0.23 \text{ kHz}}{f}+\frac{0.2 \text{ L}}{\text{dB}}]
  \quad
  \frac{\text{dB}}{\text{Bark}}
\end{equation}

Finally, the excitation patterns are processed using a critical-band filter with
47 channel, each on them separated by 0.5 Bark and having a bandwith of 1 Bark.

\subsection{Modulation Depth Extraction Stage}

This stage's objective is to come up with an approximation of the modulation
depth present in the incoming frame on a channel basis. For each channel signal
$e_{i}(t)$ its absolute value is obtained and its mean is then calculated with
\cref{eq:h0i}.

\begin{equation}
  h_{0,i} = \overline{|e_{i}(t)|}
  \label{eq:h0i}
\end{equation}

The mean is then substracted to each signal and the resulting signal is filtered
using a bandpass filter $H[f_{mod}]$, shown in \cref{eq:hBPi}.

\begin{equation}
  h_{BP,i}(t) = ([|e_{i}(t)| - h_{0,i}] * H[f_{mod}])(t)
  \label{eq:hBPi}
\end{equation}

Finally, the modulation depth per channel is obtained by the ratio of the
\gls{RMS} value of the bandpass filterd signal and its mean, shown in
\cref{eq:mi*}.

\begin{equation}
  {m_i}^* = \tilde{h}_{BP,i}(t)/h_{0,i}
  \label{eq:mi*}
\end{equation}

Two changes are introduced in this stage to adapt the model behavior to the
fluctuation strength scenario. First, an unique filter is used for all the 47
channels, whereas in the roughness model a different filter is used for each
channel, to take into account possible effect of the center frequency on the
modulation frequency dependency. Second, since the bandpass filter response
defines the modulation depth dependence, by adjusting it a better fit for the
fluctuation strength case can be achieved

On an implementation note with regard to the bandpass filter, the filtering is
done using a Butterworth \gls{IIR} filter. Its parameters are detailed in
\cref{tab:bandpass_filter}.

\begin{table}[!ht]
  \centering
  \begin{tabu}{l l}
    \toprule
    \rowfont\bfseries
    Parameter & Value \\
    \midrule
    Stopband Frequency 1 & 0.5 [Hz] \\
    \midrule
    Passband Frequency 1 & 2 [Hz] \\
    \midrule
    Passband Frequency 2 & 8 [Hz] \\
    \midrule
    Stopband Frequency 2 & 32 [Hz] \\
    \midrule
    Stopband Attenuation 1 & 100 [dB] \\
    \midrule
    Passband Ripple & 3 [dB] \\
    \midrule
    Stopband Attenuation 2 & 100 [dB] \\
    \bottomrule
  \end{tabu}
  \caption{Bandpass filter characteristics}
  \label{tab:bandpass_filter}
\end{table}

\subsection{Specific Roughness Stage}

\section{Procedure}




\section{Results}

\modelresultsfigure{am-fm}{modulation frequency}{AM}
\modelresultsfigure{am-fc}{center frequency}{AM}
\modelresultsfigure{am-spl}{sound pressure level}{AM}
\modelresultsfigure{am-md}{modulation depth}{AM}
\modelresultsfigure{fm-fm}{modulation frequency}{FM}
\modelresultsfigure{fm-fc}{center frequency}{FM}
\modelresultsfigure{fm-spl}{sound pressure level}{FM}
\modelresultsfigure{fm-df}{frequency deviation}{FM}


\end{modelchapter}

\end{document}
