\documentclass[../main.tex]{subfiles}

\begin{document}

\chapter{Introduction}
\label{cha:introduction}

The auditory sensation of fluctuation strength is the sensation that arises due
to fluctuating or circulating patterns present in certain stimuli. For instance,
ambulance sirens and the sounds of working washing machines have this quality,
as their sounds have a certain movement, rotation or circulation associated to
it. Fluctuation strength is part of a group of perceptual attributes in the body
of knowledge of the psychoacoustics discipline. These components allow to
understand from a perceptual point of view sound events, quantifying phenomena
that develop themselves within the human mind.

Fluctuation strength relates to several cognitive processes that mediate
important human phenomena. A possible relation with the speech system has been
suggested, and could prove to be an important indicator for it. Furthermore, it
is known that fluctuation strength plays an important role when it comes to the
perceived ``pleasantness'' of given sounds.

There are several knowledge gaps when it comes to fluctuation strength in the
available literature. First, methodological issues in past studies have resulted
in inability to reproduce reported findings. Second, to our knowledge, there is
no publicly available work when it comes to modeling the sensation of
fluctuation strength.

Taking the preceding points into account, the objective of this research is
twofold:
\begin{enumerate}
  \item Formulate a clear methodological procedure to eliminate the problems
    found in past studies
  \item Propose a fluctuation strength model based on existing roughness models,
    adjusted to collected experimental data
\end{enumerate}

Furthermore, this study will focus on two types of stimuli, \gls{AM} tones and
\gls{FM} tones, leaving \gls{AM} \gls{BBN} stimuli aside due to time
constraints.

\Cref{cha:theoretical} lays down the theoretical underpinnings that support the
knowledge regarding the sensation of fluctuation strength. An introduction to
the topic of perceptual attributes is given, followed by a thorough description
of the sensation of fluctuation strength. Then, literature concerning the
modeling of fluctuation strength is reviewed. Afterwards, related studies
investigating fluctuation strength are presented.

\Cref{cha:methods} presents the general methods used to obtain subjective data
from participants, used both in the pilot and the main experiments. The
equipment and the stimuli used are described, and then the magnitude estimation
process used to obtain participants data is detailed.

\Cref{cha:pilot,cha:experiment} deal with the process of designing and carrying
out the experiments of this study. First, the pilots experiments used to shape
the initial design are presented, with their results and conclusions given
afterwards. Following this, the final form of experiment is described, and the
results are reported along with the data already existing in the literature.

\Cref{cha:model} documents the process of developing a model for the obtained
experimental data, based on an existing model intended for the sensation of
roughness. The procedure used to modify and adjust the roughness model is
detailed. Afterwards, a comparison between the model output and the
experimental data is presented.

\Cref{cha:discussion} presents a discussion regarding the results of both the
experimental and modeling stages of this research, finalizing in a series of
conclusions and limitation to be taken into account for future works.

\end{document}
