\documentclass[../main.tex]{subfiles}

\begin{document}

\chapter{Discussion}

\iffalse

\section{Summmary}

\begin{itemize}
  \item Experiment
  \begin{itemize}
    \item sequential stimuli in training phase useless
    \item possible confusion of participants, fixation on modulation and not
      on sensation
    \item long duration -> participants tired; reduce number of repetitions?
      this could affect the outcome? (medians with big IQR)

    \item modulation frequency curves have the expected bandpass response,
      although the bandwidth of these responses is wider
    \begin{itemize}
      \item addition of tones in modulation frequency conditions can bias
        results? this could also explain the shift in the peak value of the
        curve
      \item AM curve presents more uncertainty that FM curve, is this an
        indication that fm fluctuation is easier to understand than AM?
    \end{itemize}
    \item AM-fc presents a flat response, different from Fastl's but deemed
      satisfactory due to the high uncertainty in the original results
    \item SPL curves present some variations but overall behave as expected
   \end{itemize}
    \item There are sistematical variations in the AM-md, FM-fc and FM-df curves
    \item Taking md and df as analogous quantities, participants seem to exhibit
      a high sensitivity towards these parameters
    \item FM-fc deviates significantly from Fastl's data, the later presenting
      a monotonic descent, while the obtain data remains flat. Less sensitivity
      to auditory filtering, frequency spread?
    \item for FM tones, center frequency experiment not very representative
  \item Model
  \begin{itemize}
    \item Absolute value of cross correlation coefficients may pose problems
      when model is extended to BBN
    \item Model designed for AM, FM fit is worse, bandpass filtering less
      efective in this case, modulation filterbank as a solution?
    \item Big frame size makes model slow, must be improved
    \item ERB and Gammatone filters could be used for a more up to date model
  \end{itemize}
\end{itemize}

\fi

This chapter develops on the insights gathered from the experimental and
modeling phases of the study. First the remarks concerning the experiments will
be addressed, and later the details regarding the model will be explored.

\section{Experiment}

\subsection{Methods}

The training phase, product of the pilot experiments carried out before the
actual experimental, proved to be beneficial. Participants were able to
understand better the concept of fluctuation strength as a result of the
presentation of key examples, the stimuli comparison section of the training
phase. Nonetheless, the confusion between roughness and fluctuation strength was
not completely eliminated. In the modulation frequency sections of the
experiment, some participants exhibited increasing values of fluctuation
strength as the modulation frequency increased too. This reveals a lack of
understanding of the sensation. However, the concept of fluctuation strength is
difficult to grasp, and it depends largely on individuals' past experiences and
capability to focus on the stimuli correctly. As so, some confusion will always
exists, and this phenomena can only be reduced so much.

Furthermore, the instructions that the participants received, and how they
understand them, can influence greatly the expected results. In this case,
participants were told to focus on the actual sensation of fluctuation and they
should not try to associate it to any physical parameters of the sounds, namely
modulation frequency which leads to the roughness confusion. Some participants
interpreted this as meaning that they should negate any other physical parameter
effects on the fluctuation as long as it does not affects it directly. For
example, sound pressure level would sometimes be considered to have a flat
response since it does not modifies the modulation itself, only the sound's
loudness. Explicitly addressing these facts could leads to more accurate
fluctuation estimation on behalf of the participants.

Expanding on the training phase, the stimuli comparison and the test section
experiment parts of it proved to be helpful for participants. The sequential
stimuli section was not that helpful as the other sections, as participants
always stated that they could difference the stimuli. This latter section
could be removed from future experiment, in order to make them shorter and
easier on the participants.

With regard to the experimental sections themselves, some participants noted
the fact that the slider would not reset to its original position after each
trial was done. This was a technical limitation, due to software errors in the
chosen platform. Although one may wonder whether the fact that slider's initial
position could somehow affect participants' responses, changing the starting
100\% reference point from a visual point of view, nothing but speculation
can be formulated at this point.

The experiment duration rounds about an hour, and participants were required to
be sited and watching at computer screen while listening continuously for the
duration. Although rests between the experimental sections were proposed to
participants, almost none of them took them, preferring to finish the experiment
as fast as possible. Moreover, some participants reported feeling tired, dizzy
or with a slight headache after the experiment's conclusion. These tiredness
effects were intended to be balanced with the use of the latin square design,
so they would distribute among experimental conditions. Another possible
solution would be to reduce the number of repetitions per pair presentation.
The chosen value, 4, was obtained from \citeauthor{Fastl1982Fluctuation}'s
experiment~\cite{Fastl1982Fluctuation}, as it provided a $\pm10$ value deviation
between answers. Related to experimental duration, the sections corresponding to
the modulation frequency dependency were split in two, to mantain all sections
duration around 6 minutes. Whether this introduces any changes in participants
responses is unknown.

\end{document}
