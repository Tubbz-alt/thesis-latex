\documentclass[../main.tex]{subfiles}

\begin{document}

\chapter{Discussion}
\label{cha:discussion}

This chapter discusses the main points of the experimental and modeling stages
of the sensation of fluctuation strength, presented in
\Cref{cha:experiment,cha:model}. First the remarks concerning the experiments
will be addressed, and later the details regarding the model will be explored.

\section{Experiment}

\subsection{Methods}

The training phase, which was developed based on of the pilot experiments carried
out before the actual experimental, proved to be beneficial. Participants were
able to understand better the concept of fluctuation strength as a result of the
presentation of key examples, the stimulus comparison section of the training
phase. This was done without forcing participants to specific responses, thus
allowing them to learn by themselves the differences between sensation.
Nonetheless, the confusion between roughness and fluctuation strength was not
completely eliminated. In the modulation frequency sections of the experiment,
some participants exhibited increasing values of fluctuation strength as the
modulation frequency increased too. This reveals a lack of understanding of the
sensation. However, the concept of fluctuation strength is difficult to grasp,
and it depends largely on individuals' past experiences and capability to focus
on the stimuli correctly. As such, some confusion will always exists, and this
phenomenon can only be reduced to a certain extent.

Furthermore, the instructions that the participants received, and how they
understand them, can influence greatly the expected results. In this case,
participants were told to focus on the actual sensation of fluctuation and they
should not try to associate it to any physical parameters of the sounds, namely
modulation frequency which leads to the roughness confusion. Some participants
interpreted this as meaning that they should negate any other physical parameter
effects on the fluctuation as long as it does not affects it directly. For
example, sound pressure level would sometimes be considered to have a flat
response since it does not modify the modulation itself, only the loudness of
the sounds. Explicitly addressing these facts could lead to more accurate
fluctuation estimation on behalf of the participants.

During the training phase, the stimulus comparison and the test section
experiment parts of it proved to be helpful for participants. The long interval
section was not as helpful as the other sections, as participants always stated
that they could distinguish the stimuli. This latter section could be removed
from future experiments, in order to make them shorter and easier for the
participants.

With regard to the experimental sections themselves, some participants noted
the fact that the slider would not reset to its original position after each
trial was done. This was a technical limitation of the chosen platform. Although
one may wonder whether the fact that slider's initial position could somehow
affect participants' responses, changing the starting 100\% reference point from
a visual point of view, nothing but speculation can be formulated at this point.

The experimental session had an approximate duration of one hour. During the
experiment participant were sitting and staring at the computer screen while
they listened to the stimuli. Although breaks between the experimental sections
were suggested to participants, almost none of them took them, preferring to
finish the experiment as fast as possible. Moreover, some participants reported
feeling tired, dizzy or with a slight headache after the conclusion of the
experiment. These tiredness effects were intended to be balanced with the use of
the latin square design, so they would distribute among experimental conditions.
Another possible solution would be to reduce the number of repetitions per pair
presentation. The chosen value of 4 repetitions was obtained from
\citeauthor{Fastl1982Fluctuation}'s experiment~\cite{Fastl1982Fluctuation}, as
it provided a $\pm10$\% value deviation between answers. Related to experimental
duration, the sections corresponding to the modulation frequency dependency were
split into two sections, to keep all sections duration around 6~minutes. Whether
this introduces any changes in participants responses, i.e., having two shorter
sections instead of a longer section, is unknown.

\subsection{Results}

The fluctuation strength as a function of modulation frequency curves, both for
\gls{AM} and \gls{FM} tones showed the expected bandpass responses, although
the bandwidth of these response is wider than that reported by
\citeauthor{Fastl2007Psychoacoustics}. One possible reason for this is the
influence of the additional tones ($f_m = \{0, 64, 128\}$ Hz), added to
counteract the confusion of roughness and fluctuation strength. This changes
the lowest and highest values present for modulation frequency. These points
act as anchors, providing participants with values that implicitly have an
almost zero value of fluctuation strength. As so, the response is ``stretched'',
similar to the enlargement of the bandwidth observed in the characteristic
band-pass responses. Furthermore, during the course of the experiment
participants were exposed slowly to the whole range of fluctuation values. Since
they usually do not have the concept completely clear from the start, during the
experiment they progressively adjust the internal references values assigned to
the two standards provided. This could also explain why in general responses are
higher compared to the literature data. Internal-state models for magnitude
estimation procedures have been proposed in the past~\cite{Marley1972},
supporting this view on the cognitives process that govern participants answers.
Furthermore, past studies~\cite{Teghtsoonian1978} have shown that the range of
stimuli affects the outcome of a magnitude estimation process.

Regarding the fluctuation strength as a function of center frequency, for
\gls{AM} a mostly flat response was found. Although it is somewhat different
from that presented by \citeauthor{Fastl2007Psychoacoustics}, this data also
present large \gls{IQR} values. Therefore the obtained curve is deemed to be
qualitatively similar. With respect to \gls{FM} tones the obtained data
deviates significantly from \citeauthor{Fastl2007Psychoacoustics}, the former
having a flat trend, while the latter decreases monotonically with the increase
of the center frequency. This difference will be addressed later, when the
frequency deviation curve will be discussed.

Some participants stated that they were not sure if the variation of the
sound pressure level should be considered to influence the sensation of
fluctuation. However, the sound pressure level curves showed some minor
differences compared to the literature, but overall are consistent with the
expected behavior.

The last two parameters, modulation depth and frequency deviation, present also
systematical variations with respect to those reported by
\citeauthor{Fastl2007Psychoacoustics}. Since these two parameters can be
considered analogous indicators for the amount of modulation for each type of
tone, they will be analyzed at the same time. Participants exhibit a lack of
sensitivity to changes in modulation, resulting in higher than expected values
for both curves. A small value of \gls{m_d} or \gls{d_f} results in a higher
value of fluctuation, while the increase of modulation resulting in a less
steep increase in fluctuation.

For \gls{FM} tones, this lack of sensitivity can also explain the differences
found in the center frequency and frequency deviation curves. In both cases,
the number of auditory filters excited by the incoming signal depends on
the parameters. For center frequency, an increase of frequency corresponds to a
decrease in the number of filters excited, since for higher frequency the
auditory filters have wider bandwidths. For the frequency deviation the opposite
occurs, because an increase in the frequency deviation increases the
stimulus bandwidth to, resulting in more auditory filters covered by it. It
seems that, as the number of auditory filters that are excited increases, their
contribution to the overall fluctuation strength of the sound decreases.

Finally, the fluctuation strength as a function of center frequency curve used
for \gls{FM} tones is not very suitable for analyzing only the effect of center
frequency itself. As the center frequency increases, the number of auditory
filters excited decreases, due to an enlargment of their bandwidth for higher
center frequency values. As so, two effects are present in this response, center
frequency and number of auditory filters. As an improvement to the methodology,
stimuli with a lower value of frequency deviation could be used, to eliminate
the auditory filter effect.

\section{Model}

\subsection{Results}

Overall the model provides qualitatively similar estimates for the fluctuation
strength values presented by the experimental data. However, it adapts better to
the \gls{AM} tones than to the \gls{FM} tones. This make sense, since the model
was intended to be used with \gls{AM} stimuli, hence the fundamental use of
modulation depth (a property found in \gls{AM} only) to estimate the amount of
fluctuation. Therefore, a trade-off must be established when adjusting the model
parameters having each type of tone in mind. Nevertheless, the model predicts
values close to those found in the data for \gls{FM} tones. One possible
improvement in this regard is to substitute the generalized modulation depth
formula proposed by \citeauthor{daniel1997psychoacoustical} with a modulation
filterbank~\cite{Dau1997}, that could model better the energy distribution of
the signal among specific frequency bands.

\subsection{Limitations}

The use of the absolute value of cross correlation coefficients may pose
problems when using the model with \gls{BBN} \gls{AM} tones, the other type
of stimulus used by \citeauthor{Fastl2007Psychoacoustics} in their study.
Although some \gls{FM} tones used in this study present already negative cross
correlation coefficients, they do not affect negatively the overall fit obtain
with the model.

The increased frame size needed to achieve the frequency resolution for the
model renders it unfeasible to implement in practical application. Currently
a frame size of $2^{20}$ samples is needed, which corresponds to around
24~seconds for a sampling frequency of 44.1~kHz.

Finally, more modern modeling techniques, such as the use of the ERB perceptual
scale and a Gammatone filterbank instead of \citeauthor{Terhardt1979}'s can be
implemented, to improve the model and bring it to a more up to date state.

\section{Conclusions}

\begin{itemize}
  \item The training phase helps participants understand better the concept of
    fluctuation strenght
  \item \citeauthor{Fastl2007Psychoacoustics} data and the data obtained for
    this are qualitatively similar
  \item It is possible to adjust \citeauthor{daniel1997psychoacoustical} model
    to the fluctuation strength sensation
\end{itemize}

\end{document}
