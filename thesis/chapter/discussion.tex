\documentclass[../main.tex]{subfiles}

\begin{document}

\chapter{Discussion}

\section{Summmary}

\begin{itemize}
  \item Experiment
  \begin{itemize}
    \item sequential stimuli in training phase useless
    \item for FM tones, center frequency experiment not very representative
    \item possible confusion of participants, fixation on modulation and not
      on sensation
    \item modulation frequency curves have the expected bandpass response,
      although the bandwidth of these responses is wider
    \item long duration -> participants tired; reduce number of repetitions?
      this could affect the outcome? (medians with big IQR)
    \begin{itemize}
      \item addition of tones in modulation frequency conditions can bias
        results? this could also explain the shift in the peak value of the
        curve
      \item AM curve presents more uncertainty that FM curve, is this an
        indication that fm fluctuation is easier to understand than AM?
    \end{itemize}
    \item AM-fc presents a flat response, different from Fastl's but deemed
      satisfactory due to the high uncertainty in the original results
    \item SPL curves present some variations but overall behave as expected
   \end{itemize}
    \item There are sistematical variations in the AM-md, FM-fc and FM-df curves
    \item Taking md and df as analogous quantities, participants seem to exhibit
      a high sensitivity towards these parameters
    \item FM-fc deviates significantly from Fastl's data, the later presenting
      a monotonic descent, while the obtain data remains flat. Less sensitivity
      to auditory filtering, frequency spread?
  \item Model
  \begin{itemize}
    \item Absolute value of cross correlation coefficients may pose problems
      when model is extended to BBN
    \item Model designed for AM, FM fit is worse, bandpass filtering less
      efective in this case, modulation filterbank as a solution?
    \item Big frame size makes model slow, must be improved
    \item ERB and Gammatone filters could be used for a more up to date model
  \end{itemize}

\end{itemize}



\end{document}
