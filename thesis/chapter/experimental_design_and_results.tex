\documentclass[../main.tex]{subfiles}

\begin{document}

\chapter{Experimental Design and Results}
\label{cha:experiment}

In this chapter the final experimental design and results are presented.

\section{Design}

\subsection{Equipment}

The same equipment used in the pilot experiments was used in the final
experiments (\Cref{subsec:pilot_equipment}).

\subsection{Subjects}

Twenty-four participants were recruited from the JF Schouten database of the
Eindhoven University of Technology. Participants were between 19 and 31 years
old. There were in total six females and eighteen males. All of them reported
to have normal hearing, however this was not confirmed in any way. Subjects were
paid for  their participation.

\subsection{Methods}

Participants were assigned either to \gls{AM} tones or to \gls{FM} tones, both
conditions having 12 participants. Furthermore, as in the pilot experiment, the
experiment consisted of a training and an experimental phase. The whole
experiment had an approximate duration of 60~minutes. The experiment protocol
followed during the experiment can be found in the appendix of this document
(\Cref{chap:experiment_protocol}).

\subsubsection{Training Phase}

Adding all the improvements postulated in the pilot experiment section, the
training phase was expanded and became a 3-part phase, described below.

\paragraph{Stimuli Comparison}

As in the pilot experiments, the initial part of the training phase consisted
of comparison between stimuli. An additional stimulus was added ($f_m = 128$ Hz)
and a subset of \gls{FM} stimuli was also added to complements the \gls{AM}
tones. The stimuli presented in
\Cref{tab:am_training_stimuli,tab:fm_training_stimuli} were reproduced
sequentially according to their assigned id number.

\begin{table}[!ht]
  \centering
  \begin{tabu} to \linewidth{XXXXX}
    \toprule
    \rowfont\bfseries
    Id & $\bm{f_m}$ [Hz] & $\bm{f_c}$ [kHz] & SPL [dB] & $\bm{m_d}$ [dB] \\
    \midrule
    1 & 0   & 1 & 70 & 40 \\
    2 & 0.5 & 1 & 70 & 40 \\
    3 & 4   & 1 & 70 & 40 \\
    4 & 32  & 1 & 70 & 40 \\
    5 & 128 & 1 & 70 & 40 \\
    \bottomrule
  \end{tabu}
  \caption{Subset of AM stimuli for training phase}
\label{tab:am_training_stimuli}
\end{table}

\begin{table}[!ht]
  \centering
  \begin{tabu} to \linewidth{XXXXX}
    \toprule
    \rowfont\bfseries
    Id & $\bm{f_m}$ [Hz] & $\bm{f_c}$ [kHz] & SPL [dB] & $\bm{d_f}$ [Hz] \\
    \midrule
    1 & 0   & 1 & 70 & 700 \\
    2 & 0.5 & 1 & 70 & 700 \\
    3 & 4   & 1 & 70 & 700 \\
    4 & 32  & 1 & 70 & 700 \\
    4 & 128 & 1 & 70 & 700 \\
    \bottomrule
  \end{tabu}
  \caption{Subset of FM stimuli for training phase}
\label{tab:fm_training_stimuli}
\end{table}

\paragraph{Long Interval}

Two long intervals were used in this section of the training phase, composed
of \gls{AM} and \gls{FM} tones, respectively. These two intervals are described
in \Cref{tab:am_all_stimulus,tab:fm_all_stimulus}.

\begin{table}[!ht]
  \centering
  \begin{tabu} to \linewidth{XXXXX}
    \toprule
    \rowfont\bfseries
    Order & $\bm{f_m}$ [Hz] & $\bm{f_c}$ [kHz] & SPL [dB] & $\bm{m_d}$ [dB] \\
    \midrule
    1  & 0.5  & 1 & 70 & 40 \\
    2  & 32   & 1 & 70 & 40 \\
    3  & 2    & 1 & 70 & 40 \\
    4  & 16   & 1 & 70 & 40 \\
    5  & 4    & 1 & 70 & 40 \\
    6  & 1    & 1 & 70 & 40 \\
    7  & 0    & 1 & 70 & 40 \\
    8  & 64   & 1 & 70 & 40 \\
    9  & 0.25 & 1 & 70 & 40 \\
    10 & 128  & 1 & 70 & 40 \\
    11 & 8    & 1 & 70 & 40 \\
    \bottomrule
  \end{tabu}
  \caption{Long interval composed of \gls{AM} stimuli for training
  phase}
\label{tab:am_all_stimulus}
\end{table}

\begin{table}[!ht]
  \centering
  \begin{tabu} to \linewidth{XXXXX}
    \toprule
    \rowfont\bfseries
    Order & $\bm{f_m}$ [Hz] & $\bm{f_c}$ [kHz] & SPL [dB] & $\bm{d_f}$ [Hz] \\
    \midrule
    1  & 0.25 & 1 & 70 & 700 \\
    2  & 64   & 1 & 70 & 700 \\
    3  & 32   & 1 & 70 & 700 \\
    4  & 2    & 1 & 70 & 700 \\
    5  & 1    & 1 & 70 & 700 \\
    6  & 0    & 1 & 70 & 700 \\
    7  & 8    & 1 & 70 & 700 \\
    8  & 4    & 1 & 70 & 700 \\
    9  & 0.5  & 1 & 70 & 700 \\
    10 & 128  & 1 & 70 & 700 \\
    11 & 16   & 1 & 70 & 700 \\
    \bottomrule
  \end{tabu}
  \caption{Long interval composed of \gls{FM} stimuli for training
  phase}
\label{tab:fm_all_stimulus}
\end{table}

\paragraph{Test Section}

In order to familiarize participant with the interface used during the
experiment and with the magnitude estimation procedure, a small test section
was added at the end of the training phase. This section consisted of four
pairs (\Cref{tab:pairs_test_section}), which were presented to the participants
in randomized order.

\begin{table}[!ht]
  \centering
  \begin{tabu} to \linewidth{XXXXXX}
    \toprule
    \rowfont\bfseries
    \multirow{2}{*}{Pair} &
    \multicolumn{5}{c}{Parameters} \\
    \cmidrule{2-6}
    \rowfont\bfseries
    & $\bm{f_m}$ [Hz] & $\bm{f_c}$ [kHz] & SPL [dB] & $\bm{m_d}$ [dB] & $\bm{d_f}$ [Hz] \\
    \midrule
    \multirow{2}{*}{1} & 4  & 1 & 70 & 40 & --- \\
                       & 32 & 1 & 70 & 40 & --- \\
    \midrule
    \multirow{2}{*}{2} & 4  & 6 & 70 & --- & 200 \\
                       & 4  & 6 & 70 & --- & 200 \\
    \midrule
    \multirow{2}{*}{3} & 4  & 1 & 70 & 40 & --- \\
                       & 0  & 1 & 70 & 40 & --- \\
    \midrule
    \multirow{2}{*}{4} & 4  & 1.5 & 60 & --- & 700 \\
                       & 4  & 1.5 & 80 & --- & 700 \\
    \bottomrule
  \end{tabu}
  \caption{Pairs used in training phase test section}
  \label{tab:pairs_test_section}
\end{table}

\subsubsection{Experimental Phase}

In comparison to the pilot experiment, the experimental phase remained almost
without changes. The only effective change was the splitting of the modulation
frequency sections (AM-fm and FM-fm) into two separate sections. This was done
due to the longer duration of these sections when compared to the others ones.
By keeping all the sections relatively short (around 6 minutes each) it was
expected that participants attention and focus would be retained from one
section to another.

\subsection{Stimuli}

The stimuli used in the final experiment present the same characteristics as the
ones used in the pilot experiments (\Cref{subsec:pilot_stimuli}).
\Cref{tab:stimuli} details the stimuli used in the final experiments.

\begin{table}[!ht]
  \centering
  \begin{tabu} to \linewidth{p{2cm}Xp{9cm}}
  \toprule
  \rowfont\bfseries
  \multirow{2}{*}{Section} &
  \multicolumn{2}{c}{Parameters} \\
  \cmidrule{2-3}
  \rowfont\bfseries
  & \multicolumn{1}{c}{Fixed} & \multicolumn{1}{c}{Varied} \\
  \midrule
  AM-fm  & $f_c$ = 1 [kHz]\par SPL = 70 [dB]\par $m_d$ = 40 [dB]
         & $f_m$ = \{0, 0.25, 0.5, 1, 2, 4, 8, 16, 32, 64, 128\} [Hz] \\
  \midrule
  AM-fc  & $f_m$ = 4 [Hz]\par SPL = 70 [dB]\par $m_d$ = 40 [dB]
         & $f_c$ = \{0.125, 0.25, 0.5, 1, 2, 4, 8\} [kHz] \\
  \midrule
  AM-SPL & $f_m$ = 4 [Hz]\par $f_c$ = 1 [kHz]\par $m_d$ = 40 [dB]
         & SPL = \{50, 60, 70, 80, 90\} [dB] \\
  \midrule
  AM-md  & $f_m$ = 4 [Hz]\par $f_c$ = 1 [kHz]\par SPL = 70 [dB]
         & $m_d$ = \{1, 2, 4, 10, 20, 40\} [dB] \\
  \midrule
  FM-fm  & $f_c$ = 1.5 [kHz]\par SPL = 70 [dB]\par $d_f$ = 700 [Hz]
         & $f_m$ = \{0, 0.25, 0.5, 1, 2, 4, 8, 16, 32, 64, 128\} [Hz] \\
  \midrule
  FM-fc  & $f_m$ = 4 [Hz]\par SPL = 70 [dB]\par $d_f$ = 200 [Hz]
         & $f_c$ = \{0.5, 1, 1.5, 2, 3, 4, 6, 8\} [kHz] \\
  \midrule
  FM-SPL & $f_m$ = 4 [Hz]\par $f_c$ = 1.5 [kHz]\par $d_f$ = 700 [Hz]
         & SPL = \{40, 50, 60, 70, 80\} [Hz] \\
  \midrule
  FM-df  & $f_m$ = 4 [Hz]\par $f_c$ = 1.5 [kHz]\par SPL = 70 [dB]
         & $d_f$ = \{16, 32, 100, 300, 700\} [Hz] \\
  \bottomrule
  \end{tabu}
  \caption{Description of stimuli used per experiment section}
\label{tab:stimuli}
\end{table}

\section{Results}

The following section presents the results of the experiments, compared to
\citeauthor{Fastl2007Psychoacoustics} data~\cite{Fastl2007Psychoacoustics}.
Additionally, participants details and remarks obtained from them after the
execution of the experiment can be found in \Cref{chap:participants_sheet}.

\begin{experimentalresults}

\myfigurefastlexpstds%
  {AM-fm}
  {modulation frequency --- AM tones}

\myfigurefastlexpstds%
  {AM-fc}
  {center frequency --- AM tones}

\myfigurefastlexpstds%
  {AM-SPL}
  {sound pressure level --- AM tones}

\myfigurefastlexpstds%
  {AM-md}
  {modulation depth --- AM tones}

\myfigurefastlexpstds%
  {FM-fm}
  {modulation frequency --- FM tones}

\myfigurefastlexpstds%
  {FM-fc}
  {center frequency --- FM tones}

\myfigurefastlexpstds%
  {FM-SPL}
  {sound pressure level --- FM tones}

\myfigurefastlexpstds%
  {FM-df}
  {frequency deviation --- FM tones}

\myfigurequad%
  {AM-fm_all_comparison}
  {AM-fc_all_comparison}
  {AM-SPL_all_comparison}
  {AM-md_all_comparison}
  {
    \caption{Comparison between Fastl and experimental data --- AM tones}
    \label{fig:am_comparison}
  }

\myfigurequad%
  {FM-fm_all_comparison}
  {FM-fc_all_comparison}
  {FM-SPL_all_comparison}
  {FM-df_all_comparison}
  {
    \caption{Comparison between Fastl and experimental data --- FM tones}
    \label{fig:fm_comparison}
  }

\end{experimentalresults}

\end{document}
