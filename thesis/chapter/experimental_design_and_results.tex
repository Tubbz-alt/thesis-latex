\documentclass[../main.tex]{subfiles}

\begin{document}

\chapter{Experimental Design and Results}

\section{Design}

\subsection{Equipment}

A personal computer with an audio interface M-Audio Transit and a set of
headphones Sennheiser HD 265 Linear were used to conduct the experiment. The
computer was positioned in a sound-isolated booth. The experiment itself was
programmed using the APEX software platform and Windows batch files.

\subsection{Stimuli}

All the sounds presented during the experiment consisted of diotic stimuli. Two
types of stimuli were considered, \gls{am} tones (\cref{eq:am}) and \gls{fm}
tones (\cref{eq:fm}). In these equations, a $-\frac{\pi}{2}$ phase shift is
introduced to the modulating signal for both tones in order to start the
corresponding modulation at its lowest point and therefore avoid pops and clicks
due to an abrupt onset when presenting the sounds to the participants.
Additionaly, a cosine ramp with attack and release times of 25 ms was applied to
the stimuli to further prevent this phenomenon.

\begin{equation}
  x_{AM} = A_c \cdot [1 - m_d \cdot \cos(2 \pi f_m t)] \cdot \sin(2 \pi f_c t),
 \text{where } m_d = \frac{A_m}{A_c}
  \label{eq:am}
\end{equation}

\begin{equation}
  x_{FM} = A_c \cdot \sin \{2 \pi [f_c - d_f \cdot \cos(2 \pi f_m t)] t \}
  \label{eq:fm}
\end{equation}

The duration of the stimuli was specified such that it presented at least three
periods of the modulating signal, limited to range of values between 2 and 4
seconds. Therefore, stimuli with long periods (e.g., 0.25 Hz) were truncated to
4 seconds and stimuli with short periods (e.g., 128 Hz) were repeated to reach
a 2 seconds duration. This was done to mantain an uniform duration among stimuli
while keeping the whole experiment duration to an acceptable value.

Furthermore, the stimuli were generated such that a 100 dB SPL value
corresponds to a 0 dBFS value.

\Cref{tab:stimuli} presents all the stimuli used for the different experimental
sections, stating which parameters were fixed and which were varied.

\begin{table}[!ht]
  \centering
  \begin{tabular}{l p{3cm} l}
  \toprule
  \multirow{2}{*}{Section} & \multicolumn{2}{c}{Parameters} \\
  \cmidrule{2-3}
  & Fixed & Varied \\
  \midrule
  AM-fm  & fc = 1 [kHz]\par SPL = 70 [dB]\par md = 40 [dB]
         & fm = \{0, 0.25, 0.5, 1, 2, 4, 8, 16, 32, 64, 128\} [Hz] \\
  \midrule
  AM-fc  & fm = 4 [Hz]\par SPL = 70 [dB]\par md = 40 [dB]
         & fc = \{0.125, 0.25, 0.5, 1, 2, 4, 8\} [kHz] \\
  \midrule
  AM-SPL & fm = 4 [Hz]\par fc = 1 [kHz]\par md = 40 [dB]
         & SPL = \{50, 60, 70, 80, 90\} [dB] \\
  \midrule
  AM-md  & fm = 4 [Hz]\par fc = 1 [kHz]\par SPL = 70 [dB]
         & md = \{1, 2, 4, 10, 20, 40\} [dB] \\
  \midrule
  FM-fm  & fc = 1.5 [kHz]\par SPL = 70 [dB]\par df = 700 [Hz]
         & fm = \{0, 0.25, 0.5, 1, 2, 4, 8, 16, 32, 64, 128\} [Hz] \\
  \midrule
  FM-fc  & fm = 4 [Hz]\par SPL = 70 [dB]\par df = 200 [Hz]
         & fc = \{0.5, 1, 1.5, 2, 3, 4, 6, 8\} [kHz] \\
  \midrule
  FM-SPL & fm = 4 [Hz]\par fc = 1.5 [kHz]\par df = 700 [Hz]
         & SPL = \{40, 50, 60, 70, 80\} [Hz] \\
  \midrule
  FM-df  & fm = 4 [Hz]\par fc = 1.5 [kHz]\par SPL = 70 [dB]
         & df = \{16, 32, 100, 300, 700\} [Hz] \\
  \bottomrule
  \end{tabular}
  \caption{Description of stimuli used per experiment section}
\label{tab:stimuli}
\end{table}

\subsection{Methods}

A magnitude estimation procedure used to obtain the relative fluctuation
strength as a function of a varying parameter for the two types of tones. There
were four different parametrical variations:
\begin{inparaenum}[(1)]
  \item \gls{f_m};
  \item \gls{f_c};
  \item \gls{spl};
  \item \gls{m_d} for \gls{am} tones, \gls{d_f} for \gls{fm} tones.
\end{inparaenum}

Each of these parameters defined an experiment on its own, with the exception of
the \gls{f_m} parameter whose section was split in two due to having a longer
duration that the other experiment sections.

In order to deal with possible learning effects due to the presentation order
of the parametrical sections, a latin square design was used to vary
presentation order among participants.

Each magnitude estimation presented pairs of sounds, composed of one of two
possible standards (\cref{tab:standards}) and a stimulus from the section set.
There were four repetitions per pair, and hence eight per stimulus.

\begin{table}[!ht]
  \centering
  \begin{tabular}{l l l l l l}
    \toprule
    \multirow{2}{*}{Section} & \multicolumn{5}{c}{Parameters} \\
    \cmidrule{2-6}
    & $f_m$ [Hz] & $f_c$ [Hz] & SPL [dB] & $m_d$ [dB] & $d_f$ [Hz] \\
    \midrule
    \multirow{2}{*}{AM-fm}  & 4 & 1000 & 70 & 40 & --- \\
                            & 0.25 & 1000 & 70 & 40 & --- \\
    \midrule
    \multirow{2}{*}{AM-fc}  & 4 & 1000 & 70 & 40 & --- \\
                            & 4 & 250 & 70 & 40 & --- \\
    \midrule
    \multirow{2}{*}{AM-SPL} & 4 & 1000 & 70 & 40 & --- \\
                            & 4 & 1000 & 50 & 40 & --- \\
    \midrule
    \multirow{2}{*}{AM-md}  & 4 & 1000 & 70 & 40 & --- \\
                            & 4 & 1000 & 70 & 4 & --- \\
    \midrule
    \multirow{2}{*}{FM-fm}  & 4 & 1500 & 70 & --- & 700 \\
                            & 0.5 & 1500 & 70 & --- & 700 \\
    \midrule
    \multirow{2}{*}{FM-fc}  & 4 & 6000 & 70 & --- & 200 \\
                            & 4 & 500 & 70 & --- & 200 \\
    \midrule
    \multirow{2}{*}{FM-SPL} & 4 & 1500 & 60 & --- & 700 \\
                            & 4 & 1500 & 40 & --- & 700 \\
    \midrule
    \multirow{2}{*}{FM-df}  & 4 & 1500 & 70 & --- & 700 \\
                            & 4 & 1500 & 70 & --- & 32 \\
    \bottomrule
  \end{tabular}
  \caption{Description of the standards used per experiment section}
\label{tab:standards}
\end{table}

\subsection{Procedure}

Experiment prompt available in appendix

\section{Results}

\begin{results}

\myfigurefastlexpstds%
  {AM-fm}
  {modulation frequency --- AM tones}

\myfigurefastlexpstds%
  {AM-fc}
  {center frequency --- AM tones}

\myfigurefastlexpstds%
  {AM-SPL}
  {sound pressure level --- AM tones}

\myfigurefastlexpstds%
  {AM-md}
  {modulation depth --- AM tones}

\myfigurefastlexpstds%
  {FM-fm}
  {modulation frequency --- FM tones}

\myfigurefastlexpstds%
  {FM-fc}
  {center frequency --- FM tones}

\myfigurefastlexpstds%
  {FM-SPL}
  {sound pressure level --- FM tones}

\myfigurefastlexpstds%
  {FM-df}
  {frequency deviation --- FM tones}

\myfigurequad%
  {AM-fm_all_comparison}
  {AM-fc_all_comparison}
  {AM-SPL_all_comparison}
  {AM-md_all_comparison}
  {
    \caption{Comparison between Fastl and experimental data --- AM tones}
    \label{fig:am_comparison}
  }

\myfigurequad%
  {FM-fm_all_comparison}
  {FM-fc_all_comparison}
  {FM-SPL_all_comparison}
  {FM-df_all_comparison}
  {
    \caption{Comparison between Fastl and experimental data --- FM tones}
    \label{fig:fm_comparison}
  }

\end{results}

\end{document}
