\documentclass[../../main.tex]{subfiles}

\begin{document}

\chapter{Experimental Design and Results}
\label{cha:experiment}

In this chapter the final experimental design and results for the evaluation
of the fluctuation strength attribute are presented.

\section{Design}

\subsection{Subjects}

Twenty-four participants were recruited from the JF Schouten database of the
Eindhoven University of Technology. Participants were between 19 and 31 years
old. There were in total six females and eighteen males. All of them reported
to have normal hearing, however this was not confirmed in any way. Subjects were
paid for their participation.

\subsection{Stimuli}

The stimuli used in the final experiment have the same characteristics as the
ones used in the pilot experiments (\Cref{subsec:pilot_stimuli}).
\Cref{tab:stimuli} shows the stimuli used in the final experiments.

\begin{table}[!ht]
  \centering
  \begin{tabu} to \linewidth{p{2cm}Xp{9cm}}
  \toprule
  \rowfont\bfseries
  \multirow{2}{*}{Section} &
  \multicolumn{2}{c}{Parameters} \\
  \cmidrule{2-3}
  \rowfont\bfseries
  & \multicolumn{1}{c}{Fixed} & \multicolumn{1}{c}{Varied} \\
  \midrule
  AM-fm  & $f_c$ = 1 [kHz]\par SPL = 70 [dB]\par $m_d$ = 40 [dB]
         & $f_m$ = \{0, 0.25, 0.5, 1, 2, 4, 8, 16, 32, 64, 128\} [Hz] \\
  \midrule
  AM-fc  & $f_m$ = 4 [Hz]\par SPL = 70 [dB]\par $m_d$ = 40 [dB]
         & $f_c$ = \{0.125, 0.25, 0.5, 1, 2, 4, 8\} [kHz] \\
  \midrule
  AM-SPL & $f_m$ = 4 [Hz]\par $f_c$ = 1 [kHz]\par $m_d$ = 40 [dB]
         & SPL = \{50, 60, 70, 80, 90\} [dB] \\
  \midrule
  AM-md  & $f_m$ = 4 [Hz]\par $f_c$ = 1 [kHz]\par SPL = 70 [dB]
         & $m_d$ = \{1, 2, 4, 10, 20, 40\} [dB] \\
  \midrule
  FM-fm  & $f_c$ = 1.5 [kHz]\par SPL = 70 [dB]\par $d_f$ = 700 [Hz]
         & $f_m$ = \{0, 0.25, 0.5, 1, 2, 4, 8, 16, 32, 64, 128\} [Hz] \\
  \midrule
  FM-fc  & $f_m$ = 4 [Hz]\par SPL = 70 [dB]\par $d_f$ = 200 [Hz]
         & $f_c$ = \{0.5, 1, 1.5, 2, 3, 4, 6, 8\} [kHz] \\
  \midrule
  FM-SPL & $f_m$ = 4 [Hz]\par $f_c$ = 1.5 [kHz]\par $d_f$ = 700 [Hz]
         & SPL = \{40, 50, 60, 70, 80\} [Hz] \\
  \midrule
  FM-df  & $f_m$ = 4 [Hz]\par $f_c$ = 1.5 [kHz]\par SPL = 70 [dB]
         & $d_f$ = \{16, 32, 100, 300, 700\} [Hz] \\
  \bottomrule
  \end{tabu}
  \caption{Description of stimuli used per experiment section}
\label{tab:stimuli}
\end{table}

\subsection{Procedure}

Participants were assigned either to \gls{AM} tones or to \gls{FM} tones, both
conditions having 12 participants. Furthermore, the presentation order of the
experimental sections was varied, using a latin square design. The order of
the parameters used is presented in \Cref{tab:presentation_order_parameters}.

\begin{table}[!ht]
  \centering
  \begin{tabu} to \linewidth{XX}
    \toprule
    \rowfont\bfseries
    Order\footnotemark[1] & Parameters \\
    \midrule
    1 & \gls{f_m}, \gls{f_c}, \{\gls{m_d} or \gls{d_f}\}, \gls{f_m}, \gls{SPL}\\
    2 & \gls{f_c}, \{\gls{m_d} or \gls{d_f}\}, \gls{f_m}, \gls{SPL}, \gls{f_m}\\
    3 & \{\gls{m_d} or \gls{d_f}\}, \gls{f_m}, \gls{SPL}, \gls{f_m}, \gls{f_c}\\
    4 & \gls{SPL}, \gls{f_m}, \gls{f_c}, \{\gls{m_d} or \gls{d_f}\}, \gls{f_m}\\
    \bottomrule
  \end{tabu}
  \caption{Presentation order of parameters}
\label{tab:presentation_order_parameters}
\end{table}

The whole experiment had an approximate duration of 60~minutes. The experimental
protocol followed during the experiment can be found in the appendix of this
document (\Cref{cha:experimental_protocol}).

The experimental phase of the experiment remained the same as the one of the
last pilot experiment. The only sensible change compared to the procedure
described in \Cref{cha:methods} is the split of the modulation frequency
sections into two separate sections. The training phase had some changes,
described below.

\subsubsection{Training Phase}

Adding all the improvements postulated in the pilot experiment section, the
training phase was expanded and became a 3-part phase, described below.

\paragraph{Stimulus Comparison}

As in the pilot experiments, the initial part of the training phase consisted
of comparison between stimuli. An additional stimulus was added ($f_m = 128$ Hz)
and a subset of \gls{FM} stimuli was also added to complement the \gls{AM}
tones.

\footnotetext[1]{This variable was assigned to participants using the following
equation: $\text{Order}=(\text{Participant ID}-1)\%4+1$}

The stimuli presented in
\Cref{tab:am_training_stimuli,tab:fm_training_stimuli} were reproduced
in groups, according to their ID values. First, stimuli with ID values of 1 and
2 were reproduced. This pair presented the difference between a pure tone and
a modulated tone. Then, stimuli with ID values of 2 and 3 followed. This pair
presented the difference between modulated tones with low and high values of
fluctuation strength. Finally, stimuli with ID values of 3, 4 and 5 were
reproduced. This last group presented the difference between fluctuating and
rough tones. After each group presentation, participants were asked whether the
specific difference of sensation of the group was acknowledged. In case of a
negative answer, the stimuli of the group were once again reproduced.

\begin{table}[!ht]
  \centering
  \begin{tabu} to \linewidth{XXXXX}
    \toprule
    \rowfont\bfseries
    ID & $\bm{f_m}$ [Hz] & $\bm{f_c}$ [kHz] & SPL [dB] & $\bm{m_d}$ [dB] \\
    \midrule
    1 & 0   & 1 & 70 & 40 \\
    2 & 0.5 & 1 & 70 & 40 \\
    3 & 4   & 1 & 70 & 40 \\
    4 & 32  & 1 & 70 & 40 \\
    5 & 128 & 1 & 70 & 40 \\
    \bottomrule
  \end{tabu}
  \caption{Subset of AM stimuli for training phase}
\label{tab:am_training_stimuli}
\end{table}

\begin{table}[!ht]
  \centering
  \begin{tabu} to \linewidth{XXXXX}
    \toprule
    \rowfont\bfseries
    ID & $\bm{f_m}$ [Hz] & $\bm{f_c}$ [kHz] & SPL [dB] & $\bm{d_f}$ [Hz] \\
    \midrule
    1 & 0   & 1 & 70 & 700 \\
    2 & 0.5 & 1 & 70 & 700 \\
    3 & 4   & 1 & 70 & 700 \\
    4 & 32  & 1 & 70 & 700 \\
    4 & 128 & 1 & 70 & 700 \\
    \bottomrule
  \end{tabu}
  \caption{Subset of FM stimuli for training phase}
\label{tab:fm_training_stimuli}
\end{table}

\paragraph{Long Interval}

This part of the training phase presented participants with long intervals,
which consisted of stimuli separated by 800~ms of silence. Two long intervals
were reproduced, composed of \gls{AM} and \gls{FM} tones, respectively. These
two intervals are described in \Cref{tab:am_all_stimulus,tab:fm_all_stimulus}.

\begin{table}[!ht]
  \centering
  \begin{tabu} to \linewidth{XXXXX}
    \toprule
    \rowfont\bfseries
    Presentation & \multirow{2}{*}{$\bm{f_m}$ [Hz]} & \multirow{2}{*}{$\bm{f_c}$ [kHz]} & \multirow{2}{*}{SPL [dB]} & \multirow{2}{*}{$\bm{m_d}$ [dB]} \\
    \rowfont\bfseries
    order \\
    \midrule
    1  & 0.5  & 1 & 70 & 40 \\
    2  & 32   & 1 & 70 & 40 \\
    3  & 2    & 1 & 70 & 40 \\
    4  & 16   & 1 & 70 & 40 \\
    5  & 4    & 1 & 70 & 40 \\
    6  & 1    & 1 & 70 & 40 \\
    7  & 0    & 1 & 70 & 40 \\
    8  & 64   & 1 & 70 & 40 \\
    9  & 0.25 & 1 & 70 & 40 \\
    10 & 128  & 1 & 70 & 40 \\
    11 & 8    & 1 & 70 & 40 \\
    \bottomrule
  \end{tabu}
  \caption{Long interval composed of \gls{AM} stimuli for training
  phase}
\label{tab:am_all_stimulus}
\end{table}

\begin{table}[!ht]
  \centering
  \begin{tabu} to \linewidth{XXXXX}
    \toprule
    \rowfont\bfseries
    Presentation & \multirow{2}{*}{$\bm{f_m}$ [Hz]} & \multirow{2}{*}{$\bm{f_c}$ [kHz]} & \multirow{2}{*}{SPL [dB]} & \multirow{2}{*}{$\bm{d_f}$ [Hz]} \\
    \rowfont\bfseries
    order \\
    \midrule
    1  & 0.25 & 1 & 70 & 700 \\
    2  & 64   & 1 & 70 & 700 \\
    3  & 32   & 1 & 70 & 700 \\
    4  & 2    & 1 & 70 & 700 \\
    5  & 1    & 1 & 70 & 700 \\
    6  & 0    & 1 & 70 & 700 \\
    7  & 8    & 1 & 70 & 700 \\
    8  & 4    & 1 & 70 & 700 \\
    9  & 0.5  & 1 & 70 & 700 \\
    10 & 128  & 1 & 70 & 700 \\
    11 & 16   & 1 & 70 & 700 \\
    \bottomrule
  \end{tabu}
  \caption{Long interval composed of \gls{FM} stimuli for training
  phase}
\label{tab:fm_all_stimulus}
\end{table}

\paragraph{Test Section}

In order to familiarize participants with the interface used during the
experiment and with the magnitude estimation procedure, a small test section
was added at the end of the training phase. This section consisted of four
pairs (\Cref{tab:pairs_test_section}), which were presented to the participants
in randomized order.

\begin{table}[!ht]
  \centering
  \begin{tabu} to \linewidth{XXXXXX}
    \toprule
    \rowfont\bfseries
    \multirow{2}{*}{Pair} &
    \multicolumn{5}{c}{Parameters} \\
    \cmidrule{2-6}
    \rowfont\bfseries
    & $\bm{f_m}$ [Hz] & $\bm{f_c}$ [kHz] & SPL [dB] & $\bm{m_d}$ [dB] & $\bm{d_f}$ [Hz] \\
    \midrule
    \multirow{2}{*}{1} & 4  & 1 & 70 & 40 & --- \\
                       & 32 & 1 & 70 & 40 & --- \\
    \midrule
    \multirow{2}{*}{2} & 4  & 6 & 70 & --- & 200 \\
                       & 4  & 6 & 70 & --- & 200 \\
    \midrule
    \multirow{2}{*}{3} & 4  & 1 & 70 & 40 & --- \\
                       & 0  & 1 & 70 & 40 & --- \\
    \midrule
    \multirow{2}{*}{4} & 4  & 1.5 & 60 & --- & 700 \\
                       & 4  & 1.5 & 80 & --- & 700 \\
    \bottomrule
  \end{tabu}
  \caption{Pairs used in training phase test section}
\label{tab:pairs_test_section}
\end{table}

\section{Results}

The following section presents the results of the experiments, compared to the
data published by \textcite{Fastl2007Psychoacoustics}. Overall the obtained data
is qualitatively similar to the data by \citeauthor{Fastl2007Psychoacoustics},
although some differences do exist. Most notably, in the modulation depth
response curve for \gls{AM} tones and in the center frequency and frequency
deviations curves for \gls{FM} tones. In the following paragraphs the obtained
curves will be described, focusing on similarities and discrepancies with the
literature data. Possible causes of this discrepancies will be discussed in
\Cref{cha:discussion}.

\Cref{fig:main-fastlexpstds_AM-fm,fig:main-fastlexpstds_FM-fm} show the
dependency of fluctuation strength on modulation frequency for \gls{AM} and
\gls{FM} tones. In both cases, the characteristic band-pass response was
obtained. The maximum values for both curves occurs at around 4~Hz, as expected.
However, for values of modulation frequency below 4~Hz on average higher values
of fluctuation were obtained. This leads to the fact that the response from the
obtained data has a wider bandwidth than the data from the literature. Also,
more variability seems to exist with the use of the first standard, evidenced by
the difference of length between the larger \gls{IQR}s of the first standard
and the smaller \gls{IQR}s of the second standard.

\Cref{fig:main-fastlexpstds_AM-fc} shows the dependency of fluctuation strength
on center frequency from \gls{AM} tones. In this case both responses present a
similar flat response with large \gls{IQR}s. \Cref{fig:main-fastlexpstds_AM-fc}
shows the dependency of fluctuation strength on center frequency from \gls{FM}
tones. Here the difference is more dramatic, the data from the literature
decreases monotonically with the increase of center frequency, whereas the data
from this study remains mostly flat.

\Cref{fig:main-fastlexpstds_AM-SPL,fig:main-fastlexpstds_FM-SPL} show the
dependency of fluctuation strength on sound pressure level for \gls{AM} and
\gls{FM} tones. Although the curves from the obtained data are not as linear as
the data from the literature, in both cases fluctuation strength increases with
sound pressure level.

\Cref{fig:main-fastlexpstds_AM-md} shows the dependency of fluctuation strength
on modulation depth from \gls{AM} tones. Here a difference exists between the
obtained data and the literature data. Both response curves show an increase
of fluctuation strength with the increase of modulation depth. However, the
obtained data increases more quickly with modulation depth than the data from
the literature.

\Cref{fig:main-fastlexpstds_FM-df} shows the dependency of fluctuation strength
on frequency deviation from \gls{FM} tones. In this case a clear difference
exists between the obtained data and the literature data. For the obtained data,
for small values of frequency deviation a significant fluctuation strength
(around 50\%) does exist. This causes that the curve presents a less steep slope
when compared to the literature data. Both curves present an increase in
fluctuation strength with an increase in frequency deviation.

Finally, to summarize \Cref{fig:am_comparison,fig:fm_comparison} compared the
mean of the values of the two standard for each experimental condition. The
discrepancies between obtained data and literature data can be further observed
in these figures.

\begin{experimentalresults}

\myfigurefastlexpstds%
  {AM-fm}
  {Relative fluctuation strength as a function of modulation frequency for
    \gls{AM} tones with center frequency of 1~kHz, sound pressure level of
    70~dB and modulation depth of 40~dB.  The two standards had modulation
    frequencies of 4 and 0.5~Hz. The data points show the median and
    interquartile ranges per standard. The black line represents the mean values
    of the medians of each standard. Panel~(a): data adapted from
    \cite[pp.248]{Fastl2007Psychoacoustics}. Panel~(b): own results}
  {main}

\myfigurefastlexpstds%
  {AM-fc}
  {Relative fluctuation strength as a function of center frequency for
    \gls{AM} tones with modulation frequency of 4~Hz, sound pressure level of
    70~dB and modulation depth of 40~dB.  The two standards had center
    frequencies of 1 and 0.25~kHz. The data points show the median and
    interquartile ranges per standard. The black line represents the mean values
    of the medians of each standard. Panel~(a): data adapted from
    \cite[pp.250]{Fastl2007Psychoacoustics}. Panel~(b): own results}
  {main}

\myfigurefastlexpstds%
  {AM-SPL}
  {Relative fluctuation strength as a function of sound pressure level for
    \gls{AM} tones with modulation frequency of 4~Hz, center frequency of 1~kHz
    and modulation depth of 40~dB.  The two standards had sound pressure levels
    of 70 and 50~dB. The data points show the median and
    interquartile ranges per standard. The black line represents the mean values
    of the medians of each standard. Panel~(a): data adapted from
    \cite[pp.249]{Fastl2007Psychoacoustics}. Panel~(b): own results}
  {main}

\myfigurefastlexpstds%
  {AM-md}
  {Relative fluctuation strength as a function of modulation depth for
    \gls{AM} tones with modulation frequency of 4~Hz, center frequency of 1~kHz
    and sound pressure level of 70~dB.  The two standards had modulation depths
    of 40 and 4~dB. The data points show the median and
    interquartile ranges per standard. The black line represents the mean values
    of the medians of each standard. Panel~(a): data adapted from
    \cite[pp.249]{Fastl2007Psychoacoustics}. Panel~(b): own results}
  {main}

\myfigurefastlexpstds%
  {FM-fm}
  {Relative fluctuation strength as a function of modulation frequency for
    \gls{FM} tones with center frequency of 1.5~kHz, sound pressure level of
    70~dB and frequency deviation of 700 Hz.  The two standards had modulation
    frequencies of 4 and 0.5~Hz. The data points show the median and
    interquartile ranges per standard. The black line represents the mean values
    of the medians of each standard. Panel~(a): data adapted from
    \cite[pp.248]{Fastl2007Psychoacoustics}. Panel~(b): own results}
  {main}

\myfigurefastlexpstds%
  {FM-fc}
  {Relative fluctuation strength as a function of center frequency for
    \gls{FM} tones with modulation frequency of 4~Hz, sound pressure level of
    70~dB and frequency deviation of 200~Hz.  The two standards had center
    frequencies of 6 and 0.5~kHz. The data points show the median and
    interquartile ranges per standard. The black line represents the mean values
    of the medians of each standard. Panel~(a): data adapted from
    \cite[pp.250]{Fastl2007Psychoacoustics}. Panel~(b): own results}
  {main}

\myfigurefastlexpstds%
  {FM-SPL}
  {Relative fluctuation strength as a function of sound pressure level for
    \gls{FM} tones with modulation frequency of 4~Hz, center frequency of
    1.5~kHz and frequency deviation of 700~Hz.  The two standards had sound
    pressure levels of 60 and 40~dB. The data points show the median and
    interquartile ranges per standard. The black line represents the mean values
    of the medians of each standard. Panel~(a): data adapted from
    \cite[pp.249]{Fastl2007Psychoacoustics}. Panel~(b): own results}
  {main}

\myfigurefastlexpstds%
  {FM-df}
  {Relative fluctuation strength as a function of modulation depth for
    \gls{FM} tones with modulation frequency of 4~Hz, center frequency of
    1.5~kHz and sound pressure level of 70~dB.  The two standards had frequency
    deviations of 700 and 32~Hz. The data points show the median and
    interquartile ranges per standard. The black line represents the mean values
    of the medians of each standard. Panel~(a): data adapted from
    \cite[pp.251]{Fastl2007Psychoacoustics}. Panel~(b): own results}
  {main}

\myfigurequadlabeled%
  {AM-fm_all_comparison}
  {AM-fc_all_comparison}
  {AM-SPL_all_comparison}
  {AM-md_all_comparison}
  {Relative fluctuation strength for AM tones as a function of: (a) modulation
    frequency, (b) center frequency, (c) sound pressure level, (d) modulation
    depth. The black solid line corresponds to the data from
    \textcite{Fastl2007Psychoacoustics}. The blue dashed line corresponds to the
    data from this study. Both curves represents the mean value of the two used
    standards}
  {am_comparison}

\myfigurequadlabeled%
  {FM-fm_all_comparison}
  {FM-fc_all_comparison}
  {FM-SPL_all_comparison}
  {FM-df_all_comparison}
  {Relative fluctuation strength for FM tones as a function of: (a) modulation
    frequency, (b) center frequency, (c) sound pressure level, (d) frequency
    deviation. The black solid line corresponds to the data from
    \textcite{Fastl2007Psychoacoustics}. The blue dashed line corresponds to the
    data from this study. Both curves represents the mean value of the two used
    standards}
  {fm_comparison}

\end{experimentalresults}

\end{document}
