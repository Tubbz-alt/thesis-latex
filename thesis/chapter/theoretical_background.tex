\documentclass[../main.tex]{subfiles}

\begin{document}

\chapter{Theoretical Background}

In order to introduce the topic of auditory perceptual attributes, a small
overview of the perceptual processes and their associated perceptual quantities
is presented. \Cref{tab:stimsens} summarizes all the perceptual measures, along
with their dominant physical stimuli. It is interesting to note that all these
perceptual dimensions, except for density, were derived from the
``Munich school'' work of Zwicker and Fastl~\cite{Fastl2007Psychoacoustics}.

\begin{table}[ht]
  \centering
  \begin{tabu}{ l l }
    \toprule
    \rowfont\bfseries
    Dominant stimuli & Cognitive parameters \\
    \midrule
    Sound pressure level (dB) & Loudness (sone) \\
    \cmidrule{2-2}
    & Loudness level (phon) \\
    \midrule
    Frequency (Hz) & Critical band rate (Bark) \\
    \cmidrule{2-2}
    & Ratio pitch (mel) \\
    \midrule
    Degree of modulation (\%) & Roughness (asper)\\
    \cmidrule{1-1}
    Modulation frequency (Hz) & \\
    \midrule
    Frequency (Hz) & Sharpness (acum) \\
    \midrule
    Degree of modulation (\%) & Fluctuation strength (vacil) \\
    \cmidrule{1-1}
    Modulation frequency (Hz) & \\
    \midrule
    Spectral components (Pa) & Pitch strength \\
    \cmidrule{2-2}
    & Tonality (tu) \\
    \midrule
    Impulse duration (s) & Subjective duration of impetus (IU) \\
    \midrule
    Sound pressure level (dB) & Density (dasy) \\
    Frequency (Hz) & \\
    \bottomrule
  \end{tabu}
  \caption{Stimuli and sensations~\cite[pp.~70]{Mueller2012Handbook}}
  \label{tab:stimsens}
\end{table}

It is important to note that the human auditory system can generate these
perceptual sensations independently of each other, although to understand the
general ``pleasantness'' of a sound a bigger context needs to be taken into
account. For instance the emotions of the listeners can have an important
effect on the cognitive construal of a given sound.

\end{document}
