\documentclass[../main.tex]{subfiles}

\begin{document}

\chapter{Theoretical Background}
\label{cha:theoretical}

In this chapter the theoretical background for the sensation of fluctuation
strength is presented. First, an introduction to the topic of perceptual
attributes is given. Afterwards, the sensation of fluctuation strength is
addressed, exploring its dependencies on stimulus parameters; most of this
information comes out from the work done by \textcite{Fastl2007Psychoacoustics}.
Next, literature concerning the modeling of fluctuation strength is given.
Finally, several studies expanding upon the basic concepts of fluctuation
strength are presented.

\begin{theoreticalbackground}

\section{Perceptual Attributes}

Perceptual attributes are discernible dimensions in which an auditory event can
be decomposed. They are derived from the physical characteristics of sounds. By
using them the effects of incoming audio events can be understood from a
perceptual point of view. A small overview of the perceptual processes and their
associated perceptual quantities is presented in \Cref{tab:stimsens}, which
summarizes all the perceptual measures, along with their dominant physical
stimuli. Most of the research on perceptual dimensions, except for density,
comes from the ``Munich school'' work of \textcite{Fastl2007Psychoacoustics}.

\begin{table}[ht]
  \centering
  \begin{tabu} to \linewidth{ X X }
    \toprule
    \rowfont\bfseries
    Dominant stimulus dimension & Perceptual parameters \\
    \midrule
    Sound pressure level (dB) & Loudness (sone) \\
    \cmidrule{2-2}
    & Loudness level (phon) \\
    \midrule
    Frequency (Hz) & Critical band rate (Bark) \\
    \cmidrule{2-2}
    & Ratio pitch (mel) \\
    \midrule
    Degree of modulation (\%) & Roughness (asper)\\
    \cmidrule{1-1}
    Modulation frequency (Hz) & \\
    \midrule
    Frequency (Hz) & Sharpness (acum) \\
    \midrule
    Degree of modulation (\%) & Fluctuation strength (vacil) \\
    \cmidrule{1-1}
    Modulation frequency (Hz) & \\
    \midrule
    Spectral components (Pa) & Pitch strength \\
    \cmidrule{2-2}
    & Tonality (tu) \\
    \midrule
    Impulse duration (s) & Subjective duration of impetus (IU) \\
    \midrule
    Sound pressure level (dB) & Density (dasy) \\
    Frequency (Hz) & \\
    \bottomrule
  \end{tabu}
  \caption{Stimuli and sensations~\cite[pp.~70]{Mueller2012Handbook}}
\label{tab:stimsens}
\end{table}

It is important to note that the human auditory system can generate these
perceptual sensations independently of each other, although to understand the
psychological impact of them, for instance the ``pleasantness'' of a given
sound, a bigger context needs to be taken into account. As an example, the
emotions of the listeners can have an important effect on the cognitive
construal of a given sound.

\section{Fluctuation Strength}

Fluctuation strength corresponds to the sensation that arises when a sound
has a slowly varying envelope (i.e., a modulation signal whose frequency is less
than 20~Hz). Fluctuation strength is closely related to roughness, the
difference between the two being the range of modulation frequencies where each
sensation is predominant. In the case of roughness, this range corresponds to
modulation frequency values greater than 20~Hz.

Fluctuation strength can have a significant effect on the pleasantness of sound,
and a particularly clear example of this are alarms, which must have a sharp and
distinctive sound.

The effect of fluctuation strength can be seen as a temporary masking pattern on
the original signal, in which the modulation depth is of utmost importance. This
is also the case for roughness, which resembles fluctuation strength in this
regard.

\subsection{Dependencies of Fluctuation Strength}

The unit used to quantify the sensation of fluctuation strength is the vacil. As
a reference, 1 vacil is defined as the fluctuation strength created by a 1~kHz
tone with a \gls{SPL} of 60~dB, an \gls{AM} envelope of 4~Hz and a modulation
index of 1. The maximum value of fluctuation strength seems to occur for
modulation rates around 4~Hz, regardless of the modulation technique used
(\Cref{fig:flucstrenvmodfreq}).

\begin{figure}[!ht]
  \centering
  \includegraphics[width=\textwidth]{FluctuationStrengthVsModulationFrequency}
  \caption{Fluctuation strength as a function of modulation frequency for (a)
    amplitude-modulated broad-band noises, (b) amplitude-modulated tones and (c)
    frequency-modulated tones~\cite[pp.~248]{Fastl2007Psychoacoustics}}
\label{fig:flucstrenvmodfreq}
\end{figure}

It seems that a relation between fluctuation strength and speech production
exists, as the normal production rate of syllables during normal conversation
speed is about 4 syllables/second. This coincides with the frequency in which a
maximum value of fluctuation strength occurs (4~Hz).

\Cref{fig:flucstrenvsndpreslvl} shows the relation between \gls{SPL} and
fluctuation strength for two different stimuli, tones and \gls{BBN}, and two
modulation techniques, amplitude modulation and frequency modulation. An
increase in \gls{SPL} entails an increase of fluctuation strength, this being
stronger for the \gls{AM} tones.

\begin{figure}[!ht]
  \centering
  \includegraphics[width=\textwidth]{FluctuationStrengthVsSoundPressureLevel}
  \caption{Fluctuation strength as a function of sound pressure level for (a)
    amplitude-modulated broad-band noises, (b) amplitude-modulated tones and (c)
    frequency-modulated tones; modulation frequency of
    4Hz~\cite[pp.~249]{Fastl2007Psychoacoustics}}
\label{fig:flucstrenvsndpreslvl}
\end{figure}

Next the effect of modulation depth on fluctuation strength is analyzed, shown
in \Cref{fig:flucstrenvsmoddep}. It can be observed that, between 3~dB and 30~dB
the relation between fluctuation strength and modulation depth is somewhat
linear. After reaching a maximum value at around 30~dB, which corresponds to a
modulation factor of 94\%, fluctuation strength remains constant with further
increments of modulation depth.

\begin{figure}[!ht]
  \centering
  \includegraphics[width=\textwidth]{FluctuationStrengthVsModulationDepth}
  \caption{Fluctuation strength as a function of modulation depth for (a)
    amplitude-modulated broad-band noises of 60~dB SPL and (b)
    amplitude-modulated tones of 70~dB SPL and 1~kHz frequency; both with a
    modulation frequency of 4~Hz~\cite[pp.~249]{Fastl2007Psychoacoustics}}
\label{fig:flucstrenvsmoddep}
\end{figure}

The relation between the center frequency and fluctuation strength is shown in
\Cref{fig:flucstrenvscfreq}. Here a clear difference exists between the type of
modulation used. For \gls{AM} tones there is a small variation in the
fluctuation strength, implying this dependence of fluctuation strength on the
center frequency. For \gls{FM} tones a clear dependence between
both variables exists. For center frequencies below 1~kHz the fluctuation
strength is almost constant; above 1~kHz it experiences a linear decrease until
it fades away at around 8~kHz.

\begin{figure}[!ht]
  \centering
  \includegraphics[width=\textwidth]{FluctuationStrengthVsCenterFrequency}
  \caption{Fluctuation strength as a function of center frequency for an
    amplitude-modulated tone of 70~dB SPL, 4~Hz modulation frequency and 40~dB
    modulation depth (a), and a frequency-modulated tone with 70~dB SPL, 4~Hz
    modulation frequency and $\pm200$ Hz frequency
    deviation~\cite[pp. 250]{Fastl2007Psychoacoustics}}
\label{fig:flucstrenvscfreq}
\end{figure}

To understand why this change of fluctuation strength occurs in the case of the
\gls{FM} tones, it is necessary to take into account the excitation patterns
that the modulated sounds cause. The auditory filters are a series of
overlapping bandpass filters that model the frequency selective response of the
auditory system. The excitation pattern refers to the pattern that results as
the outcome of all the precedent stages of the hearing process, being the outer
ear transmission, middle ear transmission, and the auditory filters themselves.

In the case of fluctuation strength, for a 0.5~kHz tone the frequency would vary
between 300 and 700~Hz, this corresponding to a 3.5~Bark interval. For a 8~kHz
these values would be 7.8~kHz and 8.2~kHz, leading to a 0.2~Bark interval. The
proportion between these two is 17.5, which seems to be also the proportion
between the relative fluctuation strength for these two frequencies. Thus, this
leads to the idea that fluctuation strength can be explained in terms of the
excitation patterns that present themselves across the auditory filters.

\Cref{fig:flucstrenvsfreqdev} shows the relation between fluctuation strength
and frequency deviation. It can be seen that, for frequencies above 20~Hz, there
is a linear increase of fluctuation strength with frequency deviation.

\begin{figure}[!ht]
  \centering
  \includegraphics[height=8cm]{FluctuationStrengthVsFrequencyDeviation}
    \caption{Fluctuation strength as a function of frequency deviation for a
      tone with 70~dB SPL, 1.5~kHz center frequency and a modulation frequency
      of 4~Hz~\cite[pp. 251]{Fastl2007Psychoacoustics}}
\label{fig:flucstrenvsfreqdev}
\end{figure}

Another variable that affects fluctuation strength is the modulation index $h$
(also called modulation factor $m$). It is defined differently according to the
type of modulation used:
\begin{itemize}
  \item For \gls{AM} signals is the ratio between the modulating signal
    amplitude $M$ and the carrier signal amplitude $A$,
      \begin{equation}
        y_{am} = [1+h \cdot \sin(2\pi f_m t)]\cdot A \cdot \sin(2\pi f_c t)
      \end{equation}
      \begin{equation}
        h=\frac{M}{A}
      \end{equation}
  \item For \gls{FM} signals is the ratio between the maximum frequency
    deviation in the carrier signal ($\Delta f$) and the maximum frequency
    component of the modulating signal ($f_m$),
      \begin{equation}
        y_{fm} = A \cdot \sin[2\pi f_c t + h \cdot \sin(2 \pi f_m t)]
      \end{equation}
      \begin{equation}
        h=\frac{\Delta f}{f_m}
      \end{equation}
\end{itemize}

It seems that a significant fluctuation strength (10\% of the relative
fluctuation strength~\cite[pp.~251]{Fastl2007Psychoacoustics}) is achieved with
a modulation index that would correspond to about 10 times the \gls{JND} for
modulation frequency. This would relate both the thresholds of modulation
frequency and fluctuation strength.

\Cref{fig:flucstrensnds} compares the fluctuation strength of several sounds,
whose physical characteristics are described in \Cref{tab:flucstrensnds}. The
sounds that present the largest values of fluctuation strength (sounds 1 and 2)
excite a large range of frequencies and therefore more than one auditory filter.
As so, it can be said that fluctuation strength sums across critical bands.

\begin{figure}[!ht]
  \centering
  \includegraphics[height=8cm]{FluctuationStrengthSounds}
  \caption{Fluctuation strength of sounds 1--5 as described in
    \Cref{tab:flucstrensnds}~\cite[pp. 252]{Fastl2007Psychoacoustics}}
\label{fig:flucstrensnds}
\end{figure}

\begin{table}[!ht]
  \centering
  \begin{tabu} to \linewidth{ lXXXXX }
    \toprule
    Sound & 1 & 2 & 3 & 4 & 5 \\
    \midrule
    Abbreviation & FM & AM & AM & FM & \\
    & SIN & BBN & SIN & SIN & NBN \\
    Frequency [Hz] & 1500 & --- & 2000 & 1500 & 1000 \\
    Level [dB] & 70 & 60 & 70 & 70 & 70 \\
    Modulation frequency [Hz] & 4 & 4 & 4 & 4 & --- \\
    Modulation depth [dB] & --- & 40 & 40 & --- & --- \\
    Frequency deviation [Hz] & 700 & --- & --- & 32 & --- \\
    Bandwidth [Hz] & --- & 16000 & --- & --- & 10 \\
    \bottomrule
  \end{tabu}
  \caption{Physical data of sounds
    1--5~\cite[pp. 253]{Fastl2007Psychoacoustics}}
\label{tab:flucstrensnds}
\end{table}

\subsection{Models of Fluctuation Strength}

A basic model for fluctuation strength (proposed by
\textcite[pp.~254]{Fastl2007Psychoacoustics}) based on the temporal variation of
the masking pattern of this sound is shown in \Cref{fig:flucstrenmodel}, where
the temporal variation of the amplitude of the masking pattern, also called
temporal masking depth, is denoted by the magnitude $\Delta L$. The inverse of
the time difference between peaks corresponds to the \gls{f_m}.

\begin{figure}[!ht]
  \centering
  \includegraphics[width=\textwidth]{FluctuationStrengthModel}
  \caption{Model of fluctuation
    strength~\cite[pp. 254]{Fastl2007Psychoacoustics}}
\label{fig:flucstrenmodel}
\end{figure}

\Cref{eq:flucstrentempmaskmodfreq} shows the relationship between fluctuation
strength $F$, temporal masking depth $\Delta L$ and modulation frequency
$f_{m}$, where the importance of the 4 Hz frequency is emphasized.

\begin{equation}
  F \sim \frac{\Delta L}{(f_{m}/4\text{ Hz}) + (4\text{ Hz}/f_{m})}
  \label{eq:flucstrentempmaskmodfreq}
\end{equation}

It should be noted that there is a dependency between the temporal masking depth
$\Delta L$ and the modulation frequency depending on the type of stimuli. In the
case of broad-band noise, temporal masking depth seems largely unaffected by
modulation frequency, whereas amplitude and frequency modulates tones these two
variable are dependent on each other, this being strongest for \gls{FM} stimuli.
In order to address this, when modeling the fluctuation strength for these tones
not a single $\Delta L$ value is taken, but instead it is integrated across the
critical-band rate scale.

The resulting temporal masking pattern for several values of modulation
frequency is shown in \Cref{fig:flucstrenmasking}. It can be seen that, as the
modulation frequency increases, the temporal masking depth decreases. This leads
to the idea that, although fluctuation strength presents a bandpass response
with respect to modulation frequency, the temporal masking reveals a low pass
characteristic. It can be considered that the temporal masking depth decreases
linearly with modulation frequency, as shown on \Cref{fig:flucstrenmasking}.

\begin{figure}[!ht]
  \centering
  \includegraphics[width=\textwidth]{FluctuationStrengthTemporalMasking}
  \caption{Temporal masking pattern for an amplitude-modulated broad-band
    noise~\cite[pp. 255]{Fastl2007Psychoacoustics}}
\label{fig:flucstrenmasking}
\end{figure}

The modeling based on the temporal masking property, proposed by
\citeauthor{Fastl2007Psychoacoustics}, poses difficulties when its
implementation details are addressed. Furthermore, to our knowledge, there is
not a publicly available implementation for any given model of fluctuation
strength up to this date. There are, however, implementation models when it
comes to roughness, a sensation that it was already mentioned as being similar
to fluctuation strength regarding its physical characteristics. The most
prominent of these models is the one developed by
\textcite{daniel1997psychoacoustical}, which is based on the dependency of
roughness on modulation depth, shown in \Cref{eq:R}.

\begin{equation}
  R \sim m^p
  \label{eq:R}
\end{equation}

Their model propose a generalized modulation depth function, intended to model
the modulation depth dependence per critical filterbank channel and obtain
specific roughness values. Then a total overall roughness value is obtained by
summing the specific roughness values.

Given the similarity in physical terms that fluctuation strength and roughness
possess,  model for the sensation of fluctuation strength could be elaborated
taking as a base \citeauthor{daniel1997psychoacoustical}'s model. In this
regard, \textcite{Sontacchi1998} has already formulated a model whose structure
is similar to that of the roughness model. Hovewer, as part of work done a
subjective evaluation of the model used participants was not presented. The
present work expands on this point, including a subjective evaluation of the
model with the experimental data collected.

\subsection{Related Studies}

Although the work by \citeauthor{Fastl2007Psychoacoustics} is the most extensive
reference for the fluctuation strength sensation, other studies have been
carried out to further investigate the phenomenon.
\textcite{Accolti2009Fluctuation} applied a model based on the temporal masking
envelope applied to sounds composed of two mixed \gls{AM} sources. They found
that the model could not be extended to this type of stimuli. However, the most
remarkable part of their study is the inclusion of a training phase before they
performed the actual experiment. This is supported by the fact that individuals
are not familiarized with the concept of fluctuation strength; which might lead
to erroneous answers.

Building upon this last point, \citeauthor{Wickelmaier2004Scaling} carried out
a three-part experiment that came out with similar results in that sense. First,
a full-factorial design with 54 pairs of stimuli, with nine modulation
frequencies and six modulation depths, and a magnitude estimation test was run.
The results differ from those reported by \citeauthor{Fastl2007Psychoacoustics},
since they do not show the characteristic band-pass response as a function of
modulation rate. The second experiment tried to assess the contribution of the
two factors used in the past experiment (modulation frequency and modulation
depth) separately, by varying one while leaving the other constant. It was found
that the variation of the modulation depth is similar to
\citeauthor{Fastl2007Psychoacoustics} data, while the variation of the
modulation frequency was not. In the last experiment they tried to assess
whether perceived fluctuation can be represented by an additive combination of
modulation frequency and modulation depth. To test this they used the Thomsen
condition. Their results suggest that listeners do not integrate these two
variables as an unidimensional percept. \citeauthor{Wickelmaier2004Scaling}
conclusions are that \citeauthor{Fastl2007Psychoacoustics} data do not conform
properly to their data, and that the status of fluctuation strength as a basic
auditory perceptual attribute could be debated. The discrepancy between the two
data sets could also be attributed to the lack of understanding of participants,
as \citeauthor{Accolti2009Fluctuation} stated.

\end{theoreticalbackground}

\end{document}
