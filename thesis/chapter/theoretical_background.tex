\documentclass[../main.tex]{subfiles}

\begin{document}

\chapter{Theoretical Background}

In this chapter the theoretical background for the sensation of fluctuation
strength is presented. First, an introduction to the topic of perceptual
attributes is given. Afterwards, the sensation of fluctuation strength is
addressed, exploring its dependencies on stimuli parameters; most of this
information comes out from the work done by \textcite{Fastl2007Psychoacoustics}.
Next, literature concerning the modeling of fluctuation strength is given.
Finally, several studies expanding upon the basics concepts of fluctuation
strength  are presented.

\begin{theoreticalbackground}

\section{Perceptual Attributes}

Perceptual attributes are discernible dimensions in which an auditory event can
be decomposed. They are derived from physical characteristic of sounds. By using
them the effects of incoming audio events can be understood from a perceptual
point of view. A small overview of the perceptual processes and their associated
perceptual quantities is presented in \Cref{tab:stimsens}, which summarizes all
the perceptual measures, along with their dominant physical stimuli. Most of the
research on perceptual dimensions, except for density, comes from the
``Munich school'' work of \textcite{Fastl2007Psychoacoustics}.

\begin{table}[ht]
  \centering
  \begin{tabu} to \linewidth{ X X }
    \toprule
    \rowfont\bfseries
    Dominant stimuli & Cognitive parameters \\
    \midrule
    Sound pressure level (dB) & Loudness (sone) \\
    \cmidrule{2-2}
    & Loudness level (phon) \\
    \midrule
    Frequency (Hz) & Critical band rate (Bark) \\
    \cmidrule{2-2}
    & Ratio pitch (mel) \\
    \midrule
    Degree of modulation (\%) & Roughness (asper)\\
    \cmidrule{1-1}
    Modulation frequency (Hz) & \\
    \midrule
    Frequency (Hz) & Sharpness (acum) \\
    \midrule
    Degree of modulation (\%) & Fluctuation strength (vacil) \\
    \cmidrule{1-1}
    Modulation frequency (Hz) & \\
    \midrule
    Spectral components (Pa) & Pitch strength \\
    \cmidrule{2-2}
    & Tonality (tu) \\
    \midrule
    Impulse duration (s) & Subjective duration of impetus (IU) \\
    \midrule
    Sound pressure level (dB) & Density (dasy) \\
    Frequency (Hz) & \\
    \bottomrule
  \end{tabu}
  \caption{Stimuli and sensations~\cite[pp.~70]{Mueller2012Handbook}}
\label{tab:stimsens}
\end{table}

It is important to note that the human auditory system can generate these
perceptual sensations independently of each other, although to understand the
psychological impact of them, for instance the ``pleasantness'' of a given
sound, a bigger context needs to be taken into account. As an example, the
emotions of the listeners can have an important effect on the cognitive
construal of a given sound.

\section{Fluctuation Strength}

Fluctuation strength corresponds to the sensation that arises when a sound
presents a slow envelope (i.e., a modulation signal whose frequency is less than
20~Hz). Fluctuation strength is closely related to roughness, the difference
between the two being the range of modulation frequencies where each sensation
is predominant. In the case of roughness, this range corresponds to modulation
frequency values greater than 20~Hz.

Fluctuation strength can have a significant effect on the pleasantness of sound,
and a particularly clear example of this are alarms, which must have a sharp and
distinctive sound.

The effect of fluctuation strength can be seen as a temporary masking pattern on
the original signal, in which the modulation depth is of utmost importance. This
is also the case for roughness, which resembles fluctuation strength in this
regard.

\subsection{Dependencies of Fluctuation Strength}

The unit used to quantify the sensation of fluctuation strength is the vacil. As
a reference, 1 vacil is defined as a 1~kHz tone sound having a \gls{SPL} of
60~dB, with an \gls{AM} envelope of 4~Hz and a modulation index of 1. The
maximum value of fluctuation seems to occur around 4~Hz, regardless of the
modulation technique used (\Cref{fig:flucstrenvmodfreq}).

\begin{figure}[!ht]
  \centering
  \includegraphics[width=\textwidth]{FluctuationStrengthVsModulationFrequency}
  \caption{Fluctuation strength as a function of modulation frequency for (a)
    amplitude-modulated broad-band noises, (b) amplitude-modulated tones and (c)
    frequency-modulated tones~\cite[pp.~248]{Fastl2007Psychoacoustics}}
\label{fig:flucstrenvmodfreq}
\end{figure}

It seems that a relation between fluctuation strength and speech production
exists, as the normal production rate of syllables during normal conversation
speed is about 4 syllables/second. This coincides with the frequency in which a
maximum value of fluctuation strength occurs (4~Hz).

\Cref{fig:flucstrenvsndpreslvl} shows the relation between \gls{SPL} and
fluctuation strength for two different stimuli, tones and \gls{BBN}, and two
modulation techniques, amplitude modulation and frequency modulation. An
increase in \gls{SPL} entails an increase of fluctuation strength, this being
stronger when for the \gls{AM} tones.

\begin{figure}[!ht]
  \centering
  \includegraphics[width=\textwidth]{FluctuationStrengthVsSoundPressureLevel}
  \caption{Fluctuation strength as a function of sound pressure level for (a)
    amplitude-modulated broad-band noises, (b) amplitude-modulated tones and (c)
    frequency-modulated tones; modulation frequency of
    4Hz~\cite[pp.~249]{Fastl2007Psychoacoustics}}
\label{fig:flucstrenvsndpreslvl}
\end{figure}

Next the effect of modulation depth on fluctuation strength is analyzed, shown
in \Cref{fig:flucstrenvsmoddep}. It can be observed that, between 3~dB and 30~dB
the relation between fluctuation strength and modulation depth is somewhat
linear. After reaching a maximum value at around 30~dB, which corresponds to a
modulation factor of 94\%, fluctuation strength remains constant with further
increments of modulation depth.

\begin{figure}[!ht]
  \centering
  \includegraphics[width=\textwidth]{FluctuationStrengthVsModulationDepth}
  \caption{Fluctuation strength as a function of modulation depth for (a)
    amplitude-modulated broad-band noises of 60~dB SPL and (b)
    amplitude-modulated tones of 70~dB SPL and 1~kHz frequency; both with a
    modulation frequency of 4~Hz~\cite[pp.~249]{Fastl2007Psychoacoustics}}
\label{fig:flucstrenvsmoddep}
\end{figure}

The relation between the center frequency and fluctuation strength is shown in
\Cref{fig:flucstrenvscfreq}. Here a clear difference exists between the type of
modulation used. For \gls{AM} tones there is a small variability in the
fluctuation strength, implying this a very little dependence of the center
frequency on fluctuation strength. For \gls{FM} tones a clear dependence between
both variables exists. For center frequencies below 1~kHz the fluctuation
strength is almost constant; above 1~kHz it experiences a linear decrease until
it fades away.

\begin{figure}[!ht]
  \centering
  \includegraphics[width=\textwidth]{FluctuationStrengthVsCenterFrequency}
  \caption{Fluctuation strength as a function of center frequency for an
    amplitude-modulated tone of 70~dB SPL, 4~Hz modulation frequency and 40~dB
    modulation depth (a), and a frequency-modulated tone with 70~dB SPL, 4~Hz
    modulation frequency and $\pm200$ Hz frequency
    deviation~\cite[pp. 250]{Fastl2007Psychoacoustics}}
\label{fig:flucstrenvscfreq}
\end{figure}

To understand why this change of fluctuation strength occurs in the case of the
\gls{FM} tones, it is necessary to take into account the excitation patterns
that the modulated sounds cause. The auditory filters are a series of
overlapping bandpass filters that model the frequency selective response of the
auditory system. The excitation pattern refers to the pattern that results as
the outcome of all the precedent stages of the hearing process, being them the
outer ear transmission, middle ear transmission, and the auditory filters
themselves.

In the case of fluctuation strength, for a 0.5~kHz tone the frequency would vary
between 300 and 700~Hz, this corresponding to a 3.5~Bark interval. For a 8~kHz
these values would be 7.8~kHz and 8.2~kHz, leading to a 0.2~Bark interval. The
proportion between these two is 17.5, which seems to be also the proportion
between the relative fluctuation strength for these two frequencies. Thus, this
leads to the idea that fluctuation strength can be explained in terms of the
excitation patterns that present themselves across the auditory filters.

\Cref{fig:flucstrenvsfreqdev} shows the relation between fluctuation strength
and frequency deviation. It can be seen that, for frequencies above 20~Hz, there
is a linear increase of fluctuation strength with frequency.

\begin{figure}[!ht]
  \centering
  \includegraphics[height=8cm]{FluctuationStrengthVsFrequencyDeviation}
    \caption{Fluctuation strength as a function of frequency deviation for a
      tone with 70~dB SPL, 1.5~kHz center frequency and a modulation frequency
      of 4~Hz~\cite[pp. 251]{Fastl2007Psychoacoustics}}
\label{fig:flucstrenvsfreqdev}
\end{figure}

Another variable that affects fluctuation strength is the modulation index $h$
(also called modulation factor $m$). It is defined differently according to the
type of modulation used:
\begin{itemize}
  \item For \gls{AM} signals is the ratio between the modulating signal
    amplitude $M$ and the carrier signal amplitude $A$,
      \begin{equation}
        h=\frac{M}{A}
      \end{equation}
  \item For \gls{FM} signals is the ratio between the maximum frequency
    deviation in the carrier signal ($\Delta f$) and the maximum frequency
    component of the modulating signal ($f_m$),
      \begin{equation}
        h=\frac{\Delta f}{f_m}
      \end{equation}
\end{itemize}

It seems that a significant fluctuation strength (10\% of the relative
fluctuation strength~\cite[pp.~251]{Fastl2007Psychoacoustics}) is achieved with
a modulation index that would correspond to about 10 times the \gls{JND} for
modulation frequency. This would relate both the thresholds of modulation
frequency and fluctuation strength.

\Cref{fig:flucstrensnds} compares the fluctuation strength of several sounds,
which physical characteristics are described in \Cref{tab:flucstrensnds}. The
sounds that present the largest values of fluctuation strength (sounds 1 and 2)
excite a large range of frequencies and therefore more than one auditory filter.
As so, it can be said that fluctuation strength sums across critical bands.

\begin{figure}[!ht]
  \centering
  \includegraphics[height=8cm]{FluctuationStrengthSounds}
  \caption{Fluctuation strength of sounds 1--5 as described in
    \Cref{tab:flucstrensnds}~\cite[pp. 252]{Fastl2007Psychoacoustics}}
\label{fig:flucstrensnds}
\end{figure}

\begin{table}[!ht]
  \centering
  \begin{tabu} to \linewidth{ lXXXXX }
    \toprule
    Sound & 1 & 2 & 3 & 4 & 5 \\
    \midrule
    Abbreviation & FM & AM & AM & FM & \\
    & SIN & BBN & SIN & SIN & NBN \\
    Frequency [Hz] & 1500 & --- & 2000 & 1500 & 1000 \\
    Level [dB] & 70 & 60 & 70 & 70 & 70 \\
    Modulation frequency [Hz] & 4 & 4 & 4 & 4 & --- \\
    Modulation depth [dB] & --- & 40 & 40 & --- & --- \\
    Frequency deviation [Hz] & 700 & --- & --- & 32 & --- \\
    Bandwidth [Hz] & --- & 16000 & --- & --- & 10 \\
    \bottomrule
  \end{tabu}
  \caption{Physical data of sounds
    1--5~\cite[pp. 253]{Fastl2007Psychoacoustics}}
\label{tab:flucstrensnds}
\end{table}

\subsection{Models of Fluctuation Strength}

A basic model (proposed by \citeauthor{Fastl2007Psychoacoustics}
\cite[pp.~254]{Fastl2007Psychoacoustics}) on the temporal variation of a masking
pattern in shown on \Cref{fig:flucstrenmodel}, where the temporal variation of
the amplitude of the masker, also called temporal masking depth, is denoted by
the magnitude $\Delta L$. The inverse of the time difference between peak
corresponds to the modulation frequency $f_{m}$.

\begin{figure}[!ht]
    \centering
    \includegraphics[width=\textwidth]{FluctuationStrengthModel}
    \caption{Model of fluctuation strength
        \cite[pp. 254]{Fastl2007Psychoacoustics}}
    \label{fig:flucstrenmodel}
\end{figure}

\Cref{eq:flucstrentempmaskmodfreq} shows the relationship between fluctuation
strength $F$, temporal masking depth $\Delta L$ and modulation frequency
$f_{m}$, where the importance of the 4 Hz frequency is emphasized.

\begin{equation}
    F \sim \frac{\Delta L}{(f_{m}/4\text{ Hz}) + (4\text{ Hz}/f_{m})}
    \label{eq:flucstrentempmaskmodfreq}
\end{equation}

It is to note that there is dependency between the temporal masking depth
$\Delta L$ and the modulation frequency depending on the type of stimuli. In the
case of broad-band noise, temporal masking depth seems largely unaffected by
modulation frequency, whereas amplitude and frequency modulates tones these two
variable are dependent on each other, this being strongest on the frequency
modulated case. In order to address this, when modeling fluctuation strength
for these tones not a single $\Delta L$ value is taken, but instead it is
integrated across the critical-band rate scale.

\Cref{fig:flucstrenmasking} shows the resulting temporal masking pattern for
several values of modulation frequency. It can be seen that, as the modulation
frequency increases, the temporal masking depth decreases. This leads to the
idea that, although fluctuation strength presents a bandpass response with
respect to modulation frequency, the temporal masking suffers from a low pass
effect. It can be considered that the temporal masking depth decreases linearly
with modulation frequency.

\begin{figure}[!ht]
    \centering
    \includegraphics[width=\textwidth]{FluctuationStrengthTemporalMasking}
    \caption{Temporal masking pattern for an amplitude-modulated broad-band
        noise \cite[pp. 255]{Fastl2007Psychoacoustics}}
    \label{fig:flucstrenmasking}
\end{figure}

Taking all these into account, \Cref{eq:flucstrenexbbn} presents an
updated model, in which $m$ is the modulation factor, and $L_{BBN}$ is the level
of broad-band noise. Similarly, \Cref{eq:flucstrenexamfm} presents the
equivalent for the tone signals, where the temporal masking depth is integrated
across the auditory filters.

\begin{equation}
    F_{BBN} = \frac{5.8(1.25m-0.25)[0.05(L_{BBN}/\text{dB})-1]}
        {(f_{m}/5\text{ Hz})^2+(4\text{ Hz}/f_{m})+1.5} \text{ vacil}
    \label{eq:flucstrenexbbn}
\end{equation}

\begin{equation}
    F = \frac{0.008 \int_0^{24\text{ Bark}}(\Delta L/\text{dB Bark})\mathrm{d}z}
        {(f_{m}/4\text{ Hz})+(4\text{ Hz}/f_{m})} \text{ vacil}
    \label{eq:flucstrenexamfm}
\end{equation}

The modeling based on the temporal masking property, proposed by
\citeauthor{Fastl2007Psychoacoustics}, poses difficulties when its
implementation details are addressed. Furthermore, it does not exist up to this
date a publicly available implementation for any given model of fluctuation
strength. There are, however, implementation models when it comes to roughness,
a sensation that it was already mentioned as being similar to fluctuation
strength regarding its physical characteristics. The most prominent of these
models is the one developed by \citeauthor{daniel1997psychoacoustical}
\cite{daniel1997psychoacoustical}, which is based on the dependency of roughness
on modulation depth (\Cref{eq:R})). Their model proposed a generalized
modulation depth function, intended to model the modulation depth dependence
per critical filterbank and obtain specific roughness values. Then a total
overall roughness value is obtained.
\begin{equation}
   R \sim m^p
   \label{eq:R}
\end{equation}

\subsection{Further Studies}

Although the work by \citeauthor{Fastl2007Psychoacoustics} is the most extensive
reference for the fluctuation strength sensation, other studies have been
carried out to investigate further the phenomena.
\citeauthor{Accolti2009Fluctuation} applied a model based on the temporal
masking envelope proposed by \citeauthor{Fastl2007Psychoacoustics} to sounds
composed of two mixed \gls{AM} sources. They found that the model could not be
extended to these type of stimuli. However, the most remarkable part of their
study is the inclusion of a training phase before they performed the actual
experiment. This is supported by the fact that individuals are not familiarized
with the concept of fluctuation strength, leading this to erroneous answers.

Building upon this last point, \citeauthor{Wickelmaier2004Scaling} carried out
a three-part experiment that came out with similar results in that sense. First,
a full-factorial design with 54 pairs of stimuli, with nine modulation
frequencies and six modulation depths, and a magnitude estimation test is run.
The results differ from those reported by \citeauthor{Fastl2007Psychoacoustics}
fluctuation strength model, since they do not show the characteristic band-pass
response expected. The second experiment tries to assess the contribution of the
two factors used in the past experiment (modulation frequency and modulation
depth) separately, by varying one while leaving the other constant. It was found
that the variation of the modulation depth conforms to
\citeauthor{Fastl2007Psychoacoustics} model, while the variation of the
modulation frequency does not. In the last experiment they try to assess whether
fluctuation can be represented by an additive combination of modulation
frequency and modulation depth. To test this they use the Thomsen condition,
which results suggest that listeners do not integrate these two variables as an
unidimensional percept. \citeauthor{Wickelmaier2004Scaling} conclusions are that
\citeauthor{Fastl2007Psychoacoustics} data does not conform properly to their
data, and that the status of fluctuation strength as a basic auditory perceptual
attribute could be debated. The discrepancy between the two data sets could also
be attributed to the lack of understanding of participants, as
\citeauthor{Accolti2009Fluctuation} mentioned.

\end{theoreticalbackground}

\end{document}
