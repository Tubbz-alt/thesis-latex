\documentclass[../main.tex]{subfiles}

\begin{document}

\chapter{Pilot Experiment Design and Results}
\label{cha:pilot}

This chapter describes the experimental design for the evaluation of the
fluctuation strength attribute. First, the initial design (as
in~\cite{Fastl1982Fluctuation}) is presented. Then, the iterative process of the
pilots execution is detailed, stating the progressive changes made to the
initial design. Finally, a series of conclusions that lead to the final
experimental design are presented.

\section{Initial Design}

\subsection{Subjects}

In total 9 subjects participated in the the pilot experiments. Participants were
between 20 and 30 years old. All of them had self-reported normal hearing.

\subsection{Stimuli}
\label{subsec:pilot_stimuli}

\Cref{tab:initial_stimuli} presents all the stimuli used for the different
experimental sections, stating which parameters were fixed and which were
varied.

\begin{table}[!ht]
  \centering
  \begin{tabu} to \linewidth{p{2cm}Xp{7cm}}
  \toprule
  \rowfont\bfseries
  \multirow{2}{*}{Section} &
  \multicolumn{2}{c}{Parameters} \\
  \cmidrule{2-3}
  \rowfont\bfseries
  & \multicolumn{1}{c}{Fixed} & \multicolumn{1}{c}{Varied} \\
  \midrule
  AM-fm  & $f_c$ = 1 [kHz]\par SPL = 70 [dB]\par $m_d$ = 40 [dB]
         & $f_m$ = \{0, 0.25, 0.5, 1, 2, 4, 8, 16, 32\} [Hz] \\
  \midrule
  AM-fc  & $f_m$ = 4 [Hz]\par SPL = 70 [dB]\par $m_d$ = 40 [dB]
         & $f_c$ = \{0.125, 0.25, 0.5, 1, 2, 4, 8\} [kHz] \\
  \midrule
  AM-SPL & $f_m$ = 4 [Hz]\par $f_c$ = 1 [kHz]\par $m_d$ = 40 [dB]
         & SPL = \{50, 60, 70, 80, 90\} [dB] \\
  \midrule
  AM-md  & $f_m$ = 4 [Hz]\par $f_c$ = 1 [kHz]\par SPL = 70 [dB]
         & $m_d$ = \{1, 2, 4, 10, 20, 40\} [dB] \\
  \midrule
  FM-fm  & $f_c$ = 1.5 [kHz]\par SPL = 70 [dB]\par $d_f$ = 700 [Hz]
         & $f_m$ = \{0, 0.25, 0.5, 1, 2, 4, 8, 16, 32\} [Hz] \\
  \midrule
  FM-fc  & $f_m$ = 4 [Hz]\par SPL = 70 [dB]\par $d_f$ = 200 [Hz]
         & $f_c$ = \{0.5, 1, 1.5, 2, 3, 4, 6, 8\} [kHz] \\
  \midrule
  FM-SPL & $f_m$ = 4 [Hz]\par $f_c$ = 1.5 [kHz]\par $d_f$ = 700 [Hz]
         & SPL = \{40, 50, 60, 70, 80\} [Hz] \\
  \midrule
  FM-df  & $f_m$ = 4 [Hz]\par $f_c$ = 1.5 [kHz]\par SPL = 70 [dB]
         & $d_f$ = \{16, 32, 100, 300, 700\} [Hz] \\
  \bottomrule
  \end{tabu}
  \caption{Description of initial set of stimuli used per experiment section}
\label{tab:initial_stimuli}
\end{table}

\subsection{Procedure}

The pilot experiment was divided in two phases, the training phase and
experimental phase, as stated earlier in \Cref{cha:methods}. The experimental
phase remained the same as explained before, using the stimuli set defined in
\Cref{tab:initial_stimuli}. The training phase for the pilot experiment is
described as follows.

\subsubsection{Training Phase}
\label{subsub:training_phase}

The objective of this phase was to make the concept of fluctuation strength
clear to the participants and to familiarize them with the range of stimuli.
Past studies~\cite{Accolti2009Fluctuation} have pointed out the need of such a
phase to familiarize subjects with the sensation. However, this must be
approached with caution, as the intention of the training phase is to show
participants what the sensation is, not teach them how to answer to specific
questions regarding the stimuli.

\paragraph{Stimulus Comparison}

First, a subset of \gls{AM} tones (\Cref{tab:initial_am_training_stimuli}) was
presented to the participants sequentially according to their ID in pairs (i.e.,
first stimuli 1 and 2, then 2 and 3, etc.). After each pair presentation,
participants were asked whether a difference in the fluctuation strength among
stimuli was detected. In the case of a negative answer the pair was repeated
until a positive answer was obtained.

\begin{table}[!ht]
  \centering
  \begin{tabu} to \linewidth{XXXXX}
    \toprule
    \rowfont\bfseries
    ID & $\bm{f_m}$ [Hz] & $\bm{f_c}$ [kHz] & SPL [dB] & $\bm{m_d}$ [dB] \\
    \midrule
    1 & 0   & 1 & 70 & 40 \\
    2 & 0.5 & 1 & 70 & 40 \\
    3 & 4   & 1 & 70 & 40 \\
    4 & 32  & 1 & 70 & 40 \\
    \bottomrule
  \end{tabu}
  \caption{Initial subset of AM stimuli for training phase}
\label{tab:initial_am_training_stimuli}
\end{table}

\paragraph{Long Interval}

Afterwards, a long interval was presented to the participants. The long interval
consisted of the sequential reproduction of several stimuli
(\Cref{tab:initial_am_training_stimuli}) separated by 800~ms of silence. This
was done with the objective of exposing the subjects to a wider range of stimuli
and thus familiarize them more to these kind of sounds.

\begin{table}[!ht]
  \centering
  \begin{tabu} to \linewidth{XXXXX}
    \toprule
    \rowfont\bfseries
    Presentation & \multirow{2}{*}{$\bm{f_m}$ [Hz]} & \multirow{2}{*}{$\bm{f_c}$ [kHz]} & \multirow{2}{*}{SPL [dB]} & \multirow{2}{*}{$\bm{m_d}$ [dB]} \\
    \rowfont\bfseries
    order \\
    \midrule
    1 & 8    & 1 & 70 & 40 \\
    2 & 0.5  & 1 & 70 & 40 \\
    3 & 0    & 1 & 70 & 40 \\
    4 & 2    & 1 & 70 & 40 \\
    5 & 32   & 1 & 70 & 40 \\
    6 & 4    & 1 & 70 & 40 \\
    7 & 16   & 1 & 70 & 40 \\
    8 & 1    & 1 & 70 & 40 \\
    9 & 0.25 & 1 & 70 & 40 \\
    \bottomrule
  \end{tabu}
  \caption{Initial long interval composed of AM stimuli for training phase}
\label{tab:initial_am_all_stimulus}
\end{table}

\section{Iterative Improvements}

Not all participants were subjected to the same experimental conditions, and
not all of them used the same version of the experiments.
\Cref{tab:partexpconver} presents the conditions and in which one of them the
subjects participated.

\begin{table}[!ht]
  \centering
  \begin{tabu} to \linewidth{XXXX}
    \toprule
    \rowfont\bfseries
    Participant & AM & FM & Version \\
    \midrule
    1 & All & All & 1 \\
    2 & All & None & 1 \\
    3 & All & None & 1 \\
    4 & All & None & 1 \\
    5 & \gls{f_m} & None & 2 \\
    6 & \gls{f_m} & None & 3 \\
    7 & None & \gls{f_m}, \gls{f_c} & 3 \\
    8 & None & All & 3 \\
    9 & None & All & 4 \\
    \bottomrule
  \end{tabu}
  \caption{Participants experimental conditions and versions}
\label{tab:partexpconver}
\end{table}

The experimental procedure was varied during the pilot experiment to accommodate
perceived errors during the realization of them. The first version of the
experiment yielded unsatisfactory results with regard to the relation between
fluctuation strength and modulation frequency (participants 2, 3 and 4). The
procedure was then modified, adding two more \gls{AM} tones with modulation
frequencies of 64 and 128~Hz. The idea behind the addition of these two tones
was that, if participant have stimuli that give a distinguishing roughness
sensation, it would be easier for them to distinguish between a fluctuating and
a rough tone. Additionally, \gls{FM} tones were included in the training, since
up to this point only \gls{AM} tones were used in the training phase. This
constitutes the second version of the experiment.

Participant 5 was the only participant that was subjected to version 2 of the
experiment. The results did not show any significant improvements with regard
to the confusion between fluctuation strength and roughness. However, by talking
to the participants it was discovered that the training phase was not able to
make the concept of fluctuation strength clear. Participants were only asked to
tell whether there was a difference of fluctuation among the presented stimuli,
without explaning what fluctuation strength actually was. As such, several
participants associated the rate of change of the stimuli (modulation frequency
in this case) with a bigger fluctuation in the presented sounds. Hence, they
tended to deem as highly fluctuating the sounds that had a high modulation
frequency. Participants 4 and 5 explicitly stated that they were counting the
number of cycles in the stimuli, due to confusion on what to answer.

Taking all these comments as feedback, version 3 of the experiment was
elaborated. In this version, explicit instructions regarding the rough tones
were given. It was indicated that the sensation of fluctuation was unrelated to
the apparent `speed' of the stimuli, and that the answers should be intuitive,
based on the arising sensation and not rationalizing any judgment about it
(for instance by counting cycles). Using this approach participants were able to
understand better the fluctuation strength concept, some of them even coming
with analogies to the sensation itself (the sound of an ambulance alarm, the
sound of a washing machine). The actual instructions used in the final
experiment can be found in the prompts of the experimental protocol
(\Cref{cha:experimental_protocol}).

The final version, number 4, of the experiment added a small test experiment
before starting the actual experimental sections. This was added as a suggestion
from participant 8, who indicated that although the training phase was effective
in making the fluctuation strength concept clear, it did not show the
participant how to do the expected judgments using the magnitude estimation
procedure. Moreover, a latin square randomization approach was used, rotating
the order of the experimental sections for each participant. The purpose of this
was to distribute possible learning effects of participants among the
experimental conditions. Finally, the modulation frequency sections (AM-fm and
FM-fm) were split into two separate sections, each one with 2 repetitions per
pairs instead of 4. Therefore, two smaller sections for the AM-fm and FM-fm were
used in the final experiment. This was done due to the longer duration of the
modulation frequency sections when compared to the other sections. By keeping
all the sections relatively short (around 6 minutes each) it was expected that
participants attention and focus would be retained from one section to another.

\section{Results}

The following figures show the results of the pilot experiments, together with
the results reported by \textcite{Fastl2007Psychoacoustics}.  Comparing the two
plots per experimental condition, it can be concluded that both graphs are
similar from a qualitative point of view. Therefore, the pilot experiment was
deemed as successful in obtaining the relevant subjective data from
participants. A more detailed description of the particularities of each
experimental condition curve will be given in \Cref{cha:experiment}.

\begin{pilotresults}

\myfigurefastlexpstds%
  {AM-fm}
  {Relative fluctuation strength as a function of modulation frequency for
    \gls{AM} tones with center frequency of 1~kHz, sound pressure level of
    70~dB and modulation depth of 40~dB.  The two standards had modulation
    frequencies of 4 and 0.5~Hz. The data points show the median and
    interquartile ranges per standard. The black line represents the mean values
    of the medians of each standard. Panel~(a): data adapted from
    \cite[pp.248]{Fastl2007Psychoacoustics}. Panel~(b): own results}
  {pilot}

\myfigurefastlexpstds%
  {AM-fc}
  {Relative fluctuation strength as a function of center frequency for
    \gls{AM} tones with modulation frequency of 4~Hz, sound pressure level of
    70~dB and modulation depth of 40~dB.  The two standards had center
    frequencies of 1 and 0.25~kHz. The data points show the median and
    interquartile ranges per standard. The black line represents the mean values
    of the medians of each standard. Panel~(a): data adapted from
    \cite[pp.250]{Fastl2007Psychoacoustics}. Panel~(b): own results}
  {pilot}

\myfigurefastlexpstds%
  {AM-SPL}
  {Relative fluctuation strength as a function of sound pressure level for
    \gls{AM} tones with modulation frequency of 4~Hz, center frequency of 1~kHz
    and modulation depth of 40~dB.  The two standards had sound pressure levels
    of 70 and 50~dB. The data points show the median and
    interquartile ranges per standard. The black line represents the mean values
    of the medians of each standard. Panel~(a): data adapted from
    \cite[pp.249]{Fastl2007Psychoacoustics}. Panel~(b): own results}
  {pilot}

\myfigurefastlexpstds%
  {AM-md}
  {Relative fluctuation strength as a function of modulation depth for
    \gls{AM} tones with modulation frequency of 4~Hz, center frequency of 1~kHz
    and sound pressure level of 70~dB.  The two standards had modulation depths
    of 40 and 4~dB. The data points show the median and
    interquartile ranges per standard. The black line represents the mean values
    of the medians of each standard. Panel~(a): data adapted from
    \cite[pp.249]{Fastl2007Psychoacoustics}. Panel~(b): own results}
  {pilot}

\myfigurefastlexpstds%
  {FM-fm}
  {Relative fluctuation strength as a function of modulation frequency for
    \gls{FM} tones with center frequency of 1.5~kHz, sound pressure level of
    70~dB and frequency deviation of 700 Hz.  The two standards had modulation
    frequencies of 4 and 0.5~Hz. The data points show the median and
    interquartile ranges per standard. The black line represents the mean values
    of the medians of each standard. Panel~(a): data adapted from
    \cite[pp.248]{Fastl2007Psychoacoustics}. Panel~(b): own results}
  {pilot}

\myfigurefastlexpstds%
  {FM-fc}
  {Relative fluctuation strength as a function of center frequency for
    \gls{FM} tones with modulation frequency of 4~Hz, sound pressure level of
    70~dB and frequency deviation of 200~Hz.  The two standards had center
    frequencies of 6 and 0.5~kHz. The data points show the median and
    interquartile ranges per standard. The black line represents the mean values
    of the medians of each standard. Panel~(a): data adapted from
    \cite[pp.250]{Fastl2007Psychoacoustics}. Panel~(b): own results}
  {pilot}

\myfigurefastlexpstds%
  {FM-SPL}
  {Relative fluctuation strength as a function of sound pressure level for
    \gls{FM} tones with modulation frequency of 4~Hz, center frequency of
    1.5~kHz and frequency deviation of 700~Hz.  The two standards had sound
    pressure levels of 60 and 40~dB. The data points show the median and
    interquartile ranges per standard. The black line represents the mean values
    of the medians of each standard. Panel~(a): data adapted from
    \cite[pp.249]{Fastl2007Psychoacoustics}. Panel~(b): own results}
  {pilot}

\myfigurefastlexpstds%
  {FM-df}
  {Relative fluctuation strength as a function of modulation depth for
    \gls{FM} tones with modulation frequency of 4~Hz, center frequency of
    1.5~kHz and sound pressure level of 70~dB.  The two standards had frequency
    deviations of 700 and 32~Hz. The data points show the median and
    interquartile ranges per standard. The black line represents the mean values
    of the medians of each standard. Panel~(a): data adapted from
    \cite[pp.251]{Fastl2007Psychoacoustics}. Panel~(b): own results}
  {pilot}

\end{pilotresults}

\section{Conclusions}

From the obtained data it can be concluded that the main problem when dealing
with the perceptual attribute of fluctuation strength is its ambiguity and
confusion with the perceptual attribute of roughness. The proposed training
phase was effective in clarifying the concept to participants, by adding stimuli
with a clear rough sensation, and by clearly instructing them on what the
sensation is about. It should be noted that only with regard to modulation
frequency this confusion arises, the other parameters do not present this
particularity and as so it was not necessary to adapt the experimental procedure
with them.

\end{document}
