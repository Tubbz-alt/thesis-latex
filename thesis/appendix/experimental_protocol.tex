\documentclass[../main.tex]{subfiles}

\begin{document}

\chapter{Experimental Protocol}
\label{cha:experimental_protocol}

\section{Procedure}

Three subprocedures are defined, corresponding to the period before, during, and
after the subjects participation.

\subsection{Before}

\begin{enumerate}
  \item Verify that the sound interface is calibrated and working properly
  \item Execute the corresponding experiment batch script\footnotemark[1]
  \item Input participant's assigned ID and condition\footnotemark[2]
  \item Turn off computer's monitor and wait for the participant to arrive
\end{enumerate}

\footnotetext[1]{\emph{run\_am\_experiments.bat} for AM tones,
\emph{run\_fm\_experiments.bat} for FM tones}
\footnotetext[2]{The condition refers to the order in which the experimental
sections are presented, using a latin square design. It is assigned using the
following equation: $\text{condition}=(\text{ID}-1)\%4+1$}

\subsection{During}

\begin{enumerate}
  \item Present the participant with the informed consent form and ask him to
  sign it
  \item Read the training phase prompt (\ref{sub:training_phase})
  \item Present twice the pairs of non fluctuating and low fluctuating sounds,
  indicating before and after their reproduction the level of fluctuation they
  present
  \item Ask whether a difference in sensation was felt between the sounds, if
  not repeat previous point
  \item Present twice the pairs of non fluctuating and low fluctuating sounds,
  indicating before and after their reproduction the level of fluctuation they
  present
  \item Ask whether a difference in sensation was felt between the sounds, if
  not repeat previous point
  \item Read the rough tones prompt (\ref{sub:before_rough_tones})
  \item Play the remaining triplets of sounds in descending order according to
  their fluctuation strength
  \item Ask for participant's perception of fluctuation strength
  \item Read the rough tones prompt (\ref{sub:after_rough_tones})
  \item Play the again the remaining triplets of sounds in descending order
  according to their fluctuation strength
  \item Ask to the participant is the difference between the high fluctuating
  sound and the faster but less fluctuating rough tones is clear and agreed
  upon
  \item Read the long stimuli prompt (\ref{sub:long_stimuli}).
  \item Play each of the long stimulus sounds, asking after each reproduction
  whether a difference between the sounds regarding their fluctuation strength
  was felt
  \item Close the current experiment (pressing the ALT+F4 key combination)
  \item Read the test trials prompt (\ref{sub:test_trials})
  \item Observe the participant during the test trials completion. If he makes
  any mistake correct him
  \item Read the before sections prompt (\ref{sub:before_sections})
  \item Leave the experiment room and wait for the participant to finish
\end{enumerate}


\subsection{After}

\begin{enumerate}
  \item Copy the results files to a safer location, possibly to an USB Drive
  \item Shutdown the experiment computer
\end{enumerate}

\section{Prompts}

\subsection{Training Phase}

As you read in the informed consent form, this experiment is about an auditory
sensation called fluctuation strength. Fluctuation strength is related to a
fluctuating or circulating feeling that arises from certain sounds. For
instance, ambulance sirens and washing machines have this quality, as their
sounds have a certain movement, rotation or circulation associated to it.

Since the concept of fluctuation strength is difficult to explain with words,
the first part of the experiment consists of a training phase, aimed at making
clear what is fluctuation strength to you. During this phase I will be
presenting you with different sounds with different levels of fluctuation
strength. After the training phase is over, and the concept of fluctuation
strength is clear, we will start with the experiment itself.

Now, we will start with the training phase.

\label{sub:training_phase}

\subsection{Before Rough Tones}
\label{sub:before_rough_tones}

The following three sounds are a little bit difference. First I will play them
for you, and then I will ask you to tell me how do you perceive them in terms
of fluctuation strength.

\subsection{After Rough Tones}
\label{sub:after_rough_tones}

As you may have perceived, as we move from one sound to another between these
three sounds, their rate of change or speed increases. You may have also noted
that also the sound begin to resemble noise. However, the increase of rate of
change does not mean that the sound fluctuates more, as the sensation between
the first sound and the other two is difference. The first sound is considered
to have a high fluctuation, whereas the other two sound are considered to have
a lower fluctuation.

It is important that you focus on your sensation, on what you feel when you
hear the sound. Since the fluctuating is not related to the speed of the sound,
I do not want you to count cycles in order to estimate a fluctuation strength
sensation. Remember that the fluctuating sensation is sometimes that moves, that
circulates.

Now I will reproduce the sounds again and I will ask you again what do you think
about their fluctuation strength.

\subsection{Long Stimuli}
\label{sub:long_stimuli}

Now I'm going to present you with a couple of long duration sounds. Each sound
is composed of shorter sounds, with silence pauses in between each of them. You
will listen to approximately 10 sounds. What I want you to do is to pay
attention to the sounds, and try to detect whether a difference of fluctuation
between the sounds exist. I'm not interested in a specific pair of sounds, but
in the set of sounds as a whole.

\subsection{Test Trials}
\label{sub:test_trials}

This is a test experiment, to familiarize yourself with the interface that you
will use for the rest of the experiment. When you click on the Start button
on the top right corner of the screen, the experiment will start. You will
hear a pair of sounds, with a silence pause in between. So, it will be one
sound, then a silence, then another sound. What I want you to do is to estimate
how much does the second sound fluctuates with respect to the first sound.

In order to give an estimate, you will use the slider that is situated in the
bottom part of the screen. For instance, if the second sound fluctuates more
than the first sound, you will drag the slider to the right part of the area
that it contains. In case both stimuli fluctuate around the same, you will
position the slider close to the `Equal' label. In case the second sound
fluctuates less than the first, you will position the slider in the left part
of the area that in contains. Lastly, if the second sound presents no
fluctuation at all, you will position the slider below the `None' label. When
you are done with your answer, you can click the `Next' button below the slider,
and another pair of sound will be presented to you.

If by any chance you need to listen to the pair of sounds again, there is the
`Repeat stimulus' button to the bottom right part of the screen that you can
use. Below this button there is also a progress bar, so you can see how much of
the experiment is left for you to do.

Whenever you are ready you can click the `Start' button and begin with these
test trials.

\subsection{Before Sections}
\label{sub:before_sections}

We have just finished the training phase. The remaining sections constitute the
actual experiment that we will be doing. These sections are similar to the test
trials phase that you just did. Each section takes approximately 10 minutes.
When you click the `Start' button the section will begin. Once you are done with
it, the screen will close and another one that looks exactly the same will
present to you. At this point you can choose to have a small break, go outside
the isolation booth and stretch a bit, or you can continue if you wish.

I will leave now and close the doors behind me, so you are in a complete audio
isolated environment. The doors are not locked, so you may exist at any time
you want. Just bear in mind that the doors are quite heavy and you need to pull
them strongly.

\end{document}
