\documentclass[../main.tex]{subfiles}

\begin{document}

\chapter{Experiment Individual Results}

\begin{experimentalresults}

\section{AM Tones}

\individualresults{Modulation Frequency}
  {AM-fm}
  {Relative fluctuation strength as a function of modulation frequency for AM
  tones}
  {1,3,5,7,9,11,13,15,17,21,23}

\individualresults{Center Frequency}
  {AM-fc}
  {Relative fluctuation strength as a function of center frequency for AM
  tones}
  {1,3,5,7,9,11,13,15,17,19,21,23}

\individualresults{Sound Pressure Level}
  {AM-SPL}
  {Relative fluctuation strength as a function of sound pressure level for AM
  tones}
  {1,3,5,7,9,11,13,15,17,21,23}

\individualresults{Modulation Depth}
  {AM-md}
  {Relative fluctuation strength as a function of modulation depth for AM
  tones}
  {1,3,5,7,9,11,13,15,17,19,21,23}

\section{FM Tones}

\individualresults{Modulation Frequency}
  {FM-fm}
  {Relative fluctuation strength as a function of modulation frequency for FM
  tones}
  {2,4,6,8,10,12,14,16,18,20,22,24}

\individualresults{Center Frequency}
  {FM-fc}
  {Relative fluctuation strength as a function of center frequency for FM
  tones}
  {2,4,6,8,10,12,14,16,18,20,22,24}

\individualresults{Sound Pressure Level}
  {FM-SPL}
  {Relative fluctuation strength as a function of sound pressure level for FM
  tones}
  {2,4,6,8,10,12,14,16,18,20,22,24}

\individualresults{Frequency Deviation}
  {FM-df}
  {Relative fluctuation strength as a function of frequency deviation for FM
  tones}
  {2,4,6,8,10,12,14,16,18,20,22,24}

\end{experimentalresults}

\end{document}
