\documentclass[a4paper]{report}
\usepackage{mystyle}

\begin{document}

\title{Fluctuation Strength}
\author{Rodrigo García León\\
        M.Sc Student Human-Technology Interaction\\
        Department of Industrial Engineering \& Innovation Sciences\\
        Eindhoven University of Technology\\
        \texttt{r.garcia.leon@student.tue.nl}}

\maketitle

\section{Books}

\subsection{Handbook of Engineering Acoustics}

This book contains a chapter about the effect of sound of humans. First a small
overview of the perceptual processes and the associated perceptual quantities is
given \cite[p.~72]{Fastl2007Psychoacoustics}. \ref{tab:stimsens} illustrates
all the perceptual measures, along with their dominant physical stimuli.

\begin{table}
    \centering
    \begin{tabu}{ l l }
        \tabucline[1pt]{-}
        Dominant stimuli & Cognitive parameters \\\tabucline[1pt]{-}
        Sound pressure level (dB) & Loudness (sone) \\\cline{2-2}
        & Loudness level (phon) \\\hline
        Frequency (Hz) & Critical band rate (Bark) \\\cline{2-2}
        & Ratio pitch (mel) \\\hline
        Degree of modulation (\%) & Roughness (asper)\\\cline{1-1}
        Modulation frequency (Hz) & \\\hline
        Frequency (Hz) & Sharpness (acum) \\\hline
        Degree of modulation (\%) & Fluctuation strength (vacil) \\\cline{1-1}
        Modulation frequency (Hz) & \\\hline
        Spectral components (Pa) & Pitch strength \\\cline{2-2}
        & Tonality (tu) \\\hline
        Impulse duration (s) & Subjective duration of impetus (IU) \\\hline
        Sound pressure level (dB) & Density (dasy) \\
        Frequency (Hz) & \\\tabucline[1pt]{-}
    \end{tabu}
    \caption{Stimuli and sensations}
    \label{tab:stimsens}
\end{table}

\subsection{Psychoacoustics: Facts and models}

There are two kinds of perceptual effects that arise due to the modulation of
sounds: fluctuation strength and roughness. The soft limit (i.e.\ there is a
smooth transition between the sensations) occurs at approximately 20 Hz. For
frequencies below this limit is where the fluctuation strength perception
corresponds.

The reference value for the perceptual quantity for fluctuation of strength is
defined at 1 kHz, with a sound level pressure (SPL) of 60-dB and 100\% amplitude
modulated with a 4 Hz tone. This reference is quantified as having 1 vacil.

When comparing amplitude-modulated broad-band noise (AM BBN), \\
amplitude-modulated pure tones (AM SIN) and frequency-modulated pure tones
(FM SIN), the fluctuation strength against modulating frequency plot shows a
characteristic bandpass shape, with a maximum value around 4 Hz.

It seems that a relation between fluction strength and speech production exists,
as the normal production rate of syllables during normal conversation speed is
about 4 syllables/second.

\bibliographystyle{plain}
\bibliography{Remote}

\end{document}
