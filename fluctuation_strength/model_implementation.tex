\documentclass[a4paper]{article}
\usepackage{mystyle}

\begin{document}

\customtitle{Fluctuation Strength Model Implementation}

\section{Introduction}

The following document details the implementation of a fluctuation strength
model, based on the roughness model implemented by \citeauthor{Schrader2002}.
This is similar to the work of \citeauthor{Sontacchi1998}, which will be also
taken into account when making the necessary adjustments.

First the model will be implemented for AM (amplitude-modulated) tones; later it
will be extended to cover also FM (frequency-modulated) tones and AM BBN
(broadband noise).

In order to validate the implemented model with the data available on literature
(e.g., \citeauthor{Fastl2007Psychoacoustics}), several values and plots will be
specified. These will determine the expected output from the model, and will
serve to confirm the correctness of the implementation.

\section{Implementation}

The following are implementation details of the roughness model that probably
will have to be adjusted for the fluctuation strength model.

\begin{itemize}
    \item Does the FFT resolution affects in any way the results of the model?
    \item \matlabinline{Hweight} models the bandpass response of the modulation
        frequency, do these weights have to be changed for the fluctuation
        strength case?
    \item \matlabinline{gzi} represents the effect of the carrier frequency on
        the output, do these values have to be changed for the fluctuation
        strength case?
    \item Does the relation $f_s \propto m_d ^ 2$ still holds?
\end{itemize}

\subsection{FFT Resolution}

\subsection{Modulation Frequency Dependency}

\subsection{Carrier Frequency Dependency}

\subsection{Modulation Depth Dependency}

\bibliographystyle{plainnat}
\bibliography{../References}

\end{document}
