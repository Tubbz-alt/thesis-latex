\documentclass[a4paper]{article}
\usepackage{mystyle}

\begin{document}

\customtitle{Fluctuation Strength Test Battery Experiment}

\section{Introduction}

This document describes the experimental setup proposed to obtain the data
needed regarding the tests described in the fluctuation strength test battery.
The results of pilot experiments are also reported.

\section{Results}

\subsection{AM Tones}

\subsubsection{Modulation Frequency}

Figure \ref{fig:amstds} shows the results regarding the relation between
fluctuation strength and modulation frequency by standard used. Figure
\ref{fig:amcomp} shows a comparison between the relative fluctuation strength
of the experimental data, \citeauthor{Fastl2007Psychoacoustics} data and the
proposed model.

\begin{figure}[ht!]
    \centering
    \includegraphics[height=10cm]{img/AM_tones-fm-results-All-standards}
    \caption{Relative fluctuation strength as a function of modulation
        frequency by standard used for AM tones with $f_c = 1$ kHz, $m_d = 40$
        dB, SPL $= 70$ dB (error bars indicate standard deviation)}
    \label{fig:amstds}
\end{figure}

\begin{figure}[ht!]
    \centering
    \includegraphics[height=10cm]{img/AM_tones-fm-results-All-comparison}
    \caption{Comparison between relative fluctuation strength as a function of
        modulation frequency for AM tones using the experimental data,
        \citeauthor{Fastl2007Psychoacoustics} data and the proposed model
        (error bars indicate standard deviation)}
    \label{fig:amcomp}
\end{figure}

\subsection{FM Tones}

\subsubsection{Modulation Frequency}

Figure \ref{fig:fmstds} shows the results regarding the relation between
fluctuation strength and modulation frequency by standard used. Figure
\ref{fig:fmcomp} shows a comparison between the relative fluctuation strength
of the experimental data, \citeauthor{Fastl2007Psychoacoustics} data and the
proposed model.

\begin{figure}[ht!]
    \centering
    \includegraphics[height=10cm]{img/FM_tones-fm-results-All-standards}
    \caption{Relative fluctuation strength as a function of modulation
        frequency by standard used for FM tones with $f_c = 1.5$ kHz,
        $d_f = 700$ Hz, SPL $= 70$ dB (error bars indicate standard deviation)}
    \label{fig:fmstds}
\end{figure}

\begin{figure}[ht!]
    \centering
    \includegraphics[height=10cm]{img/FM_tones-fm-results-All-comparison}
    \caption{Comparison between relative fluctuation strength as a function of
        modulation frequency for FM tones using the experimental data,
        \citeauthor{Fastl2007Psychoacoustics} data and the proposed model
        (error bars indicate standard deviation)}
    \label{fig:fmcomp}
\end{figure}

\custombibliography

\clearpage

\appendix

\section{Individual Results}

\subsection{AM Tones}

\subsubsection{Modulation Frequency}

\paragraph{Alejandro} ~\\

\begin{figure}[H]
    \centering
    \includegraphics[height=8cm]{img/AM_tones-fm-results-AO-standards}
    \caption{Relative fluctuation strength as a function of modulation
        frequency by standard used for AM tones with $f_c = 1$ kHz, $m_d = 40$
        dB, SPL $= 70$ dB (error bars indicate standard deviation)}
\end{figure}

\begin{figure}[H]
    \centering
    \includegraphics[height=8cm]{img/AM_tones-fm-results-AO-comparison}
    \caption{Comparison between relative fluctuation strength as a function of
        modulation frequency for AM tones using the experimental data,
        \citeauthor{Fastl2007Psychoacoustics} data and the proposed model
        (error bars indicate standard deviation)}
\end{figure}

\paragraph{Armin} ~\\

\begin{figure}[H]
    \centering
    \includegraphics[height=8cm]{img/AM_tones-fm-results-Armin-standards}
    \caption{Relative fluctuation strength as a function of modulation
        frequency by standard used for AM tones with $f_c = 1$ kHz, $m_d = 40$
        dB, SPL $= 70$ dB (error bars indicate standard deviation)}
\end{figure}

\begin{figure}[H]
    \centering
    \includegraphics[height=8cm]{img/AM_tones-fm-results-Armin-comparison}
    \caption{Comparison between relative fluctuation strength as a function of
        modulation frequency for AM tones using the experimental data,
        \citeauthor{Fastl2007Psychoacoustics} data and the proposed model
        (error bars indicate standard deviation)}
\end{figure}

\paragraph{Ryan} ~\\

\begin{figure}[H]
    \centering
    \includegraphics[height=8cm]{img/AM_tones-fm-results-Ryan-standards}
    \caption{Relative fluctuation strength as a function of modulation
        frequency by standard used for AM tones with $f_c = 1$ kHz, $m_d = 40$
        dB, SPL $= 70$ dB (error bars indicate standard deviation)}
\end{figure}

\begin{figure}[H]
    \centering
    \includegraphics[height=8cm]{img/AM_tones-fm-results-Ryan-comparison}
    \caption{Comparison between relative fluctuation strength as a function of
        modulation frequency for AM tones using the experimental data,
        \citeauthor{Fastl2007Psychoacoustics} data and the proposed model
        (error bars indicate standard deviation)}
\end{figure}

\subsection{FM Tones}

\subsubsection{Modulation Frequency}

\paragraph{Alejandro} ~\\

\begin{figure}[H]
    \centering
    \includegraphics[height=8cm]{img/FM_tones-fm-results-AO-standards}
    \caption{Relative fluctuation strength as a function of modulation
        frequency by standard used for FM tones with $f_c = 1.5$ kHz,
        $d_f = 700$ Hz, SPL $= 70$ dB (error bars indicate standard deviation)}
\end{figure}

\begin{figure}[H]
    \centering
    \includegraphics[height=8cm]{img/FM_tones-fm-results-AO-comparison}
    \caption{Comparison between relative fluctuation strength as a function of
        modulation frequency for FM tones using the experimental data,
        \citeauthor{Fastl2007Psychoacoustics} data and the proposed model
        (error bars indicate standard deviation)}
\end{figure}

\paragraph{Rodrigo} ~\\

\begin{figure}[H]
    \centering
    \includegraphics[height=8cm]{img/FM_tones-fm-results-Rodrigo-standards}
    \caption{Relative fluctuation strength as a function of modulation
        frequency by standard used for FM tones with $f_c = 1.5$ kHz,
        $d_f = 700$ Hz, SPL $= 70$ dB (error bars indicate standard deviation)}
\end{figure}

\begin{figure}[H]
    \centering
    \includegraphics[height=8cm]{img/FM_tones-fm-results-Rodrigo-comparison}
    \caption{Comparison between relative fluctuation strength as a function of
        modulation frequency for FM tones using the experimental data,
        \citeauthor{Fastl2007Psychoacoustics} data and the proposed model
        (error bars indicate standard deviation)}
\end{figure}

\end{document}
