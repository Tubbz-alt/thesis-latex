\documentclass[a4paper]{article}
\usepackage{mystyle}

\begin{document}

\customtitle{Fluctuation Strength Test Battery Results}

\section{Introduction}

This document presents the results obtained by validating our fluctuation
strength model against the tests contained in the test battery.

\section{Results}

\subsection{Reference Value}

Using the model implementation a value of 0.989692 vacil is obtained for the
reference tone.

\subsection{AM Tones}

\subsubsection{Modulation Frequency}

Figure \ref{fig:AMtonesfmplot} shows the relation between fluctuation strength
and modulation frequency, as estimated by the model.

\begin{figure}[ht]
    \centering
    \includegraphics[height=8cm]{img/am_tones_fm_plot}
    \caption{Fluctuation strength as a function of modulation frequency for AM
        tones with $f_c = 1 $kHz, $m_d = 40 $dB, SPL $= 70 $dB}
    \label{fig:AMtonesfmplot}
\end{figure}

\subsubsection{Carrier Frequency}

Figure \ref{fig:AMtonesfcplot} shows the relation between fluctuation strength
and carrier frequency, as estimated by the model.

\begin{figure}[ht]
    \centering
    \includegraphics[height=8cm]{img/am_tones_fc_plot}
    \caption{Fluctuation strength as a function of carrier frequency for AM
        tones with $f_m = 4 $Hz, $m_d = 40 $dB, SPL $= 70 $dB}
    \label{fig:AMtonesfcplot}
\end{figure}

\subsubsection{Sound Pressure Level}

Figure \ref{fig:AMtonesSPLplot} shows the relation between fluctuation strength
and sound pressure level, as estimated by the model.

\begin{figure}[ht]
    \centering
    \includegraphics[height=8cm]{img/am_tones_SPL_plot}
    \caption{Fluctuation strength as a function of sound pressure level for AM
        tones with $f_c = 1 $kHz, $f_m = 4 $Hz, $m_d = 40 $dB}
    \label{fig:AMtonesSPLplot}
\end{figure}

\subsubsection{Modulation Depth}

Figure \ref{fig:AMtonesmdplot} shows the relation between fluctuation strength
and modulation depth, as estimated by the model.

\begin{figure}[ht]
    \centering
    \includegraphics[height=8cm]{img/am_tones_md_plot}
    \caption{Fluctuation strength as a function of modulation depth for AM tones
        with $f_c = 1 $kHz, $f_m = 4 $Hz, $SPL = 70 $dB}
    \label{fig:AMtonesmdplot}
\end{figure}

\subsection{FM Tones}

\subsubsection{Modulation Frequency}

Figure \ref{fig:FMtonesfmplot} shows the relation between fluctuation strength
and modulation frequency, as estimated by the model.

\begin{figure}[ht]
    \centering
    \includegraphics[height=8cm]{img/fm_tones_fm_plot}
    \caption{Fluctuation strength as a function of modulation frequency for FM
        tones with $f_c = 1.5$ kHz, $d_f = 700$ Hz, SPL $= 70$ dB}
    \label{fig:FMtonesfmplot}
\end{figure}

\subsubsection{Carrier Frequency}

Figure \ref{fig:FMtonesfcplot} shows the relation between fluctuation strength
and carrier frequency, as estimated by the model.

\begin{figure}[ht]
    \centering
    \includegraphics[height=8cm]{img/fm_tones_fc_plot}
    \caption{Fluctuation strength as a function of carrier frequency for FM
        tones with $f_m = 4$ Hz, $d_f = 700$ Hz, SPL $= 70$ dB}
    \label{fig:FMtonesfcplot}
\end{figure}

\subsubsection{Sound Pressure Level}

Figure \ref{fig:FMtonesSPLplot} shows the relation between fluctuation strength
and sound pressure level, as estimated by the model.

\begin{figure}[ht]
    \centering
    \includegraphics[height=8cm]{img/fm_tones_SPL_plot}
    \caption{Fluctuation strength as a function of sound pressure level for FM
        tones with $f_c = 1.5$ kHz, $f_m = 4$ Hz, $d_f = 700$ Hz}
    \label{fig:FMtonesSPLplot}
\end{figure}

\subsubsection{Frequency Deviation}

Figure \ref{fig:FMtonesdfplot} shows the relation between fluctuation strength
and frequency deviation, as estimated by the model.

\begin{figure}[ht]
    \centering
    \includegraphics[height=8cm]{img/fm_tones_df_plot}
    \caption{Fluctuation strength as a function of deviation frequency for FM
        tones with $f_c = 1.5$ kHz, $f_m = 4$ Hz, $SPL = 70$ dB}
    \label{fig:FMtonesdfplot}
\end{figure}

\end{document}
