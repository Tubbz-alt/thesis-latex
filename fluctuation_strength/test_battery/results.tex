\documentclass[a4paper]{article}
\usepackage{mystyle}

\begin{document}

\customtitle{Fluctuation Strength Test Battery Results}

\section{Introduction}

This document presents the results obtained by validating our fluctuation
strength model against the tests contained in the test battery.

\section{Results}

\subsection{Reference Value}

Using the model implementation a value of 1.008473 vacil is obtained for the
reference tone.

\subsection{AM Tones}

\subsubsection{Modulation Frequency}

Figure \ref{fig:AMtonesfmplot} shows the relation between fluctuation strength
and modulation frequency, as estimated by the model.

\begin{figure}[ht]
    \centering
    \resizebox{!}{8cm}{
        % This file was created by matlab2tikz.
% Minimal pgfplots version: 1.3
%
%The latest updates can be retrieved from
%  http://www.mathworks.com/matlabcentral/fileexchange/22022-matlab2tikz
%where you can also make suggestions and rate matlab2tikz.
%
\begin{tikzpicture}

\begin{axis}[%
width=6.027778in,
height=4.754167in,
at={(1.011111in,0.641667in)},
scale only axis,
xmin=-1,
xmax=5,
xtick={-1,0,1,2,3,4,5},
xticklabels={{0.5},{1},{2},{4},{8},{16},{32}},
xlabel={$\text{f}_{\text{mod}}\text{ [Hz]}$},
ymin=0,
ymax=1.4,
ylabel={Fluctuation strength [vacil]}
]
\addplot [color=blue,dashed,mark=o,mark options={solid},forget plot]
  table[row sep=crcr]{%
-1	0.067305967550042\\
0	0.238582377746568\\
1	0.80303955794024\\
2	1.31808380343364\\
3	0.881822652571635\\
4	0.115745399813139\\
5	0.000376522248278075\\
};
\end{axis}
\end{tikzpicture}%
    }
    \caption{Fluctuation strength as a function of modulation frequency for AM
        tones with $f_c = 1 $ kHz, $m_d = 1$, SPL $= 70$.}
    \label{fig:AMtonesfmplot}
\end{figure}

\subsubsection{Carrier Frequency}

Figure \ref{fig:AMtonesfcplot} shows the relation between fluctuation strength
and carrier frequency, as estimated by the model.

\begin{figure}[ht]
    \centering
    \resizebox{!}{8cm}{
        % This file was created by matlab2tikz.
% Minimal pgfplots version: 1.3
%
%The latest updates can be retrieved from
%  http://www.mathworks.com/matlabcentral/fileexchange/22022-matlab2tikz
%where you can also make suggestions and rate matlab2tikz.
%
\begin{tikzpicture}

\begin{axis}[%
width=6.027778in,
height=4.754167in,
at={(1.011111in,0.641667in)},
scale only axis,
xmin=7,
xmax=13,
xtick={7,8,9,10,11,12,13},
xticklabels={{125},{250},{500},{1000},{2000},{4000},{8000}},
xlabel={$\text{f}_{\text{c}}\text{ [Hz]}$},
xmajorgrids,
ymin=0,
ymax=2,
ylabel={Fluctuation strength [vacil]},
ymajorgrids
]
\addplot [color=blue,dashed,mark=o,mark options={solid},forget plot]
  table[row sep=crcr]{%
6.96578428466209	0.86337891740161\\
7.96578428466209	1.08943588761812\\
8.96578428466209	1.16782792972356\\
9.96578428466209	1.25939581505104\\
10.9657842846621	1.17529945106717\\
11.9657842846621	1.1314316832654\\
12.9657842846621	0.994878741468494\\
};
\end{axis}
\end{tikzpicture}%
    }
    \caption{Fluctuation strength as a function of carrier frequency for AM
        tones with $f_m = 4 $ Hz, $m_d = 1$, SPL $= 70$.}
    \label{fig:AMtonesfcplot}
\end{figure}

\custombibliography

\end{document}
