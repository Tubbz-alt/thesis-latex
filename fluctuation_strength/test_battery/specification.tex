\documentclass[a4paper]{article}
\usepackage{mystyle}

\begin{document}

\customtitle{Fluctuation Strength Test Battery Specification}

\section{Introduction}

This document details a series of test that will be used to validate our
fluctuation strength model. These tests will be based on the reference value of
fluctuation strength and its the parametric dependencies. For all the tests a
sampling frequency $f_s = 44100 $ Hz will be used.

\section{Reference Value}

An amplitude-modulated (AM) tone ($f_c = 1$ kHz, $f_m = 4$ Hz, $m_d = 1$,
SPL $=60$ dB) is assigned a fluctuation strength of 1 vacil, and thus
constitutes the reference value. As so, the model implementation must comply
with this requisite.

\section{Tests}

This section details the specific tests that the model must pass to be
considered adequate. According to the type of input stimuli different tests
will be specified, starting with AM tones and then expanding to
frequency-modulated (FM) tones and AM broadband noise (BBN).

\subsection{AM Tones}

\subsubsection{Modulation Frequency}

The most characteristic attribute of fluctuation strength is its bandpass
response with regard to the variation of the modulation frequency (Figure
\ref{fig:flucstrenvmodfreq}b). As so, this particular plot will be reproduced
using the input stimuli whose parameters are shown on Listing
\ref{lst:amparamsfmplot}.

\begin{lstlisting}[
    style=MATLAB-editor,
    caption={AM tones parameters modulation frequency plot},
    label={lst:amparamsfmplot}
]
fc  = 1000; % [Hz]
md  = 40; % [dB]
SPL = 70; % [dB]
fs  = 44100; % [Hz]
N   = 262144; % samples
FMs = [0.50 1.00 2.00 4.00 8.00 16.00 32.00]; % [Hz]
\end{lstlisting}

\begin{figure}[ht]
    \centering
    \includegraphics[height=8cm]
        {Mueller2012Handbook/img/FluctuationStrengthVsModulationFrequency}
    \caption{Fluctuation strength of three modulated sounds as a function of
        modulation frequency. (a) Amplitude-modulated broad-band noise of 60-dB
        SPL and 40-dB modulation depth; (b) amplitude-modulated 1-kHz tone of
        70-dB SPL and 40-dB modulation depth; (c) frequency-modulated pure tone
        of 70-dB SPL, 1500-Hz center frequency and $\pm$ 700-Hz frequency
        deviation; \cite[pp. 248]{Fastl2007Psychoacoustics}}
    \label{fig:flucstrenvmodfreq}
\end{figure}

\subsection{Carrier Frequency}

In the implementation of \citeauthor{Schrader2002}, the parameter
\matlabinline{gzi} is used to model the dependency of roughness on the carrier
frequency. In order to evaluate whether the proposed values hold, and whether
it is necessary to adjust them, Figure \ref{fig:flucstrenvscfreq}a will be used.
The input stimuli parameters to be used are listed in Listing
\ref{lst:amparamsfcplot}.

\begin{lstlisting}[
    style=MATLAB-editor,
    caption={AM tones parameters carrier frequency plot},
    label={lst:amparamsfcplot}
]
FCs = [125 250 500 1000 2000 4000 8000]; % [Hz]
md  = 40; % [dB]
SPL = 70; % [dB]
fs  = 44100; % [Hz]
N   = 262144; % samples
fm  = 4; % [Hz]
\end{lstlisting}

\begin{figure}[ht]
    \centering
    \includegraphics[height=5cm]
        {Fastl2007Psychoacoustics/img/FluctuationStrengthVsCenterFrequency}
    \caption{Fluctuation strength of modulated tones as a function of
        frequency. (a) Amplitude-modulated pure tone of 70-dB SPL, 4-Hz
        modulation frequency and 40-dB modulation depth; (b)
        frequency-modulated pure tone of 70-dB SPL, 4-Hz modulation frequency,
        and ±200-Hz frequency deviation
        \cite[pp. 250]{Fastl2007Psychoacoustics}}
    \label{fig:flucstrenvscfreq}
\end{figure}

\subsection{Sound Pressure Level}

Figure \ref{fig:flucstrenvsndpreslvl} will be reproduced using the stimuli whose
parameters are shown in Listing \ref{lst:amparamssplplot};

\begin{lstlisting}[
    style=MATLAB-editor,
    caption={AM tones parameters sound pressure level plot},
    label={lst:amparamssplplot}
]
FCs     = 1000; % [Hz]
md      = 40; % [dB]
SPLs    = [50 60 70 80 90]; % [dB]
fs      = 44100; % [Hz]
N       = 262144; % samples
fm      = 4; % [Hz]
\end{lstlisting}

\begin{figure}[ht]
    \centering
    \includegraphics[height=5cm]
        {Fastl2007Psychoacoustics/img/FluctuationStrengthVsSoundPressureLevel}
    \caption{Fluctuation strength of modulated sounds as a function of sound
        pressure level. Stimulus parameters are the same as in Figure
        \ref{fig:flucstrenvmodfreq}, but the modulation frequency is 4Hz
        \cite[pp. 249]{Fastl2007Psychoacoustics}}
    \label{fig:flucstrenvsndpreslvl}
\end{figure}

\subsection{Modulation Depth}

Figure \ref{fig:flucstrenvsmoddep} will be reproduced using the stimuli whose
parameters are presented in Listing \ref{lst:amparamssmdplot}.

\begin{lstlisting}[
    style=MATLAB-editor,
    caption={AM tones parameters modulation depth plot},
    label={lst:amparamssmdplot}
]
FCs = 1000; % [Hz]
MDs = [1 2 4 6 8 12 16 20 40]; % [dB]
SPL = 70; % [dB]
fs  = 44100; % [Hz]
N   = 262144; % samples
fm  = 4; % [Hz]
\end{lstlisting}

\begin{figure}[ht]
    \centering
    \includegraphics[height=5cm]
        {Fastl2007Psychoacoustics/img/FluctuationStrengthVsModulationDepth}
    \caption{Fluctuation strength of two amplitude-modulated sounds as a
        function of modulation depth (or modulation factor). (a)
        Amplitude-modulated broadband noise of 60-dB SPL and 4-Hz modulation
        frequency; (b) amplitude-modulated 1-kHz tone of 70-dB SPL and 4-Hz
        modulation frequency \cite[pp. 249]{Fastl2007Psychoacoustics}}
    \label{fig:flucstrenvsmoddep}
\end{figure}

\custombibliography

\end{document}
