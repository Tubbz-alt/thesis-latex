\documentclass[a4paper]{article}
\usepackage{mystyle}

\begin{document}

\customtitle{Pilot Experiments Results}

\section{Introduction} % (fold)
\label{sec:introduction}

This documents presents the results obtained from the pilot experiments aimed
at validating the experimental procedure to be used in the real experiments.
The pilots were conducted using four participants, from which only one had prior
experience with the concept of fluctuation strength. Only amplitude-modulated
(AM) tones were used. From the results in can be concluded that overall the
concept of fluctuation strength was not understood correctly by the
participants. As so, modification to the experimental procedure are given in
order to accommodate the obtained results.

% section introduction (end)

\section{Results} % (fold)
\label{sec:results}

In the following figures the results from all the participants are shown; in the
appendix the individual results of each participant can be found. The error bars
indicate standard deviation.

\begin{figure}[ht!]
  \centering
  \includegraphics[height=10cm]{img/AM-fm-All-standards.eps}
  \caption{Relative fluctuation strength as a function of modulation frequency}
\label{fig:fm-all}
\end{figure}

\begin{figure}[ht!]
  \centering
  \includegraphics[height=10cm]{img/AM-fc-All-standards.eps}
  \caption{Relative fluctuation strength as a function of center frequency}
\label{fig:fc-all}
\end{figure}

\begin{figure}[ht!]
  \centering
  \includegraphics[height=10cm]{img/AM-SPL-All-standards.eps}
  \caption{Relative fluctuation strength as a function of sound pressure level}
\label{fig:SPL-all}
\end{figure}

\begin{figure}[ht!]
  \centering
  \includegraphics[height=10cm]{img/AM-md-All-standards.eps}
  \caption{Relative fluctuation strength as a function of modulation depth}
\label{fig:md-all}
\end{figure}

% section results (end)

\section{Discussion} % (fold)
\label{sec:discussion}

From all the obtained result curved, only the one corresponding to modulation
depth differs significantly from the one found in the literature. The other
curves are similar to the ones present in the literature, and therefore in this
case the procedure is considered to be adequate.

The main explanation arising from the fact that the obtained and expected curves
regarding modulation frequency differ so much is that the concept of fluctuation
strength was not clearly established during the training phase. One participant
commented that, in order to give an estimate of the fluctuation strength of the
sounds presented, he counted the cycles of the sounds. This explains the
obtained results, where the fluctuation strength increases as the modulation
frequency increases too. At this moment to possible solutions can be implemented
to deal with this problem.

\begin{enumerate}
  \item Add one of two additional AM tones with higher modulation frequencies
  (e.g., 64 Hz and 128 Hz). The idea behind this is to provide an anchor point,
  similar to the use of the pure tone in the low frequencies, that will make
  more clear the difference between fluctuation strength and roughness.
  \item Add FM tones to the training phase, in order to decouple the notion that
  the number of cycles or oscillations in the envelope of the AM tones
  correspond to the fluctuation strength sensation.
\end{enumerate}

Other than that, participants commented that they found confusing the scale
used in the experiments. They had some troubles understanding how to make a
judgment, and sometimes they confused the reference value (i.e., the second
sound was used as a reference instead of the first). Additionally, one
participant commented that it would be easier if numerical values were present,
instead of the textual indications.

In order to familiarize participants with the procedure of the experiment, a
small set of trials could be added to the training phase, to test whether the
way in which judgments are made is clear.

% section discussion (end)

\section{Conclusion} % (fold)
\label{sec:conclusion}

In order to obtain reliable data changes in the experimental procedure are
needed. After evaluating and deciding which of the presented changes in this
document will be implemented, a couple of pilot experiments using only the
modulation frequency section of the experiment will be conducted. If the
expected curves are obtained, then another couple of pilot experiments with
FM tones will conducted.

% section conclusion (end)

\clearpage

\appendix

\section{Individual Results}

\subsection{Modulation Frequency}

\subsubsection{Anuar}

\begin{figure}[H]
  \centering
  \includegraphics[height=8cm]{img/AM-fm-Anuar-standards.eps}
  \caption{Relative fluctuation strength as a function of modulation frequency}
\label{fig:fm-anuar}
\end{figure}

\subsubsection{Chitra}

\begin{figure}[H]
  \centering
  \includegraphics[height=8cm]{img/AM-fm-Chitra-standards.eps}
  \caption{Relative fluctuation strength as a function of modulation frequency}
\label{fig:fm-chitra}
\end{figure}

\subsubsection{Edgar}

\begin{figure}[H]
  \centering
  \includegraphics[height=8cm]{img/AM-fm-Edgar-standards.eps}
  \caption{Relative fluctuation strength as a function of modulation frequency}
\label{fig:fm-edgar}
\end{figure}

\subsubsection{Rodrigo}

\begin{figure}[H]
  \centering
  \includegraphics[height=8cm]{img/AM-fm-Rodrigo-standards.eps}
  \caption{Relative fluctuation strength as a function of modulation frequency}
\label{fig:fm-rodrigo}
\end{figure}

\subsection{Center Frequency}

\subsubsection{Anuar}

\begin{figure}[H]
  \centering
  \includegraphics[height=8cm]{img/AM-fc-Anuar-standards.eps}
  \caption{Relative fluctuation strength as a function of center frequency}
\label{fig:fc-anuar}
\end{figure}

\subsubsection{Chitra}

\begin{figure}[H]
  \centering
  \includegraphics[height=8cm]{img/AM-fc-Chitra-standards.eps}
  \caption{Relative fluctuation strength as a function of center frequency}
\label{fig:fc-chitra}
\end{figure}

\subsubsection{Edgar}

\begin{figure}[H]
  \centering
  \includegraphics[height=8cm]{img/AM-fc-Edgar-standards.eps}
  \caption{Relative fluctuation strength as a function of center frequency}
\label{fig:fc-edgar}
\end{figure}

\subsubsection{Rodrigo}

\begin{figure}[H]
  \centering
  \includegraphics[height=8cm]{img/AM-fc-Rodrigo-standards.eps}
  \caption{Relative fluctuation strength as a function of center frequency}
\label{fig:fc-rodrigo}
\end{figure}

\subsection{Sound Pressure Level}

\subsubsection{Anuar}

\begin{figure}[H]
  \centering
  \includegraphics[height=8cm]{img/AM-SPL-Anuar-standards.eps}
  \caption{Relative fluctuation strength as a function of sound pressure level}
\label{fig:SPL-anuar}
\end{figure}

\subsubsection{Chitra}

\begin{figure}[H]
  \centering
  \includegraphics[height=8cm]{img/AM-SPL-Chitra-standards.eps}
  \caption{Relative fluctuation strength as a function of sound pressure level}
\label{fig:SPL-chitra}
\end{figure}

\subsubsection{Edgar}

\begin{figure}[H]
  \centering
  \includegraphics[height=8cm]{img/AM-SPL-Edgar-standards.eps}
  \caption{Relative fluctuation strength as a function of sound pressure level}
\label{fig:SPL-edgar}
\end{figure}

\subsubsection{Rodrigo}

\begin{figure}[H]
  \centering
  \includegraphics[height=8cm]{img/AM-SPL-Rodrigo-standards.eps}
  \caption{Relative fluctuation strength as a function of sound pressure level}
\label{fig:SPL-rodrigo}
\end{figure}

\subsection{Modulation Depth}

\subsubsection{Anuar}

\begin{figure}[H]
  \centering
  \includegraphics[height=8cm]{img/AM-md-Anuar-standards.eps}
  \caption{Relative fluctuation strength as a function of modulation depth}
\label{fig:md-anuar}
\end{figure}

\subsubsection{Chitra}

\begin{figure}[H]
  \centering
  \includegraphics[height=8cm]{img/AM-md-Chitra-standards.eps}
  \caption{Relative fluctuation strength as a function of modulation depth}
\label{fig:md-chitra}
\end{figure}

\subsubsection{Edgar}

\begin{figure}[H]
  \centering
  \includegraphics[height=8cm]{img/AM-md-Edgar-standards.eps}
  \caption{Relative fluctuation strength as a function of modulation depth}
\label{fig:md-edgar}
\end{figure}

\subsubsection{Rodrigo}

\begin{figure}[H]
  \centering
  \includegraphics[height=8cm]{img/AM-md-Rodrigo-standards.eps}
  \caption{Relative fluctuation strength as a function of modulation depth}
\label{fig:md-rodrigo}
\end{figure}

\end{document}
