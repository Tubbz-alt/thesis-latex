\documentclass[a4paper]{article}
\usepackage{mystyle}

\begin{document}

\customtitle{Fluctuation Strength Experiment\textquotesingle s Training Phase}

\section{Introduction} % (fold)
\label{sec:introduction}

This document describes the experiment\textquotesingle s training phase, aimed
at making the concept of fluctuation strength clear to the participants.

% section introduction (end)

\section{Description} % (fold)
\label{sec:description}

The training phase consist on two phases, namely the
\hyperref[sub:reference_stimuli]{reference stimuli phase} and the
\hyperref[sub:continuous_stimuli]{continuous stimuli phase}.

\subsection{Reference Stimuli} % (fold)
\label{sub:reference_stimuli}

During this phase four stimuli will be used:

\begin{itemize}
  \item A pure sinusoidal tone
  \item A slightly fluctuating tone
  \item A decisively fluctuating tone
  \item A rough tone
\end{itemize}

Using these stimuli the following pairs will be presented to the participants:

\begin{itemize}
  \item The pure sinosoidal tone with the decisively fluctuating tone
  \item The slightly fluctuating tone with the decisively fluctuating tone
  \item The rough tone with the decisively fluctuating tone
\end{itemize}

After each pair the participants will be asked if they perceive the difference
in fluctuation between the stimuli.

% subsection reference_stimuli (end)

\subsection{Continuous Stimuli} % (fold)
\label{sub:continuous_stimuli}

During this phase, all the experiment\textquotesingle s stimuli will be
presented one after another, with a 800 ms silence between them. After this,
participants will be asked if they perceive a general difference in fluctuation
among stimuli.

% subsection continuous_stimuli (end)

% section description (end)

\end{document}
