\documentclass[a4paper]{article}
\usepackage{mystyle}

\begin{document}

\customtitle{Fluctuation Strength Experiment\textquotesingle s Training Phase}

\section{Introduction} % (fold)
\label{sec:introduction}

This document describes the experiment\textquotesingle s training phase, aimed
at making the concept of fluctuation strength clear to the participants.

% section introduction (end)

\section{Description} % (fold)
\label{sec:description}

The training phase consist on two phases, namely the
\hyperref[sub:reference_stimuli]{reference stimuli phase} and the
\hyperref[sub:continuous_stimuli]{continuous stimuli phase}. After each of these
phases, the participants will be asked if they perceive a difference in the
fluctuation for the presented stimuli. If so, the experiment will start.

\subsection{Reference Stimuli} % (fold)
\label{sub:reference_stimuli}

% subsection reference_stimuli (end)

\subsection{Continuous Stimuli} % (fold)
\label{sub:continuous_stimuli}

% subsection all_stimuli (end)

% section description (end)

\end{document}
