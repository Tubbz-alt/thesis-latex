\section{Scaling the Perceived Fluctuation Strength of Frequency-Modulated
    Tones}

This short paper by \citet{Wickelmaier2004Scaling} is structured in three
separate experiments. First, a full-factorial design with 54 pairs of stimuli,
with nine modulation frequencies and six modulation depths, and a magnitude
estimation test is run. The results differ from those reported by Zwicker's
fluctuation strength model, since they do not show the characteristic band-pass
response expected.

The second experiment tries to assess the contribution of the two factors used
in the past experiment (modulation frequency and modulation depth) separately,
by varying one while leaving the other constant. It was found that the variation
of the modulation depth conforms to Zwicker's model, while the variation of the
modulation frequency does not.

In the last experiment they try to assess whether fluctuation can be
represented by an additive combination of modulation frequency and modulation
depth. To test this they use the Thomsen condition, which results suggest that
listeners do not integrate these two variables as an unidimensional percept.

To conclude, the authors state that Zwicker's model does not conform properly
to their data, and that the status of fluctuation strength as a basic auditory
perceptual attribute could be debated.
