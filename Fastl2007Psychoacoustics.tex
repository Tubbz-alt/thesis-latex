\section{Psychoacoustics: Facts and models}

The book b \cite{Fastl2007Psychoacoustics} contains a chapter dedicated to
discuss fluctuation strength.

There are two kinds of perceptual effects that arise due to the modulation of
sounds: fluctuation strength and roughness. The soft limit (i.e.\ there is a
smooth transition between the sensations) occurs at approximately 20 Hz. For
frequencies below this limit is where the fluctuation strength perception
corresponds.

The reference value for the perceptual quantity for fluctuation of strength is
defined at 1 kHz, with a sound level pressure (SPL) of 60-dB and 100\% amplitude
modulated with a 4 Hz tone. This reference is quantified as having 1 vacil.

When comparing amplitude-modulated broad-band noise (AM BBN), \\
amplitude-modulated pure tones (AM SIN) and frequency-modulated pure tones
(FM SIN), the fluctuation strength against modulating frequency plot shows a
characteristic bandpass shape, with a maximum value around 4 Hz.

It seems that a relation between fluction strength and speech production exists,
as the normal production rate of syllables during normal conversation speed is
about 4 syllables/second.
