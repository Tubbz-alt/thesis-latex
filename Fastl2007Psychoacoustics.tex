\section{Psychoacoustics: Facts and models}

This book contains a chapter dedicated to discuss fluctuation strength and its
properties. First, several variables and their interactions with fluctuation
strength will be described; then, a model for fluctuation strength will be
presented.

\subsection{Dependencies of Fluctuation Strength}

It seems that a relation between fluction strength and speech production exists,
as the normal production rate of syllables during normal conversation speed is
about 4 syllables/second. This coincides with the frequency in which a maximum
value of fluctuation strength occurs (4 Hz).

Regarding sound pressure level (SPL) and fluctuation strength, figure
\ref{fig:flucstrenvsndpreslvl} shows their relation for two diffent stimuli
(tones and broad-band noise) and two different modulation techniques (amplitude
modulation and frequency modulation). An increase in SPL entails an increase of
fluctuation strength, this being stronger when using modulation of amplitude.

\begin{figure}
    \centering
    \includegraphics[height=5cm]
        {Fastl2007-FluctuationStrengthVsSoundPressureLevel}
    \caption{Fluctuation strength vs. sound pressure level for
        amplitude-modulated broad-band noise (a), an amplitude-moduled tone (b)
        and a frequency-modulated tone (c); modulation frequency of 4 Hz
        \cite[pp. 249]{Fastl2007Psychoacoustics}}
    \label{fig:flucstrenvsndpreslvl}
\end{figure}

Next the effect of modulation depth on fluctuation strength is analized, which
can be observed on figure \ref{fig:flucstrenvsmoddep}. It can be observer than,
between 3 dB and 30 dB the relation between fluctuation strength and modulation
depth is somewhat linear. After reaching a maximum value at around 30 dB, which
corresponds to a modulation factor of 94\%, fluctuation strength remains
constant with further increments of modulation depth.

\begin{figure}
    \centering
    \includegraphics[height=5cm]
        {Fastl2007-FluctuationStrengthVsModulationDepth}
    \caption{Fluctuation strength vs. modulation depth for amplitude-modulated
        broad-band noise of 60 dB SPL (a), and amplitude-modulated tone of 70 dB
        SPL and 1 kHz frequency; both with a modulation frequency of 4 Hz
        \cite[pp. 249]{Fastl2007Psychoacoustics}}
    \label{fig:flucstrenvsmoddep}
\end{figure}

Figure \ref{fig:flucstrenvscfreq} shows the relation between the center
frequency and fluctuation strength. Here a clear difference exists between the
type of modulation used; for amplitude-modulated tones although there is some
variability in the fluctuation strength it is small, whereas for
frequency-modulated tones there is a clear relation between both variables.
Before a center frequency of 1 kHz the fluctuation strength is almost constant,
experiencing later a linear decrease until it fades away.

\begin{figure}
    \centering
    \includegraphics[height=5cm]
        {img/Fastl2007-FluctuationStrengthVsCenterFrequency.jpg}
    \caption{Fluctuation strength vs. center frequency for an
        amplitude-modulated tone of 70 dB SPL, 4 Hz modulation frequency and 40
        dB modulation depth (a), and a frequency-modulated tone with 70 dB SPL,
        4 Hz modulation frequency and $\pm200$ Hz frequency deviation
        \cite[pp. 250]{Fastl2007Psychoacoustics}}
    \label{fig:flucstrenvscfreq}
\end{figure}

To understand why this change of fluctuation strength occurs in the case of the
frequency-modulated tones, it is necessary to take into account the excitation
patterns that the modulated sounds cause regarding auditory filters. For
instance, for a 0.5 kHz tone the frequency would vary between 300 and 700 Hz,
this corresponding to a 3.5 Bark interval. For a 8 kHz these values would be 7.8
kHz and 8.2 kHz, leading to a 0.2 Bark interval. The proportion between these
two is 17.5, which seems to be also the proportional between the relative
fluctuation strength for these two frequencies. Thus, this leads to the idea
that fluctuation strength can be explained in terms of the excitation patterns
that present themselves across the auditory filters.

Regarding fluctuation strength and frequency deviation, figure
\ref{fig:flucstrenvsfreqdev} depicts the relation between these two variables.
It can be seen that, after a frequency deviation of 20 Hz, there is a linear
increase of fluctuation strength with frequency.

\begin{figure}
    \centering
    \includegraphics[height=5cm]
        {Fastl2007-FluctuationStrengthVsFrequencyDeviation}
    \caption{Fluctuation strength vs. frequency deviation for a tone with 70 dB
        SPL, 1.5 kHz center frequency and a modulation frequency of 4 Hz
        \cite[pp. 251]{Fastl2007Psychoacoustics}}
    \label{fig:flucstrenvsfreqdev}
\end{figure}

It is to note that, regarding modulation index, it seems that a significant
fluctuation strength (10\% of the relative fluctuation strength, as defined by
the authors) is achieved with a modulation index that would correspond to about
10 times to JND for modulation frequency. This would relate both the thresholds
of modulation frequency and fluctuation strength.

Not only modulated sounds can cause fluctuation strength, but also unmodulated
narrow-band noise. Figure \ref{fig:flucstrenvsbandwith} illustrates this, where
it can be seen that also at a 4 Hz frequency the maximum value for fluctuation
strength occurs. Regarding the SPL it behaves similarly as the other sound
sources.

\begin{figure}
    \centering
    \includegraphics[height=5cm]
        {Fastl2007-FluctuationStrengthvsBandwidth}
    \caption{Fluctuation strength vs bandwidth for unmodulated narrow-band noise
        with 70 dB SPL and a center frequency of 1 kHz
        \cite[pp. 252]{Fastl2007Psychoacoustics}}
    \label{fig:flucstrenvsbandwith}
\end{figure}

Figure \ref{fig:flucstrensnds} compares the fluctuation strength of several
sounds, which physical characteristics are described in Table
\ref{tab:flucstrensnds}. The sounds that present the largest fluctuation
strength (1 and 2) can be related by the fact that they excite a large range of
frequencies taking into account the auditory filters. As so, it can be said that
fluctuation strength sums across critical bands.

\begin{figure}
    \centering
    \includegraphics[height=5cm]
        {img/Fastl2007-FluctuationStrengthSounds.jpg}
    \caption{Fluctuation strength of sounds 1--5 as described in Table
        \ref{tab:flucstrensnds} \cite[pp. 252]{Fastl2007Psychoacoustics}}
    \label{fig:flucstrensnds}
\end{figure}

\begin{table}
    \centering
    \begin{tabu}{ l r r r r r }
        \hline
        Sound & 1 & 2 & 3 & 4 & 5 \\\hline
        Abbreviation & FM & AM & AM & FM & \\
        & SIN & BBN & SIN & SIN & NBN \\
        Frequency [Hz] & 1500 & --- & 2000 & 1500 & 1000 \\
        Level [dB] & 70 & 60 & 70 & 70 & 70 \\
        Modulation frequency [Hz] & 4 & 4 & 4 & 4 & --- \\
        Modulation depth [dB] & --- & 40 & 40 & --- & --- \\
        Frequency deviation [Hz] & 700 & --- & --- & 32 & --- \\
        Bandwidth [Hz] & --- & 16000 & --- & --- & 10 \\\hline
    \end{tabu}
    \caption{Physical data of sounds 1--5
        \cite[pp. 253]{Fastl2007Psychoacoustics}}
    \label{tab:flucstrensnds}
\end{table}

\subsection{Model of Fluctuation Strength}

A model basic on the temporal variation of a masking pattern in shown on Figure
\ref{fig:flucstrenmodel}, where the temporal variation of the amplitude of the
masker, also called temporal masking depth, is denoted by the magnitude
$\Delta L$. The inverse of the time difference between peak corresponds to the
modulation frequency $f_{mod}$.

\begin{figure}
    \centering
    \includegraphics[height=5cm]
        {Fastl2007-FluctuationStrengthModel}
    \caption{Model of fluctuation strength
        \cite[pp. 254]{Fastl2007Psychoacoustics}}
    \label{fig:flucstrenmodel}
\end{figure}

Equation \ref{eq:flucstrentempmaskmodfreq} shows the relationship between
fluctuation strength $F$, temporal masking depth $\Delta L$ and modulation
frequency $f_{mod}$, where the importance of the 4 Hz frequency is emphasized.

\begin{equation}
    F \sim \frac{\Delta L}{(f_{mod}/4\text{ Hz}) + (4\text{ Hz}/f_{mod})}
    \label{eq:flucstrentempmaskmodfreq}
\end{equation}

It is to note that there is dependency between the temporal masking depth
$\Delta L$ and the modulation frequency depending on the type of stimuli. In the
case of broad-band noise, temporal masking depth seems largely unaffected by 
modulation frequency, whereas amplitude and frequency modulates tones these two
variable are dependant on each other, this being strongest on the frequency
modulated case. In order to address this, when modelling fluctuation strength
for these tones not a single $\Delta L$ value is taken, but instead it is
integrated across the critical-band rate scale.

Figure \ref{eq:flucstrentempmaskmodfreq} shows the resulting temporal masking
pattern for several values of modulation frequency. It can be seen that, as the
modulation frequency increases, the temporal masking depth decreases. This leads
to the idea that, although fluctuation strength presents a bandpass response
with respect to modulation frequency, the temporal masking suffers from a
lowpass effect. It can be considered that the temporal masking depth decreases
linearly with modulation frequency.

\begin{figure}
    \centering
    \includegraphics[height=5cm]
        {Fastl2007-FluctuationStrengthTemporalMasking}
    \caption{Temporal masking pattern for an amplitude-modulated broad-band
        noise \cite[pp. 255]{Fastl2007Psychoacoustics}}
    \label{fig:flucstrenmasking}
\end{figure}

Taking all these into account, equation \ref{eq:flucstrenexbbn} presents an
updated model, in which $m$ is the modulation factor, and $L_{BBN}$ is the level
of broad-band noise. Similarly, equation \ref{eq:flucstrenexamfm} presents the
equivalent for the tone signals, where the temporal masking depth is integrated
across the auditory filters.

\begin{equation}
    F_{BBN} = \frac{5.8(1.25m-0.25)[0.05(L_{BBN}/\text{dB})-1]}
        {(f_{mod}/5\text{ Hz})^2+(4\text{ Hz}/f_{mod})+1.5} \text{ vacil}
    \label{eq:flucstrenexbbn}
\end{equation}

\begin{equation}
    F = \frac{0.008 \int_0^{24\text{ Bark}}(\Delta L/\text{dB Bark})\mathrm{d}z}
        {(f_{mod}/4\text{ Hz})+(4\text{ Hz}/f_{mod})} \text{ vacil}
    \label{eq:flucstrenexamfm}
\end{equation}




