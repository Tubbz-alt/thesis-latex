\section{Hearing: An introduction to psychological and physiological acoustics}

The book by Gelfand \cite{gelfand2009hearing} devotes an entire chapter to cover
psychoacoustic methods.

Absolute sensitivity accounts for the minimum level in which a stimulus is
detected. Differential sensitivity describes the extend to which changes between
stimuli can be detected. Sensory capability refers to what people actually hear,
and response proclivity refer to how people react to sounds.

\subsection{Classic Methods of Measurement}

\subsubsection{Method of Limits}

This method is used to determine the absolute sensitivity or the threshold for a
given sound characteristics. The experimenter is in control of the stimuli, and
the participant must state whether the stimuli characteristic is detected or
not. Several runs (presentations of stimuli with the physical characteristic
increasing or decreasing between presentations), alternating between ascending
and descending, are presented to the participants until the stimuli changes
status (detected to undetected, or the other way around). At the end, the mean
value of all the estimated thresholds is taken as the estimated global
threshold.

Responses biases may arise due to the anticipation of the threshold by
participants (since the run direction is known), or by habituation to the
stimuli, in this case reporting higher or lower values than the real threshold.

Step size between the stimuli can affect speed and reliability of the results.
A bigger step size allows faster experiments, but it is unreliable as it could
be not sensitive enough to detect the actual threshold. Smaller step sizes are
more effective in this regard but take more time of the experiment.

This method can also be used to determine differential threshold. In this case,
pairs of stimuli are presented, and the participant must state whether the
stimuli are equal or if one is greater than the other. One stimulus is keep as a
reference, while the other is varied, both in an ascending and in a descending
way. This gives three possible ranges of responses:
\begin{inparaenum}[(i)]
    \item stimulus A is greater than,
    \item stimulus B, stimuli are equal,
    \item stimulus A is lesser than stimulus B.
\end{inparaenum}
The range in which the stimuli are considered equal is called the interval of
uncertainty, and the just noticeable difference (JND) is usually taken as half
of this value.

\subsubsection{Method of Adjustment}

In this method the participant is in control of the stimulus, usually via a
continuous dial or knob. Then the participant must adjust the sound until it is
audible or inaudible, according to the stimulus starting point. The threshold is
then taken as the mean value of both values. In a differential setting, the
participant must adjust the stimulus strength until it matches the reference
stimulus strength.

This method can suffer from the persistence of stimulus bias, which results in
lower threshold when going in an downward direction and higher threshold when
going in an upward direction. To counteract this, both types of runs must be
used.

\subsubsection{Method of Constant Stimuli}

Similar to the method of limits, but in this case stimuli are presented in a
random order to participants. Then participants must state whether the stimulus
was detected or not. From the responses of all participants the psychometric
function can be obtained. The point where 50\% of the responses is achieved is
considered then the threshold.

In the case of a differential setting, pairs of stimuli are presented and the
participants must state whether the second stimulus was stronger or weaker than
the first (reference) stimulus. The point were the responses are at 50\% is
considered the point of subjective equality (both stimuli are perceived the
same), and the difference between the 50\% and the 75\% is the JND.

This method allows the placement of so-called ``catch trials'', i.e.\ trials
where there is no second stimulus present. This allows to control for guessing,
and reduces responses biases.

This method is more accurate than the other two methods, although is it usually
longer in duration. This can impact negatively participants, leading up to
fatigue and making it difficult for them to maintain motivation during the
experiment.

\subsection{Adaptive Procedures}

Adaptive procedures presents stimuli to the participants in one direction until
there is a change in participants answer; at that point then direction is
reversed until another change of answer is reached. It is said that adaptive
procedures converge to the threshold point, and make better use of trials by
placing most of them near the threshold itself.

\subsubsection{Bekesy's Tracking Method}

The direction of the change of the stimulus characteristic under study is under
control of the participant. For instance, in the case of loudness, the stimulus
begins with an inaudible level that is gradually incremented. As soon as the
participant starts hearing the stimulus he or she can push the button, that will
change the direction of the change to a decreasing one. While the button is
pressed, the change will continue downward. When the sound becomes inaudible
again, the participant can release the button, and the sound will start going
up again. In this way, the threshold can be tracked by the participant. This
procedure can lead to response biases in the cases which the stimulus change
rate is faster than the participant response rate.

\subsubsection{Simple Up-Down or Staircase Method}

This procedure is similar to the Bekesy's Tracking method, but in a discrete
way. Stimuli are presented sequentially either in a downward or upward direction
until a answer reversal occurs. In that case the direction is then reversed.
After six to eight changes of direction the method ends, and the 50\% point
value is calculated either with the mean value of the midpoints between changes
or with the mean value of all the peaks and troughs. As the stimuli are
presented in a sequential way, this can lead to response biases from the
participants, as they anticipate the stimuli that will be presented to them.
Also it is dependent of the step size, as the method of limits.

\subsubsection{PEST Procedure}

The parameter estimation by sequential testing (PEST) procedure is an adaptive
procedure that involves changes in direction and step size. When two consecutive
trials result in the same answer the step size is doubled. When a change of
answer occurs the step size of halved. When two consecutive changes of direction
occur the threshold is taken as the midpoint between these two values.

\subsubsection{BUDTIF Procedure}

The block up-down temporal interval forced-choice (BUDTIF) procedure is similar
to the PEST procedure. In this case blocks of stimuli are used instead of
individual ones. Also, instead of a yes-no response, a two-alternative forced
choice is used. Finally, it is possible to determine points in the psychometric
function other than 50\% by modifying the number of trials per block. For
instance, for a 75\% rate a block of 4 stimuli can be used, where if 3 out of 4
are classified correctly then the threshold is considered as the obtained value.
Otherwise the same procedure as the PEST is followed to alter the direction of
the trials, decreasing the stimuli intensity when four corrects responses are
obtained, and increasing them when there are less than three correct answers.

A variation of this procedure which utilizes a yes-no response also exists,
named the block up-down yes-no (BUDYEN) method. However, this method has the
disadvantage of not allowing to estimate false alarm responses, as a result of
the fact that pairs of stimuli are not used.

\subsubsection{Transformed Up-Down Procedures}

Simple Up-Down procedure converges only to the 50\% point in the psychometric
curves. By altering the criteria for increasing the stimuli level (up rule) and
decreasing the stimuli level (down rule), other points can be achieved. For
instance, by only decreasing the stimuli level when two consecutive positive
answer occur, the point to which the procedure converges is 70.7\%.

\subsection{Scales of Measurement}

Nominal scales refer to categorization of stimuli according to groups. Ordinal
scales imply that the values of the scale can be ranked, but a distance between
these values is not given. Interval scale specify both an order and a distance
among members of the scale; however, they do not specify a reference value or
zero. Lastly, ratio scale include a reference value, allowing to specify values
as ratios, expanding the amount of information conveyed by them.

There also exist three classes of scaling procedures. Discriminability (or
confusion) scales are generated by asking individuals to determine differences
between stimuli. Category (or partition) scales present the participant with the
task of dividing the stimuli range into equally spaced categories. Magnitude (or
ratio) scales ask individuals for an estimation of the ratio between two
stimuli.

\subsection{Direct Scaling}

\subsubsection{Ratio Estimation and Production}

Ratio estimation ask individuals to provide the relationship between to stimuli
in terms of the ratio between them. Ratio production allows individuals to
modify a given stimulus that must conform to a specific ratio, taking into
account the provided reference stimulus.

\subsubsection{Magnitude Estimation and Production}

Similar to the ratio estimation procedure, magnitude estimation ask individuals
to estimate a stimulus level taking as a reference an anchor value, usually
called a modulus. The modulus is assigned an specific value, for instance 10,
and the participant must state according to this reference value which would be
the value for the presented stimulus. For instance, a value of 60 would indicate
that the non-reference value is 6 times bigger than the reference value. An
alternative approach is to not include the modulus, in this case all the stimuli
are presented and the participant must assigned values to them taking into
account all of them. Magnitude production applies the same procedure but now the
participant must modify the non-reference stimulus level to comply with the
presented magnitude ratio.

Absolute magnitude estimation (AME) and absolute magnitude production (AMP)
do not include the reference value at all, neither they present stimuli in such
a way that a past stimulus can be used to assign the magnitude of a new one.
