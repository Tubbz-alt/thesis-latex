\section{Psychoacoustics and Sound Quality}

This chapter from the book Communications Acoustics
\cite{Blauert2005Communication} deals with important concepts related to
psychoacoustics and sound quality. The second section of the chapter describes
four of the most common methods used in psychoacoustics.

\subsection{The Ranking Procedure ``Random Access''}

The ``Random Access'' method is a ranking procedure, i.e., sound samples are to
be ordered or ranked according to some criteria, like for instance their sound
quality. In the case of the ``Random Access'' procedure, for a given set of
sounds the participants are able to listen to the samples as many times as they
want. Because of the fact that participants have total freedom on the listening
of sounds, it is the preferred method for assessing sound quality nowadays.

\subsection{The Semantic Differential}

The semantical differential method is used to assess if the perceived meaning of
a given sound is the one intended. List of adjectives pairs describing possible
characteristics of the sounds are given in order to achieve this. The
participant must then indicate which of the two attributes describe the sound
better, and to what extend.

\subsection{Category Scaling}

This is the preferred method for assessing loudness. Seven-step scales (composed
of: very soft, soft, slightly soft, neither soft nor loud, slightly loud, loud
and very loud) and five-step scales (as the seven-step but without slightly soft
and slightly loud) are usually used. There also exists a 50-point scales that
uses numerical indications, used mostly in audiology and noise-immission
assessments.

\subsection{Magnitude Estimation}

In this method several presentations of pairs of sounds are used, composed of an
anchor sound, that remains unchanged, and a test sound ,that is varied across
trials. A reference value, for example 100, is assigned to one perceptual
characteristic of the anchor sound, for instance its loudness. Is then the task
of the participants to estimate the ratio of the perceptual attribute of the
test and the anchor sound. For instance the test sound loudness is 80 percent of
the anchor sound loudness. Intra-individual and inter-individual differences
usually come out in a 10 percent variation range. The anchor sound can influence
the results significantly; therefore it is recommended to use at least two of
them.

\paragraph{Summary} ~\\
All the methods reviewed have advantages and disadvantages. The ``Random
Access'' and the semantical differential methods constitute more ``qualitative''
information, whereas the category scaling and magnitude estimation provide more
``quantitative data''. With the latter methods one can perform statistical
analyses as well.
