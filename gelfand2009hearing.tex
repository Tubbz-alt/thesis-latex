\section{Hearing: An introduction to psychological and physiological acoustics}

The book by Gelfand \cite{gelfand2009hearing} devotes an entire chapter to cover
psychoacoustic methods.

Absolute sensitivity accounts for the minimum level in which a stimulus is
detected. Differential sensitivity describes the extend to which changes between
stimuli can be detected. Sensory capability refers to what people actually hear,
and response proclivity refer to how people react to sounds.

\subsection{Classic Methods of Measurement}

\subsubsection{Method of Limits}

This method is used to determine the absolute sensitivity or the threshold for a
given sound characteristics. The experimenter is in control of the stimuli, and
the participant must state whether the stimuli characteristic is detected or
not. Several runs (presentations of stimuli with the physical characteristic
increasing or decreasing between presentations), alternating between ascending
and descending, are presented to the participants until the stimuli changes
status (detected to undetected, or the other way around). At the end, the mean
value of all the estimated thresholds is taken as the estimated global
threshold.

Responses biases may arise due to the anticipation of the threshold by
participants (since the run direction is known), or by habituation to the
stimuli, in this case reporting higher or lower values than the real threshold.

Step size between the stimuli can affect speed and reliability of the results.
A bigger step size allows faster experiments, but it is unreliable as it could
be not sensitive enough to detect the actual threshold. Smaller step sizes are
more effective in this regard but take more time of the experiment.

This method can also be used to determine differential threshold. In this case,
pairs of stimuli are presented, and the participant must state whether the
stimuli are equal or if one is greater than the other. One stimulus is keep as a
reference, while the other is varied, both in an ascending and in a descending
way. This gives three possible ranges of responses:
\begin{inparaenum}[(i)]
    \item stimulus A is greater than,
    \item stimulus B, stimuli are equal,
    \item stimulus A is lesser than stimulus B.
\end{inparaenum}
The range in which the stimuli are considered equal is called the interval of
uncertainty, and the just noticeable difference (JND) is usually taken as half
of this value.

\subsubsection{Method of Adjustment}

In this method the participant is in control of the stimulus, usually via a
continuous dial or knob. Then the participant must adjust the sound until it is
audible or inaudible, according to the stimulus starting point. The threshold is
then taken as the mean value of both values. In a differential setting, the
participant must adjust the stimulus strength until it matches the reference
stimulus strength.

This method can suffer from the persistence of stimulus bias, which results in
lower threshold when going in an downward direction and higher threshold when
going in an upward direction. To counteract this, both types of runs must be
used.

\subsubsection{Method of Constant Stimuli}

Similar to the method of limits, but in this case stimuli are presented in a
random order to participants. Then participants must state whether the stimulus
was detected or not. From the responses of all participants the psychometric
function can be obtained. The point where 50\% of the responses is achieved is
considered then the threshold.

In the case of a differential setting, pairs of stimuli are presented and the
participants must state whether the second stimulus was stronger or weaker than
the first (reference) stimulus. The point were the responses are at 50\% is
considered the point of subjective equality (both stimuli are perceived the
same), and the difference between the 50\% and the 75\% is the JND.

This method allows the placement of so-called ``catch trials'', i.e.\ trials
where there is no second stimulus present. This allows to control for guessing,
and reduces responses biases.

This method is more accurate than the other two methods, although is it usually
longer in duration. This can impact negatively participants, leading up to
fatigue and making it difficult for them to maintain motivation during the
experiment.
