\section{Handbook of Engineering Acoustics}

This book contains a chapter about the effect of sound of humans. First a small
overview of the perceptual processes and the associated perceptual quantities is
given \cite[p.~72]{Mueller2012Handbook}. \ref{tab:stimsens} illustrates
all the perceptual measures, along with their dominant physical stimuli.

\begin{table}
    \centering
    \begin{tabu}{ l l }
        \tabucline[1pt]{-}
        Dominant stimuli & Cognitive parameters \\\tabucline[1pt]{-}
        Sound pressure level (dB) & Loudness (sone) \\\cline{2-2}
        & Loudness level (phon) \\\hline
        Frequency (Hz) & Critical band rate (Bark) \\\cline{2-2}
        & Ratio pitch (mel) \\\hline
        Degree of modulation (\%) & Roughness (asper)\\\cline{1-1}
        Modulation frequency (Hz) & \\\hline
        Frequency (Hz) & Sharpness (acum) \\\hline
        Degree of modulation (\%) & Fluctuation strength (vacil) \\\cline{1-1}
        Modulation frequency (Hz) & \\\hline
        Spectral components (Pa) & Pitch strength \\\cline{2-2}
        & Tonality (tu) \\\hline
        Impulse duration (s) & Subjective duration of impetus (IU) \\\hline
        Sound pressure level (dB) & Density (dasy) \\
        Frequency (Hz) & \\\tabucline[1pt]{-}
    \end{tabu}
    \caption{Stimuli and sensations}
    \label{tab:stimsens}
\end{table}
